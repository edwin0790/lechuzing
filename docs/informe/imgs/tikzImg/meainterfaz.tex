\documentclass[11pt]{standalone}
%CORREGIR VHDL ORIGINAL

\usepackage[spanish]{babel}
\usepackage{amsmath}
\usepackage{amssymb}
\usepackage{tikz}
\usepackage{siunitx}

\usetikzlibrary{arrows,backgrounds,fit,positioning,petri,babel,shapes}

%\tikzstyle{style} = [definition]
\tikzstyle{interior}=[rectangle,rounded corners,draw=black,minimum size=40,fill=white]
\tikzstyle{core}=[interior]
\tikzstyle{perif} = [interior,minimum height=20]
\tikzstyle{exterior} = [rectangle,draw=black,minimum size=40]
\tikzstyle{contenedor} = [rectangle,draw=black]
\tikzstyle{buf}=[core, text width=55, align=center]
\tikzstyle{obuf} = [buf, node distance=.7]
\tikzstyle{env} = [fill=black!20]
\tikzstyle{simple}=[rectangle, draw=black, minimum height=220, text width=65,align=center]
\tikzstyle{mode text} = [midway,sloped,text width=200,align=center]
\tikzstyle{moore} = [rectangle,draw=black, minimum height= 30,text width=85,align=left]
\tikzstyle{mealy} = [rectangle,rounded corners=15, draw=black, text width = 80,align=left,minimum height=40]
\tikzstyle{ask} = [diamond,text width=70,draw=black,align=center,]

\newcommand{\epg}[3]{
	Buffer {#1}\\
	[44pt]EP{#2}\\
	[44pt]{#3} Bytes}

\newcommand{\ep}[3]{
		Buffer {#1}\\
		[8pt]EP{#2}\\
		[8pt]{#3} Bytes}


\begin{document}
	\begin{tikzpicture}[scale=1]
		\begin{scope}[transform shape,node distance=1,>=latex,thick]
			\node[moore]	(idle)	[]	{--idle:\\SLRD='1';\\SLOE='1';\\SLWR='1';\\FIFOADR="ZZ";};
			\node[ask]	(pr1)	[below=of idle]	{EP2\_empty='0'}
				edge[<-] (idle);
			\node[ask]	(pr2)	[right=of pr1]	{write\_req='1'};
			\node[ask]	(pr3)	[below=of pr2]	{EP8\_full='0'};
			\node[moore]	(radr)	[left=of pr1]		{--read address:\\SLRD='1';\\SLOE='1';\\SLWR='1';\\FIFOADR="00";};

			\draw[->](pr1) -- node [above,near start]{No}(radr);
			\draw[->](pr1) -- node [above,near start]{Si} (pr2);
			
			\node[node distance=0.8](aux1)[right=of pr2]{};
			\draw[->](pr2) -- node[left,near start]{Si}(pr3);
			\draw[->](pr2.east) |- node[above,near end]{No}(aux1.base);

			\node[moore]	(rnempty)	[left=of radr]	{--read no empty:\\SLRD='0';\\SLOE='0';\\SLWR='1';\\FIFOADR="00";}
				edge[<-](radr);
			\node[moore](rread)[below=of rnempty]{--read read:\\SLRD='1';\\SLOE='0';\\SLWR='1;\\FIFOADR="00";}
				edge[<-](rnempty);
				
			\node[moore](wadr)	[left=of pr3]	{--write address:\\SLRD='1';\\SLOE='1';\\SLWR='1';\\FIFOADR="11";};
			\node[node distance=.8](aux2)[right=of pr3]{};
			\draw[->](pr3) -- node[above,near start]{Si} (wadr);
			\draw[->](pr3) -- node[above,near start]{No}(aux2.base) -| (aux1.base);

			\node[moore](wnfull)[below=of wadr] {--write no full:\\SLRD='1';\\SLOE='1';\\SLWR='0';\\FIFOADR="11";}
				edge[<-] (wadr);
			\node[node distance=.5](aux3)[left=of wnfull]{};
			\draw[->] (wnfull) -- (aux3.base);
			\node[moore,node distance=.5](wwrite)[left=of aux3] {--write write:\\SLRD='1';\\SLOE='1';\\SLWR='1';\\FIFOADR="11";}
				edge[<-] (aux3.base);
				
			\node[ask](pr4)[below=of wwrite] {fifo\_valid='1'}
				edge[<-] (wwrite);
			\node[moore](wend)[left=of pr4] {--write end:\\SLRD='1';\\SLOE='1';\\SLWR='1';\\FIFOADR="11";};
			\node[node distance=.5] (aux4) [right=of pr4]{}; 
			\draw[->]	(pr4) -- node[above,near start]{Si} (wend);
			\draw[->](pr4) -- node[above] {No} (aux4.base);
			\draw[->](aux4.base) -- (aux3.base);

			\node[ask](pr5)[below=of wend] {write\_req='1'}
				edge[<-] (wend);
			\node[ask](pr6)[right=of pr5] {EP2\_empty='0'};
			\node[node distance=.8] (aux5) [left=of rread] {};
			\node[node distance=.8] (aux7) [left=of pr5] {};
			\draw[->] (rread) -- (aux5.base);
			\draw[->] (pr5)	-- node[above,near start] {No} (aux7.base);
			\draw[->] (pr5) -- node[above,near start] {Si} (pr6);
			
			\node[node distance=.8] (aux6)[below=of pr6] {};
			\draw[->] (pr6) -| node[above,near start] {Si} (wnfull);
			\draw[->] (pr6) -- node[left]{No} (aux6.base) -| (aux7.base);
			\draw[->] (aux7.base) |- (aux5.base);
			
			\node[node distance=.8] (aux8) [above=of idle] {};
			\draw[->] (aux8.base) -- (idle);
			\draw[->] (aux5.base) |- (aux8.base);
			\draw[->](aux1.base) |- (aux8.base);
			
		\end{scope}
	\end{tikzpicture}
\end{document}