\documentclass{standalone}

\usepackage{tikz,pgf}
\usetikzlibrary{positioning,circuits.logic.mux,circuits.logic.US,shapes.gates.logic}

\begin{document}
	
	\begin{tikzpicture}[circuit logic US, node distance=1,shape border uses incircle]
		\begin{scope}
			\node[mux, inputs={nn},info={[text width=10]center:0\\1}] (ifcfg4){};
			\node[buffer gate] (ifcfg5) [right=of ifcfg4] {};
			\node[not gate]	(a)		[left=of ifcfg4.input 2]	{};
			\node[node distance=2](aux a)	[left=of ifcfg4.input 1]	{};
			\node[] (aux 1) [left=of aux a] {};
			\node[mux, inputs={nn},info={[text width=10]center:0\\1}] (ifcfg6) [left=of aux 1]{};
			\node[mux, inputs={nn},info={[text width=10]center:1\\0},point left,node distance=4] (ifcfg7) [below=of ifcfg6]{};
			\node[mux,inputs={nn},info={[text width=10]center:0\\1},node distance=2,point left] (ifcfg4bis) [right=of ifcfg7.input 1]{};
			\node[not gate,node distance=1.7] (b) [right=of ifcfg4bis.input 1,point left] {};
			\node[node distance=3] (aux b) [right=of ifcfg4bis.input 2] {};
			\node[buffer gate,point left] (c)	[right=of aux b]	{};
			
			\node[]	(sel4)	[below=of ifcfg4]	{IFCFG.4};
		%	\node[node distance=1.5] (sel4bis) [above=of ifcfg4bis] {IFCFG.4};
			\node[]	(sel5)	[above=of ifcfg5]	{IFCFG.5};
			\node[] (sel6)	[below=of ifcfg6]	{IFCFG.6};
			\node[node distance=1.5]	(sel7)	[above=of ifcfg7]	{IFCFG.7};
			\node[] (aux 2) [right=of ifcfg5]		{};
			\node[node distance=1.5] (aux 3)  [below=of aux 2] {};
			\node[align=center] (output) [right=of aux 3] {Pin\\IFCLK};
			\node[] (freq1)	[left=of ifcfg6.input 1]	{30 MHz};
			\node[]	(freq2) [left=of ifcfg6.input 2]	{48 MHz};
			\node[align=center] (clkin) [left=of ifcfg7]	{Entrada\\IFCLK\\Interna};
			
			\draw (ifcfg4) to (ifcfg5);
			\draw (aux a.base) to (ifcfg4.input 1);
			\draw (aux a.base) |- (a.input);
			\draw (ifcfg6) to (aux a.base);
			\draw (aux 1.base) |- (ifcfg7.input 2);
			\draw (ifcfg4bis) to (ifcfg7.input 1);
			\draw (c) to (ifcfg4bis.input 2);
			\draw (b) to (ifcfg4bis.input 1);
			\draw (aux b.base) |- (b.input);
			\draw (a) to (ifcfg4.input 2);
			\draw (ifcfg5) to (aux 2.base);
			\draw (aux 2.base) |- (c);
			\draw (aux 3.base) -- (output);
			
			\draw (clkin) to (ifcfg7);
			\draw (freq1) to (ifcfg6.input 1);
			\draw (freq2) to (ifcfg6.input 2);
			\draw (sel5) to (ifcfg5.north);
			\draw (sel4) to (ifcfg4.select);
			\draw (sel4) -| (ifcfg4bis.select);
			\draw (sel6) to (ifcfg6.select);
			\draw (sel7) to (ifcfg7.select);	
		\end{scope}
	\end{tikzpicture}
	
\end{document}
%\documentclass[11pt]{article}
%
%\usepackage[a3paper]{geometry}
%\usepackage[spanish]{babel}
%\usepackage{amsmath}
%\usepackage{amssymb}
%\usepackage{tikz}
%\usepackage{siunitx}
%
%\usetikzlibrary{arrows,backgrounds,fit,positioning,petri,babel,shapes,circuits.logic.US}
%
%%\tikzstyle{style} = [definition]
%\tikzstyle{interior}=[
%	rectangle,
%	rounded corners,
%	draw=black,
%	minimum size=40,
%	fill=white]
%\tikzstyle{core}=[
%	interior]
%\tikzstyle{perif}=[
%	interior,
%	minimum height=20]
%\tikzstyle{exterior}=[
%	rectangle,
%	draw=black,
%	minimum size=40]
%\tikzstyle{contenedor}=[
%	rectangle,
%	draw=black]
%\tikzstyle{buf}=[
%	core,
%	text width=55,
%	align=center]
%\tikzstyle{obuf}=[
%	buf,
%	node distance=.7]
%\tikzstyle{env}=[
%	fill=black!20]
%\tikzstyle{simple}=[
%	rectangle,
%	draw=black,
%	minimum height=220,
%	text width=65,
%	align=center]
%\tikzstyle{mode text}=[midway,sloped,text width=200,align=center]
%\tikzstyle{moore}=[
%	rectangle,
%	draw=black,
%	minimum height= 30,
%	text width=80,
%	align=left]
%\tikzstyle{mealy}=[rectangle,
%	rounded corners=15,
%	draw=black,
%	text width=80,
%	align=left,
%	minimum height=40]
%\tikzstyle{ask}=[
%	diamond,
%	text width=50,
%	draw=black,
%	align=center]
%\tikzstyle{mux}=[
%	trapezium,
%	draw,
%	shape border rotate=270,
%	shape border uses incircle=true,
%	text width=20,]
%
%\newcommand{\epg}[3]{
%	Buffer {#1}\\
%	[44pt]EP{#2}\\
%	[44pt]{#3} Bytes}
%
%\newcommand{\ep}[3]{
%		Buffer {#1}\\
%		[8pt]EP{#2}\\
%		[8pt]{#3} Bytes}
%
%
%\begin{document}
%	\begin{tikzpicture}[scale=1]
%		\begin{scope}[transform shape,node distance=1,>=latex, circuit logic US]
%			\node[mux]		(clkMux)					{0\\1};
%			\node[]			(aux1)	[right=of clkMux]	{};
%			\node[]			(aux2)	[right=of aux1]		{};
%			\node[]			(aux3)	[right=of aux2]		{};
%			\node[not gate]	(n1)	[below=of aux3.north]		{};
%			\node[mux]		(outMux)[right=of aux3.south]		{0\\1};
%			\draw (clkMux.east) -- (outMux.north west) node [xshift=10] {5};
%%			\node[mealy]	(start)	[]	{Iniciar: Reset \\ \verb|main();|};
%%			\node[moore]	(init)	[below=of start]	{Inicia Variables de Estado}
%%				edge[<-,thick] (start);
%%			\node[moore]	(us1)	[below=of init]		{\verb|TD_Init();|}
%%				edge[<-,thick]	(init);
%%			\node[moore]	(EI)	[below=of us1]	{Habilita\\Interrupciones}
%%				edge[<-,thick](us1);
%%			\node[node distance=0.7]			(aux1)	[below=of EI] 	{};
%%			\draw[<-,thick](aux1.base) to (EI);
%%			\node[moore,node distance=.5]	(poll)	[below=of aux1]	{\verb|TD_Poll();|}
%%				edge[<-,thick](aux1.base);
%%			\node[ask]		(pr1)	[below=of poll]	{Paquete de Setup}
%%				edge[<-,thick](poll);
%%			\node[moore]	(setup)	[right=of pr1]	{\verb|SetupComand();|};
%%			\draw[->,thick] (setup) |- (aux1.base);
%%			\node[]			(aux2)	[below=of pr1]	{};
%%			\draw[->,thick]	(pr1) -- node[above,near start]{Si} (setup);
%%			\draw[thick]	(pr1) -- node[left,near start]{No}	(aux2.base);
%%			\node[node distance=2.5](aux3)	[left=of aux2] {};
%%			\draw[thick]	(aux2.base) -- (aux3.base);
%%			\draw[->,thick]	(aux3.base)	|-	(aux1.base);
%		\end{scope}
%	\end{tikzpicture}
%\end{document}