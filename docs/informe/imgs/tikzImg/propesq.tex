\documentclass{standalone}
%\usepackage[a3paper]{geometry}
\usepackage[spanish]{babel}
\usepackage{amsmath}
\usepackage{amssymb}
\usepackage{siunitx}
\usepackage{calc}

\usepackage{tikz,pgf}
\usetikzlibrary{calc,arrows,backgrounds,fit,positioning,petri,babel,shapes,circuits.logic.mux,circuits.logic.US}%shapes.gates.logic}

%\tikzstyle{style} = [definition]
\tikzstyle{interior}=[
	rectangle,
	rounded corners,
	draw=black,
	minimum size=40,
	fill=white]
\tikzstyle{core}=[
	interior]
\tikzstyle{perif}=[
	interior,
	minimum height=20]
\tikzstyle{exterior}=[
	rectangle,
	draw=black,
	minimum size=40]
\tikzstyle{contenedor}=[
	rectangle,
	draw=black]
\tikzstyle{buf}=[
	core,
	text width=55,
	align=center]
\tikzstyle{obuf}=[
	buf,
	node distance=.7]
\tikzstyle{env}=[
	fill=black!20]
\tikzstyle{simple}=[
	rectangle,
	draw=black,
	minimum height=220,
	text width=65,
	align=center]
\tikzstyle{mode text}=[midway,sloped,text width=200,align=center]
\tikzstyle{moore}=[
	rectangle,
	draw=black,
	minimum height= 30,
	text width=80,
	align=left]
\tikzstyle{mealy}=[rectangle,
	rounded corners=15,
	draw=black,
	text width=80,
	align=left,
	minimum height=40]
\tikzstyle{ask}=[
	diamond,
	text width=50,
	draw=black,
	align=center]
\tikzstyle{bloque}=[
	exterior,
	align=center,
	minimum size=60,
	text width=60]

\newcommand{\epg}[3]{
	Buffer {#1}\\
	[44pt]EP{#2}\\
	[44pt]{#3} Bytes}

\newcommand{\ep}[3]{
	Buffer {#1}\\
	[8pt]EP{#2}\\
	[8pt]{#3} Bytes}

\newcounter{wavecount}

\newcommand{\newwave}[1]{
	\path (0,\value{wavecount}) node[text width=70,anchor=east,align=right]{#1} node[coordinate](t_cur){};
	\draw (0,\value{wavecount}+.3) --++(.2,0);
	\draw (0,\value{wavecount}-.3) --++(.2,0);
	\path (t_cur) --++(.3,0)node[coordinate](t_cur){};
	\addtocounter{wavecount}{-1}}

\newcommand*{\bit}[2]{
	\draw (t_cur) -- ++(0.1,.6*#1-.3) -- ++(#2-.2,0) -- ++(+.1,.3-.6*#1)
	node[coordinate] (t_cur) {};}

\newcommand*{\bitvector}[2]{
	\draw[] (t_cur) -- ++( .1, .3) -- ++(#2-.2,0) -- ++(.1, -.3)
	-- ++(-.1,-.3) -- ++(.2-#2,0) -- cycle;
	\path (t_cur) -- node[align=center] {#1} ++(#2,0) node[coordinate] (t_cur) {};}

\begin{document}
	\begin{tikzpicture}[scale=1,>=latex,
	every label/.style={text width=70,align=flush center,transform shape}]
		\begin{scope}
			\begin{scope}[transform shape,node distance=.5]
				\node[core,minimum height=60]		(usb)	{USB};
				\node(text)	[above=of usb] {PC};
			\end{scope}
			\begin{scope}
				\node[bloque,fit= (usb)(text)]	(pc)	{};
				\node[bloque,fit=(pc),node distance=2, inner ysep=0] (text2) [right=of pc] {Interfaz de comunicacion};
				\node[node distance=2] (aux) [right=of usb] {};
			\end{scope}
		\end{scope}
		\begin{scope}[]
			\begin{scope}[transform shape,node distance=2]
%				\node[text width=60,align=center]	(text2)	[right=of usb]	{Interfaz de comunicación};
				\node (aux2) [right=of aux] {};
				\node[node distance=.5] (aux4) [below=of text2] {};
			\end{scope}
			\begin{scope}
%				\node[bloque,fit=(text2)(aux2)(aux4)] (bridge) {};
			\end{scope}
		\end{scope}
		\begin{scope}
			\begin{scope}[transform shape,node distance=.5]
				\node[core,node distance=2,minimum size=60]		(com)	[right=of aux2]{ICC};
				\node(text3)	[above=of com] {FPGA};
				\node[core,text width=50,align=center, minimum height=60](aux3) 	[right=of com]	{Sistema de adquisición};						
			\end{scope}
			\begin{scope}
				\node[bloque,fit= (com)(text3)(aux3)]	(fpga)	{};
			\end{scope}
		\end{scope}
	\end{tikzpicture}
\end{document}