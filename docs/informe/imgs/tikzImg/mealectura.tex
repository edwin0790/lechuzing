\documentclass[11pt]{standalone}
%CORREGIR VHDL ORIGINAL

\usepackage[spanish]{babel}
\usepackage{amsmath}
\usepackage{amssymb}
\usepackage{tikz}
\usepackage{siunitx}

\usetikzlibrary{arrows,backgrounds,decorations.pathreplacing,fit,positioning,petri,babel,shapes,circuits.logic.mux,circuits.logic.US,shapes.gates.logic,trees,patterns}
%\tikzstyle{style} = [definition]
\tikzstyle{interior}=[rectangle,rounded corners,draw=black,minimum size=3em, text width=20,fill=white]
\tikzstyle{exterior} = [rectangle,draw=black,minimum size=40]
\tikzstyle{mode text} = [midway,sloped,text width=200]
\tikzstyle{contenedor} = [rectangle,draw=black]
\tikzstyle{core}=[interior]
\tikzstyle{perif} = [interior,minimum height=20]
\tikzstyle{buf}=[core, text width=53, align=center]
\tikzstyle{obuf} = [buf, node distance=.7]
\tikzstyle{env} = [fill=black!20]
\tikzstyle{simple}=[rectangle, draw=black, minimum height=220, text width=65,align=center]
\tikzstyle{moore} = [rectangle,draw=black, minimum height= 30,text width=80,align=left]
\tikzstyle{mealy} = [rectangle,rounded corners=12, draw=black, text width = 80,align=left,minimum height=40]
\tikzstyle{ask} = [diamond,text width=50,draw=black,align=center,]
\tikzstyle{bloque}=[exterior,align=center,minimum size=60,text width=60]
\tikzstyle{hub}=[draw=black,anchor=west,circle]
\tikzstyle{dev}=[draw=black,anchor=west,rectangle,rounded corners,text width=60,align=center]
\tikzstyle{pid}=[draw=black,align=center,rectangle,rounded corners,minimum width=20,minimum height=120,pattern=north east lines,pattern color=black!35]
\tikzstyle{dir}=[draw,rectangle,rounded corners,minimum width=20,minimum height=120,align=center]
\tikzstyle{data}=[draw=black,align=center,rectangle,rounded corners,minimum width=120,minimum height=120,fill=black!05]
\tikzstyle{crc}=[draw=black,align=center,rectangle,rounded corners,minimum width=20,minimum height=120,pattern=dots,pattern color=black!25]
\tikzstyle{field}=[draw=black,rectangle,rounded corners, minimum height=30,align=center]
\tikzstyle{chart}=[circle,draw=black,text width=50,align=center]

\newcommand{\epg}[3]{
	Buffer {#1}\\
	[44pt]EP{#2}\\
	[44pt]{#3} Bytes}

\newcommand{\ep}[3]{
		Buffer {#1}\\
		[8pt]EP{#2}\\
		[8pt]{#3} Bytes}


\begin{document}
	\begin{tikzpicture}[scale=1]
		\begin{scope}[transform shape,node distance=1,>=latex,]
			\node[moore,text width=100] (inicio) [label=above right:inicio]{FIFOADR="ZZ"\\SLOE='1'\\SLRD='1'};
			\node[ask,text width=100] (vacio1) [below=of inicio]{FLAG Vacío};
			\draw[->] (inicio.south) -| (vacio1);
			\node[moore,text width=100] (lecdir) [below=of vacio1,label=above right:dirección] {FIFOADR=entrada\\SLOE='1'\\SLRD='1'};
			\draw[o->](vacio1.east) -- ($(vacio1.east)+(1,0)$);
			\draw[->] ($(vacio1.east)+(1,0)$) |- ($(inicio.north)+(0,1)$) -- (inicio.north);
			\draw[->] (vacio1.south) -| (lecdir);
			\node[moore,text width=100] (lecoe) [below=of lecdir,label=above right:habilitación] {FIFOADR=entrada\\SLOE='0'\\SLRD='1'};
			\draw[->](lecdir.south) -- ($(lecdir)!0.5!(lecoe)$);
			\draw[->]($(lecdir)!0.5!(lecoe)$) -- (lecoe);
			\node[moore,text width=100,node distance=1](lecrd)[below=of lecoe,label=above right:lectura]{FIFOADR=entrada\\SLOE='0'\\SLRD='0'};
			\draw[->](lecoe) -- (lecrd);
			\node[ask,text width=100] (vacio2)[below=of lecrd]{FLAG Vacío};
			\draw[->](lecrd) -- (vacio2);
			\draw[->](vacio2.west) -- ++(-1,0)  |- ($(lecoe.north)!0.5!(lecdir.south)$);
			\draw[o->](vacio2.east) -- ++(1,0) -- ($(vacio1.east)+(1,0)$);
		\end{scope}
	\end{tikzpicture}
\end{document}