\documentclass[12pt]{standalone}
%\usepackage[a3paper]{geometry}
\usepackage[spanish]{babel}
\usepackage[utf8]{inputenc}
\usepackage{amsmath}
\usepackage{amssymb}
\usepackage{siunitx}
\usepackage{calc}

\usepackage{tikz,pgf}
\usetikzlibrary{calc,arrows,backgrounds,fit,positioning,petri,babel,shapes,circuits.logic.mux,circuits.logic.US,trees}%shapes.gates.logic}

%\tikzstyle{style} = [definition]
\tikzstyle{interior}=[
	rectangle,
	rounded corners,
	draw=black,
	minimum size=40,
	fill=white]
\tikzstyle{core}=[
	interior]
\tikzstyle{perif}=[
	interior,
	minimum height=20]
\tikzstyle{exterior}=[
	rectangle,
	draw=black,
	minimum size=40]
\tikzstyle{contenedor}=[
	rectangle,
	draw=black]
\tikzstyle{buf}=[
	core,
	text width=55,
	align=center]
\tikzstyle{obuf}=[
	buf,
	node distance=.7]
\tikzstyle{env}=[
	fill=black!20]
\tikzstyle{simple}=[
	rectangle,
	draw=black,
	minimum height=220,
	text width=65,
	align=center]
\tikzstyle{mode text}=[midway,sloped,text width=200,align=center]
\tikzstyle{moore}=[
	rectangle,
	draw=black,
	minimum height= 30,
	text width=80,
	align=left]
\tikzstyle{mealy}=[rectangle,
	rounded corners=15,
	draw=black,
	text width=80,
	align=left,
	minimum height=40]
\tikzstyle{ask}=[
	diamond,
	text width=50,
	draw=black,
	align=center]
\tikzstyle{bloque}=[
	exterior,
	align=center,
	minimum size=60,
	text width=60]

\tikzstyle{hub}=[
	draw=black,
	anchor=north,
%	inner sep = 1,
%	growth parent anchor=bottom side,
	circle]

\tikzstyle{dev}=[
	draw=black,
	anchor=north,
	rectangle,
	rounded corners,
%	inner sep=6,
%	text width=50,
	align=center]

\newcommand{\epg}[3]{
	Buffer {#1}\\
	[44pt]EP{#2}\\
	[44pt]{#3} Bytes}

\newcommand{\ep}[3]{
	Buffer {#1}\\
	[8pt]EP{#2}\\
	[8pt]{#3} Bytes}

\newcounter{wavecount}

\newcommand{\newwave}[1]{
	\path (0,\value{wavecount}) node[text width=70,anchor=east,align=right]{#1} node[coordinate](t_cur){};
	\draw (0,\value{wavecount}+.3) --++(.2,0);
	\draw (0,\value{wavecount}-.3) --++(.2,0);
	\path (t_cur) --++(.3,0)node[coordinate](t_cur){};
	\addtocounter{wavecount}{-1}}

\newcommand*{\bit}[2]{
	\draw (t_cur) -- ++(0.1,.6*#1-.3) -- ++(#2-.2,0) -- ++(+.1,.3-.6*#1)
	node[coordinate] (t_cur) {};}

\newcommand*{\bitvector}[2]{
	\draw[] (t_cur) -- ++( .1, .3) -- ++(#2-.2,0) -- ++(.1, -.3)
	-- ++(-.1,-.3) -- ++(.2-#2,0) -- cycle;
	\path (t_cur) -- node[align=center] {#1} ++(#2,0) node[coordinate] (t_cur) {};}

\begin{document}
	\begin{tikzpicture}[scale=1,>=latex,level 1/.append style={level distance = 2ex},level 2/.append style={level distance = 40}]
		\begin{scope}
			\begin{scope}[transform shape,grow = down]
				\node[] (host) {\it HOST} [
				sibling distance=60,
%				growth parent anchor=south, 
				edge from parent fork down,
				]
					child{node[](l1r){Raíz}edge from parent[draw=none]
						child{node[](l2h1){Hub}
							child{node[](l3f1){Función}
								}
							child{node[](l3h1){Hub}
								child{node[](l4h1){Hub}
									child{node[](l5h1){Hub}
										child{node[](l6f1){Función}
											}
										child{node[](l6h1){Hub}
											child{node[](l7f1){Función}
												}
											}
										}
									child{node[](l5f1){Función}
										}
									}
								}
							child{node[] (l3f2) {Función}}
								}
						child{node[](l2f1){Función}}
						child{node[](l2h2){Hub}
							child{node[](l3h2){Hub}
								child{node[](l4f1){Función}
									}
								child{node[](l4f2){Función}
									}
								}
							}
						};
				\node[](l6)[left=of l6f1]{Grada 6};
				\node[](l7)at(l6|-l7f1){Grada 7};
				\node[](l5)at(l6|-l5h1){Grada 5};
				\node[](l4)at(l5|-l4h1){Grada 4};
				\node[](l3)at(l4|-l3f1){Grada 3};
				\node[](l2)at(l3|-l2h1){Grada 2};
				\node[](l1)at(l2|-l1r){Grada 1};
			\end{scope}
			\begin{scope}[dashed]
				\draw (l7) -| (l7f1.west);
				\draw (l6) -| (l6f1.west);
%					\draw(l6f1)--(l6h1);
				\draw (l5) -| (l5h1.west);
%					\draw(l5h1)--(l5h1);
				\draw (l4) -| (l4h1.west);
%					\draw(l4h1)--(l4f1);
%					\draw(l4f1)--(l4f2);
				\draw (l3) -| (l3f1.west);
%					\draw(l3f1)--(l3h1);
%					\draw(l3h1)--(l3f2);
%					\draw(l3f2)--(l3h2);
				\draw (l2) -| (l2h1.west);
%					\draw(l2h1)--(l2f1);
%					\draw(l2f1)--(l2h2);
				\draw (l1) -| (l1r.west);
			\end{scope}
		\end{scope}
	\end{tikzpicture}
\end{document}