%TODO Herramientas

	%TODO FPGA
	Una de las características notables de la industria de los semiconductores, es la velocidad en lo que a desarrollo se refiere. Como evidencia de esto, Gordon Moore postuló en el año 1965 lo que 
hoy en día se conoce como Ley de Moore: La complejidad para los costos de componente mínimo se ha incrementado, a groso modo, al doble por año. [...] No existe razon para pensar que no vaya a permanecer
casi constante en los próximos diez años.[https://ieeexplore.ieee.org/stamp/stamp.jsp?tp=&arnumber=4785860]. Si bien en el año 1975 Moore modificó su predicción de crecimiento de la induistria al doble
cada dos años en lugar al doble cada año, sigue siend un ritmo vertiginoso.\\
	Gracias a una sinérgia entre este crecimiento y los avances tecnológicos que él mismo genera, al punto de desconocer si una es causa de la otra o viceversa, no es extraño de pensar que la misma 
industria haya generado cada vez mejores herramientas, que posibilite avances mayores a un menor costo. En esta sinérgia, aparece una herramienta muy paricular, denominada FPGA, acrónimo de la voz 
inglesa \it{Field Programmable Gate Array}.\\
	Un FPGA es un circuito integrado programable diseñado para ser configurado por el usuario, de forma tal que permite, a través de elementos lógicos y rutas programables, diseñar, sintetizar e 
implementar circuitos digitales de muy alta velocidad y desempeño a un bajo. Actualmente, dependiendo el modelo y el fabricante, poseen diferentes bloques y configuraciones que facilitan al ingeniero la 
implementación y el diseño, talse como PLL's y DLL's, bloques de memoria, procesadores, módulos DSP, entradas y salidas con impedancias y estandares configurables, miles de flip-flops, tablas de 
busqueda y elementos de lógica discreta, entre otras.\\
	Esta gran versatilidad ha convertido en el FPGA una herramienta fundamental, por ejemplo, para el desarrollo de circuitos integrados a medida (ASIC's), y muy útil para el desarrollo rápido de 
sistemas específicos, por ejemplo el desarrollo de sensores y dispositivos digitales, muy utilizados en el ámbito científico.
	Para el presente trabajo, se escogió trabajar con una placa de desarrollo comercial MOJO-v3, desarrollada y comercializada por Embedded Micro (actualmente Alchitry). Esta placa fue desarollada
bajo la idea de placas de desarrollo de prototipado rápido económicas y modulares de fuente abierta, tanto de software como de hardware, concepto en el que tuvo mucho impacto la empresa Arduino, pero 
incorporando un FPGA como chip centra, en lugar de un microcontrolador. Esto posibilita tener desarrollos de sistemas digitales más veloces y que se adapten a las necesidades de uso mucho mejor a un 
bajo costo y con módulos que facilitan el diseño, de forma tal que este sea simple, veloz y versatil.\\
	La placa MOJO v3, que se observa en la Gifura \fig{la_mojo}%TODO metele figura de mojo
cuenta con un FPGA Xilinx Spartan-6 XC6SLX9, con un encapsulado de montaje superficial con 144 pines, de los cuales 84 son E/S digitale disponibles al usuario. Se disponen, 
además de 8 LED's de proposito general, un pulsador, 1 microcontrolador ATmega32U4, que permite una conexión USB de baja velocidad, configura el FPGA y realiza la lectura analógica de 8 entradas que se 
disponen para ese fin, una memoria flash que guarda de forma persistente los datos de configuración del FPGA[ref a MOJO]. Por su parte, el chip de Xilinx, se destacan a su vez mas de 9000 celdas 
lógicas, hasta 576 Kb de memoria RAM, 16 bloques DSP, 2 modulos de gestión de tiempo[ref a Spartan-6].\\
	
%TODO expĺain mojo v3, xilinx fpga, etc 

	%TODO Interface

	%TODO Compiler
