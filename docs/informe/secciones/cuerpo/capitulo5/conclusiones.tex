En el transcurso del trabajo reportado por este informe fue cumplido el objetivo general, el cual consistió en el desarrollo de un sistema de comunicación \acrshort{usb} 2.0 de alta velocidad destinado al intercambio de datos entre una \acrshort{pc} y un \acrshort{fpga}. El sistema desarrollado fue capaz de transmitir $1,08 \times 10^{12} \ ,bit$ sin errores ni interrupciones durante 24 horas seguidas.

Además de cumplimentar con el objetivo general perseguido por este trabajo, se logró entender conceptos fundamentales del funcionamiento del \acrshort{usb}, tal como el empaquetamiento de los datos y el tipo de transferencias que pueden realizarse a través de él. También se logró comprender cómo debe ser descripto un dispositivo \acrshort{usb} al ser desarrollado y cómo debe ser informado a la \acrshort{pc}. El sistema de comunicación implementado se compone de un software de computadora, una interfaz \acrshort{usb} y un \acrshort{fpga}.

Se utilizó el controlador FX2LP, comercializado por la empresa Cypress Semiconductor como interfaz \acrshort{usb} y a través de su estudio se pudo configurar dicho dispositivo, obteniendo un funcionamiento acorde a los requerimientos. El controlador FX2LP recibe transferencias en masa y transmite transferencias isócronas desde y hacia la \acrshort{pc} respectivamente. Con el \acrshort{fpga} se comunica por un bus de 16 bits a través de una memoria \acrshort{fifo} comandada por el \acrshort{fpga}.

Se utilizó un \acrshort{fpga} Spartan 6 comercializado por la empresa Xilinx para implementar, en lenguaje \acrshort{vhdl}, una Máquina de Estados Finitos capaz de enviar y recibir datos desde la interfaz \acrshort{usb}. Dicha máquina de estados es utilizada para comandar la memoria \acrshort{fifo} presente en el controlador FX2LP. Se destaca el bajo consumo de recursos programables de \acrshort{fpga} por parte del sistema desarrollado, dejando lugar a la implementación de aplicaciones que utilicen la comunicación desarrollada, por ejemplo el desarrollo de sensores para equipos de radiografías y neutrografías o también de detectores de partículas utilizando sensores de imagen \acrshort{cmos}. Además se elaboró un circuito impreso destinado a la conexión física entre el \acrshort{fpga} y la interfaz \acrshort{usb}.

Se desarrolló un software de computadoras que genera, envía y recibe datos hacia y desde el \acrshort{fpga}. Para la elaboración de este programa, se utilizó la biblioteca \verb|libusb-1.0|, que permite la comunicación de programas con dispositivos conectados a través del bus \acrshort{usb}. Esta es una biblioteca de código abierto y que puede ser ejecutado en cualquier sistema operativo.

Se implementó a su vez un sistema de pruebas compuesto de una memoria \acrshort{fifo} implementada en el \acrshort{fpga}, que recibe mensajes desde la interfaz \acrshort{usb} y los retransmite de vuelta. Este sistema permitió testear la funcionalidad y robustez de la comunicación desarrollada.
El sistema desarrollado fue probado y logró una conexión efectiva entre una \acrshort{pc} y un \acrshort{fpga}, logrando en la prueba un intercambio de mas de $1 \times 10^{12}$ bits sin pérdida de información. La tasa de transferencia de información útil lograda por la prueba de comunicación fue de \SI{9,12}{\mega\bit\per\second}, superior a la máxima tasa posible a través del \acrshort{spi} de la placa Mojo o del protocolo \acrshort{uart}.


%Si bien la tasa de bit hace suponer que la tasa de transferencia de \SI{12,4}{\mega\bit\per\second} puede sugerir que la comunicación está ocurriendo a una tasa de bit de velocidad completa, se debe tener en cuenta dos consideraciones.
%
%La primera de ellas tiene que ver con el hecho de que si el puerto USB estuviese transmitiendo a velocidad completa implicaría que el sistema desarrollado ocupa todo el ancho de banda disponible, lo cual no es correcto ya que en el mismo bus se encuentra incorporado el ratón y el teclado de la PC utilizada.
%
%La segunda consideración que debe tenerse en cuenta es que al tener configuraciones de emisión y transmisión diferente, la prueba realizada no sea suficiente para poder calcular la máxima tasa de bit que podría alcanzar el sistema utilizando al máximo posible el ancho de banda que podría brindar el host.
%
%Por tanto, se concluye que 