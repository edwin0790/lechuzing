A modo conceptual, la maquina de estados algorítmica (MAE) que se implementa en este trabajo es capaz de realizar dos tareas, bien definidas: leer y escribir datos desde y en la memoria FIFO. En la Figura \ref{fpga:mea:concepto} se muestra que cuando la memoria FIFO señala que no está vacía, se ejecuta la tarea de lectura de datos. Si la memoria FIFO se encuentra vacía y está activa la salida de datos, estos datos serán escritos en la memoria FIFO correspondiente.

Se puede notar que la implementación de estas tareas le otorga mayor prioridad a la operación de lectura que a la de escritura. Es decir, siempre que existan datos para leer en la memoria FIFO, serán leídos, aún cuando hayan datos para ser escritos.
Esta prioridad, viene dada en función de que se prevé que la comunicación sea utilizada por sensores que adquieren datos y los transmiten en forma inmediata a la PC y, a su vez, desde la PC se envía en forma ocasional datos que permitan configurar parámetros del sensor en particular. Por este motivo, se espera que los datos que provengan de la PC sean menos probables que los de los datos que está adquiriendo el sensor.

\begin{figure}[ht]
	\centering
	\begin{tikzpicture}[scale=.7]
		\begin{scope}[transform shape,node distance=3,>=latex,looseness=1.2]
			\node[chart](inicio){En espera};
			\node[chart](lectura)[below left=of inicio]{Leer datos};
			\node[chart](escritura)[below right=of inicio]{Escribir datos};
			\draw[->] (escritura.120) to [bend left](inicio.330);
			\draw[->] (lectura.60) to [bend right](inicio.210);
			\draw[->] (inicio.160) to [bend right] node [above,sloped] {FIFO Vacía=0} (lectura.110);
			\draw[->] (inicio.20) to [bend left] node [above,sloped,text width=80] {FIFO Vacía=1\\Salida datos=1} (escritura.70);
			\draw[->] (inicio.135) to [out=90,in=100,looseness=1.5] node [above,text width=80] {FIFO Vacía=1\\Salida datos=0} (inicio.45);
		\end{scope}
	\end{tikzpicture}
	\caption{Diagrama conceptual de la MEA}
	\label{fpga:mea:concepto}
\end{figure}

La Figura \ref{fpga:variables} muestra un esquema en donde se observan las variables que intervienen en la implementación del sistema. Como se observa en dicha figura, desde el extremo que se comunica con el controlador, las variables de entrada para la operación de lectura son el bus de datos {FD[15:0]\it} y el {\it FLAG Vacío}, conforme al diagrama temporal de la Figura \ref{fpga:lecfifo} En el caso de la operación de lectura, se utiliza además el {\it FLAG Vacío}. Las de salida son los puertos de dirección {\it FIFOADR[1:0]}, SLRD y SLOE para la operación de lectura y se agrega SLWR para la escritura.

\begin{figure}[ht]
	\centering
	\begin{tikzpicture}
		\begin{scope}
		\end{scope}
	\end{tikzpicture}
	\caption{esta figura}
	\label{fpga:variables}
\end{figure}

En la interfaz que se comunica con el interior del FPGA, existen las señales para enviar datos, los datos a enviar, los datos recibidos y una señal que indica que hay datos por recibir. 

Con base en los diagramas temporales y las variables anteriormente mencionadas, se plantea la maquina de estados que se observa en la Figura \ref{fpga:mea}.

\begin{figure}[ht]
	\centering
	\begin{tikzpicture}[ask/.style = {diamond,text width=70,draw=black,align=center,aspect=2},
	scale=.7]
		\begin{scope}[transform shape,node distance=1,>=latex,]
			\node[moore,text width=100] (inicio) [label=above right:inicio]{FIFOADR=$''$ZZ$''$\\SLOE=$'1'$\\SLRD=$'1'$\\SLWR=$'1'$};
			
			\node[ask] (vacio1) [below=of inicio]{FLAG Vacío};
			\draw[->] (inicio.south) -| (vacio1);
			%			\node[moore,text width=100] (lecdir) [below=of vacio1,label=above right:dirección] {FIFOADR=entrada\\SLOE=$'0'$\\SLRD=$'1'$\\SLWR=$'1'$};
			%			\draw[o->](vacio1.east) -- ($(vacio1.east)+(1,0)$);
			
			\node[moore,text width=100] (lecoe) [right=of vacio1,label=above right:dirección] {FIFOADR=entrada\\SLOE=$'0'$\\SLRD=$'1'$\\SLWR=$'1'$};
			\draw[->] (vacio1.east) -- ($(vacio1.east)!0.5!(lecoe.west)$);
			\draw[->]($(vacio1.east)!0.5!(lecoe.west)$) |- ($(lecoe.north)+(0,0.5)$) -- (lecoe.north);
			
			\node[moore,text width=100](lecrd)[below=of lecoe,label=above right:lectura]{FIFOADR=entrada\\SLOE=$'0'$\\SLRD=$'0'$\\SLWR=$'1'$};
			\draw[->](lecoe) -- (lecrd);
			
			\node[ask] (vacio2)[below=of lecrd]{FLAG Vacío};
			\draw[->](lecrd) -- (vacio2);
			
			\draw[o->](vacio2.west) -| ($(lecoe.west)!0.5!(vacio1.east)$);
			\draw[->](vacio2.east) -- ++(.8,0) |- ($(inicio.north)+(0,.6)$);
			\draw[->] ($(inicio.north)+(0,.6)$) -- (inicio.north);
			
			\node[ask](enviar1)[below=of vacio1]{Enviar datos};
			\draw[o->](vacio1.south) --(enviar1.north);
			
			
			\node[ask] (lleno1) [below=of enviar1]{FLAG Lleno};
			\draw[->](enviar1) -- (lleno1);
			
			\node[moore,text width=100](escdir)[below=of lleno1,label=above left:dirección]{FIFOADR=salida\\SLOE=$'1'$\\SLRD=$'1'$\\SLWR=$'1'$};
			\draw[->](lleno1) -- ($(lleno1.south)!0.5!(escdir.north)$);
			\draw[->]($(lleno1.south)!0.5!(escdir.north)$) -- (escdir);
			
			
			\node[ask](vacio3)[left=2 of vacio1]{FLAG Vacío};
			\draw[->](escdir.south)--($(escdir.south)+(0,-.5)$) -| ($(vacio3.east)+(.5,0)$) |- ($(vacio3.north)+(0,.5)$)-|(vacio3.north);
			
			\node[ask](enviar2)[below=of vacio3]{Enviar datos};
			\draw[o->](vacio3) -- (enviar2);
			
			\draw[o->] (enviar1.west) -- ($(enviar1.west)+(-.5,0)$);
			\draw[->] ($(enviar1.west)-(.5,0)$) -- ($(inicio.north -| enviar1.west)+(-.5,.6)$);
			\draw[->] ($(inicio.north -| enviar1.west)+(-.5,.6)$)--($(inicio.north)+(0,.6)$);
			\draw[->] ($(inicio.north)+(0,.6)$) -- (inicio.north);
			\draw[o->](lleno1.west) -| ($(enviar1.west)-(.5,0)$);
			\draw[->](vacio3.west) -- ($(vacio3.west)+(-.5,0)$);
			\draw[->]($(vacio3.west)+(-.5,0)$) |- ($(inicio.north -| enviar1.west)+(-.5,.6)$);
			
			\node[ask](lleno2)[below=of enviar2]{FLAG Lleno};
			\node[moore,text width=100](escwr)[below=of lleno2,label=above left:escribir]{FIFOADR=salida\\SLOE=$'1'$\\SLRD=$'1'$\\SLWR=$'0'$};
			\draw[->](lleno2) -- (escwr);
			\draw[o->](lleno2.west) -| ($(enviar2.west)+(-.5,0)$);
			\draw[->](enviar2)--(lleno2);
			\draw[o->](enviar2)--($(enviar2.west)+(-.5,0)$);
			\draw[->]($(enviar2.west)+(-.5,0)$) -- ($(vacio3.west)+(-.5,0)$);
			\draw[->](escwr) -- ($(escwr.south)+(0,-1)$) -| ($(escdir.east)+(.5,0)$) |- ($(lleno1.south)!0.5!(escdir.north)$);
		\end{scope}
	\end{tikzpicture}
	\caption{Maquina de estados que se implementa}
	\label{fpga:mea}
\end{figure}