\begin{figure}[ht]
	\centering
	\begin{tikzpicture}[scale=.7]
		\begin{scope}[transform shape,node distance=3,>=latex,looseness=1.2]
			\node[chart](inicio){En espera};
			\node[chart](lectura)[below left=of inicio]{Leer datos};
			\node[chart](escritura)[below right=of inicio]{Escribir datos};
			\draw[->] (escritura.120) to [bend left](inicio.330);
			\draw[->] (lectura.60) to [bend right](inicio.210);
			\draw[->] (inicio.160) to [bend right] node [above,sloped] {FIFO Vacía=0} (lectura.110);
			\draw[->] (inicio.20) to [bend left] node [above,sloped,text width=80] {FIFO Vacía=1\\Salida datos=1} (escritura.70);
			\draw[->] (inicio.135) to [out=90,in=100,looseness=1.5] node [above,text width=80] {FIFO Vacía=1\\Salida datos=0} (inicio.45);
		\end{scope}
	\end{tikzpicture}
	\caption{Diagrama conceptual de la MEA}
	\label{fpga:mea:concepto}
\end{figure}

A modo conceptual, la maquina de estados algorítmica (MAE) que se implementa en este trabajo es capaz de realizar dos tareas, bien definidas: leer y escribir datos desde y en la memoria FIFO. En la Figura \ref{fpga:mea:concepto} se muestra que cuando la memoria FIFO señala que no está vacía, se ejecuta la tarea de lectura de datos. Si la memoria FIFO se encuentra vacía y está activa la salida de datos, estos datos serán escritos en la memoria FIFO correspondiente.

Se puede notar que la implementación de estas tareas le otorga mayor prioridad a la operación de lectura que a la de escritura. Es decir, siempre que existan datos para leer en la memoria FIFO, serán leídos, aún cuando hayan datos para ser escritos.
Esta prioridad, viene dada en función de que se prevé que la comunicación sea utilizada por sensores que adquieren datos y los transmiten en forma inmediata a la PC y, a su vez, desde la PC se envía en forma ocasional datos que permitan configurar parámetros del sensor en particular. Por este motivo, se espera que los datos que provengan de la PC sean menos probables que los de los datos que está adquiriendo el sensor.

Se considera que la maquina de estados, además de poder comunicarse con la interfaz FX2LP, es necesario que los datos que se leen en él puedan estar disponibles para un sistema genérico, implementado en el mismo FPGA que posea la MEA. A su vez, debe poder los datos que deben ser escritos y señalar que los está emitiendo. Como señal de sincronismo, se utilizan las producidas para leer y escribir en las memorias, con el objetivo de optimizar el diseño al mínimo de recursos utilizados.

\begin{figure}[ht]
	\centering
	\begin{tikzpicture}[scale=.8]
		\begin{scope}[transform shape,node distance=4,>=latex,double distance=1.3]
			\node[simple](fx2lp){FX2LP FIFO};
			\node[simple](mea)[right=of fx2lp]{Maquina de Estados Algorítmica};
			
			\draw[double,<->] ([yshift=3.5*110/4]fx2lp.east)-- node [above]{FDATA[15:0]} ([yshift=3.5*110/4]mea.west);
			\draw[double,<->] ([yshift=2.5*110/4]fx2lp.east)-- node [above]{FADDR[1:0]} ([yshift=2.5*110/4]mea.west);
			\draw[->] ([yshift=1.5*110/4]fx2lp.east)--node[above]{FLAG\_Vacío} ([yshift=1.5*110/4]mea.west);
			\draw[->]([yshift=.5*110/4]fx2lp.east)--node[above]{FLAG\_Lleno}([yshift=.5*110/4]mea.west);
			\draw[<-]([yshift=-.5*110/4]fx2lp.east)--node[above]{SLOE}([yshift=-.5*110/4]mea.west);
			\draw[<-]([yshift=-1.5*110/4]fx2lp.east)--node[above]{SLRD}([yshift=-1.5*110/4]mea.west);
			\draw[<-]([yshift=-2.5*110/4]fx2lp.east)--node[above]{SLWR}([yshift=-2.5*110/4]mea.west);
			\draw[<-]([yshift=-3.5*110/4]fx2lp.east)--node[above]{PKTEND}([yshift=-3.5*110/4]mea.west);
			
			\node[simple,minimum size=80](clk)[right=of mea.north east,anchor=north west] {Fuente de reloj};
			\draw[<-]([yshift=-1*80/3]mea.north east)--node[above]{Reloj}([yshift=-1*80/3]clk.north west);
			\draw[<-]([yshift=-2*80/3]mea.north east)--node[above]{Reset}([yshift=-2*80/3]clk.north west);
			
			\node[simple,minimum height=130,minimum width=80](interno)[right=of mea.south east,anchor=south west]{Sistema\\Genérico};
			\draw[double,->]([yshift=5.5*130/6]mea.south east)--node[above]{Dato\_enviado[15:0]} ([yshift=5.5*130/6]interno.south west);
			\draw[double,<-]([yshift=4.5*130/6]mea.south east)--node[above]{Dato\_a\_enviar[15:0]}([yshift=4.5*130/6]interno.south west);
			\draw[<-]([yshift=3.5*130/6]mea.south east)--node[above]{Enviar\_datos}([yshift=3.5*130/6]interno.south west);
			\draw[->]([yshift=2.5*130/6]mea.south east)--node[above]{SLRD}([yshift=2.5*130/6]interno.south west);
			\draw[->]([yshift=1.5*130/6]mea.south east)--node[above]{SLWR}([yshift=1.5*130/6]interno.south west);
			\draw[<-]([yshift=.5*130/6]mea.south east)--node[above]{PKTEND}([yshift=.5*130/6]interno.south west);
			
		\end{scope}
	\begin{scope}[]
	\node[draw=black,rectangle,fit={(mea)(interno)},label=north:FPGA]{};
	\end{scope}
	\end{tikzpicture}
	\caption{Diagrama conceptual donde se observan las variables que intervienen en la MAE}
	\label{fpga:variables}
\end{figure}

Con todo esto, la maquina de estados deberá poseer las varibales que se observan en la Figura \ref{fpga:variables}. Como se muestra, desde el extremo que se comunica con el controlador, las variables de entrada para la operación de lectura son el bus de datos {FD[15:0]\it} y el {\it FLAG\_Vacío} (FLAGB), conforme al diagrama temporal de la Figura \ref{fpga:lecfifo} En el caso de la operación de escritura, se utiliza además el {\it FLAG\_Lleno} (FLAGA). Las de salida son los puertos de dirección {\it FIFOADR[1:0]}, SLRD y SLOE para la operación de lectura y se agrega SLWR para la escritura. Otra señal que se utiliza para la operación de escritura es PKTEND. Esta señal sirve para enviar paquetes más cortos que los esperados por la interfaz. El bus de datos FDATA[0:15] es bidireccional ya que se usa de entrada o salida en función de la operación que se efectúa.

Hacia la comunicación interna, se plantea una señal de envío de datos como puerto de entrada, que le permita al sistema indicar que está comunicando. Como señales de salida, se toman SLRD y SLWR como handshake, de forma tal que el sistema conozca cuando un dato es leido y/o escrito en la interfaz. Tambien se desdoblan los datos en dos buses diferentes, uno de entrada y otro de salida.

A su vez, se incorpora una señal de reloj, que estará encargada de temporizar la MEA y una señal de reset, que se encargara de iniciar todos las señales a un valor conocido de referencia, previo al comienzo del ciclo de la MEA.
Con base en los diagramas temporales y las variables anteriormente mencionadas, se diseña la maquina de estados que se observa en la Figura \ref{fpga:mea}. Se debe notar que {\it FLAG\_Vacío} y {\it FLAG\_Lleno} son activos en bajo, como asi también {\it SLOE, SLRD} y {\it SLWR}.

\begin{figure}[ht]
	\centering
	\begin{tikzpicture}[ask/.style = {diamond,text width=70,draw=black,align=center,aspect=2},
	scale=.8]
		\begin{scope}[transform shape,node distance=1,>=latex,]
			\node[moore,text width=100] (inicio) [label=above right:inicio]{FIFOADR=$''$ZZ$''$\\FDATA=$''$ZZ$''$\\SLOE=$'1'$\\SLRD=$'1'$\\SLWR=$'1'$};

			\node[ask] (vacio1) [below=of inicio]{FLAG\_Vacío};
			\draw[->] (inicio.south) -| (vacio1);
			%			\node[moore,text width=100] (lecdir) [below=of vacio1,label=above right:dirección] {FIFOADR=entrada\\SLOE=$'0'$\\SLRD=$'1'$\\SLWR=$'1'$};
			%			\draw[o->](vacio1.east) -- ($(vacio1.east)+(1,0)$);
			
			\node[moore,text width=100] (lecoe) [right=of vacio1,label=above right:dirección] {FIFOADR=$''1''$\\FDATA=D\_a\_enviar\\SLOE=$'0'$\\SLRD=$'1'$\\SLWR=$'1'$};
			\draw[->] (vacio1.east) -- ($(vacio1.east)!0.5!(lecoe.west)$);
			\draw[->]($(vacio1.east)!0.5!(lecoe.west)$) |- ($(lecoe.north)+(0,0.5)$) -- (lecoe.north);
			
			\node[moore,text width=100](lecrd)[below=of lecoe,label=above right:lectura]{FIFOADR=$''11''$\\FDATA=D\_a\_enviar\\SLOE=$'0'$\\SLRD=$'0'$\\SLWR=$'1'$};
			\draw[->](lecoe) -- (lecrd);
			
			\node[ask] (vacio2)[below=of lecrd]{FLAG\_Vacío};
			\draw[->](lecrd) -- (vacio2);
			
			\draw[o->](vacio2.west) -| ($(lecoe.west)!0.5!(vacio1.east)$);
			\draw[->](vacio2.east) -- ++(1.2,0) |- ($(inicio.north)+(0,.6)$);
			\draw[->] ($(inicio.north)+(0,.6)$) -- (inicio.north);
			
			\node[ask](enviar1)[below=of vacio1]{Enviar\_datos};
			\draw[o->](vacio1.south) --(enviar1.north);
			
			
			\node[ask] (lleno1) [below=of enviar1]{FLAG\_Lleno};
			\draw[->](enviar1) -- (lleno1);
			
			\node[moore,text width=100](escdir)[below=of lleno1,label=above left:dirección]{FIFOADR=$''00''$\\D\_recibido=FDATA\\SLOE=$'1'$\\SLRD=$'1'$\\SLWR=$'1'$};
			\draw[->](lleno1) -- ($(lleno1.south)!0.5!(escdir.north)$);
			\draw[->]($(lleno1.south)!0.5!(escdir.north)$) -- (escdir);
			
			
			\node[ask](vacio3)[left=2 of vacio1]{FLAG\_Vacío};
			\draw[->](escdir.south)--($(escdir.south)+(0,-.5)$) -| ($(vacio3.east)+(.5,0)$) |- ($(vacio3.north)+(0,.5)$)-|(vacio3.north);
			
			\node[ask](enviar2)[below=of vacio3]{Enviar\_datos};
			\draw[o->](vacio3) -- (enviar2);
			
			\draw[o->] (enviar1.west) -- ($(enviar1.west)+(-.5,0)$);
			\draw[->] ($(enviar1.west)-(.5,0)$) -- ($(inicio.north -| enviar1.west)+(-.5,.6)$);
			\draw[->] ($(inicio.north -| enviar1.west)+(-.5,.6)$)--($(inicio.north)+(0,.6)$);
			\draw[->] ($(inicio.north)+(0,.6)$) -- (inicio.north);
			\draw[o->](lleno1.west) -| ($(enviar1.west)-(.5,0)$);
			\draw[->](vacio3.west) -- ($(vacio3.west)+(-.9,0)$);
			\draw[->]($(vacio3.west)+(-.9,0)$) |- ($(inicio.north -| enviar1.west)+(-.5,.6)$);
			
			\node[ask](lleno2)[below=of enviar2]{FLAG\_Lleno};
			\node[moore,text width=100](escwr)[below=of lleno2,label=above left:escribir]{FIFOADR=$''00''$\\D\_Recibido=FDATA\\SLOE=$'1'$\\SLRD=$'1'$\\SLWR=$'0'$};
			\draw[->](lleno2) -- (escwr);
			\draw[o->](lleno2.west) -| ($(enviar2.west)+(-.9,0)$);
			\draw[->](enviar2)--(lleno2);
			\draw[o->](enviar2)--($(enviar2.west)+(-.9,0)$);
			\draw[->]($(enviar2.west)+(-.9,0)$) -- ($(vacio3.west)+(-.9,0)$);
			\draw[->](escwr) -- ($(escwr.south)+(0,-1)$) -| ($(escdir.east)+(.5,0)$) |- ($(lleno1.south)!0.5!(escdir.north)$);
		\end{scope}
	\end{tikzpicture}
	\caption{Maquina de estados que se implementa}
	\label{fpga:mea}
\end{figure}