\begin{figure}[ht]
			\centering
	\begin{tikzpicture}[scale=1*\textwidth/\paperwidth]
	\begin{scope}[transform shape,node distance=4,>=latex]
	\node[simple]	(fifo)		[]	 			{FIFO's Esclavas};
	\node[simple]	(master)	[right=of fifo]	{Maestro Externo};
	\draw[<->,thick]	([yshift=5*110/6]fifo.east) --node [above]{IFCLK} ([yshift=5*110/6]master.west);
	\draw[<->,thick]	([yshift=4*110/6]fifo.east) --node [above]{FD[15:0]} ([yshift=4*110/6]master.west);
	\draw[<-,thick]	([yshift=3*110/6]fifo.east) --node [above]{FIFOADR[1:0]} ([yshift=3*110/6]master.west);
	\draw[->,thick]	([yshift=2*110/6]fifo.east) --node [above]{FLAGA} ([yshift=2*110/6]master.west);
	\draw[->,thick]	([yshift=1*110/6]fifo.east) --node [above]{FLAGB} ([yshift=1*110/6]master.west);
	\draw[->,thick]	([yshift=0*110/6]fifo.east) --node [above]{FLAGC} ([yshift=0*110/6]master.west);
	\draw[->,thick]	([yshift=-1*110/6]fifo.east) --node [above]{FLAGD} ([yshift=-1*110/6]master.west);
	\draw[<-,thick]	([yshift=-2*110/6]fifo.east) --node [above]{SLOE} ([yshift=-2*110/6]master.west);
	\draw[<-,thick]	([yshift=-3*110/6]fifo.east) --node [above]{SLWR} ([yshift=-3*110/6]master.west);
	\draw[<-,thick]	([yshift=-4*110/6]fifo.east) --node [above]{SLRD} ([yshift=-4*110/6]master.west);
	\draw[<-,thick]	([yshift=-5*110/6]fifo.east) --node [above]{PKTEND} ([yshift=-5*110/6]master.west);
	\end{scope}				
	\end{tikzpicture}
	\caption{Puertos de interfaz entre las FIFO's y un maestro externo}
	\label{fpga:interfaz}
\end{figure}

La Figura \ref{fpga:interfaz} muestra los puertos a través de los cuales se conectan el controlador FX2LP con el FPGA Spartan-IV. La función de cada uno de estos puertos está detallada en la Sección \ref{cy:fifo} del presente informe. A continuación se explica el protocolo que debe seguir el sistema para poder leer y escribir datos. Este protocolo permite comprender de una forma más acabada las señales de control que debe proveer la MEA.

Las señales FIFOADR[1:0] se utilizan para seleccionar la memoria FIFO sobre la que se escriben o leen los datos. Estas memorias, que poseen dirección hexadecimal 02, 04, 06 y 08 para el controlador 8051 incorporado en el integrado FX2LP, tienen dirección binaria $''00'', ''01'', ''10''$ y $''11''$ en los puertos FIFOADR[1:0], respectivamente, como se muestra en la Tabla \ref{tab:fifoadr}. Se destaca que $'0'$ y $'1'$ en cada puerto FIFOADR equivale a niveles de tensión bajo y alto, respectivamente.

\begin{table}[ht]
	\centering
	\begin{tabular}{cc}
		\hline
		FIFOADR[1:0] & FIFO\\
		\hline
		00 & 0x02\\
		01 & 0x04\\
		10 & 0x06\\
		11 & 0x08\\
		\hline
	\end{tabular}
	\caption{Direcciones de selección de memoria activa}
	\label{tab:fifoadr}
\end{table}

Las señales que señalan la ocurrencia de algún evento particular con las memorias, como ser el alcance de la capacidad mínima o la inexistencia de datos, son programables. Es decir, al momento de realizar la configuración del controlador FX2LP, el desarrollador puede seleccionar que señales estarán presentes en los puertos FLAGA, FLAGB, FLAGC y FLAGD.

La configuración que se implementa en este trabajo, tal como se menciona en el Capítulo \ref{cap:cy}, dispone a la memoria FIFO 02 como puerto de entrada USB (es decir, salida desde el FPGA) y al puerto 08 como salida USB (o sea, entrada para el FPGA). A su vez, el puerto FLAGA señala que la memoria FIFO 02 está llena y el FLAGB indica que la memoria FIFO 08 está vacía.

El lector puede notar que no se detalla en ninguno de los diagramas temporales presentes en este informe la señal del puerto IFCLK. Esto se debe a que por errores de diseño del alumno, no es posible implementar un funcionamiento sincrónico, quedando sin uso la señal del puerto señalado.

\subsection{Lectura de datos desde la memoria FIFO}

	\begin{figure}[ht]
		\centering
		\begin{tikzpicture}[scale=1.4\textwidth/\paperwidth]
			\begin{scope}[transform shape,node distance=1,text width=60]
				\setcounter{wavecount}{0}
				\newwave{FIFOADR[1]}
					\bit{1}{7}
				\newwave{FIFOADR[0]}
					\bit{1}{7}
				\newwave{FLAG Vac\'io}
					\bit{1}{5}
					\bit{0}{2}
				\newwave{SLOE}
					\bit{1}{1}
					\bit{0}{5}
					\bit{1}{1}
				\newwave{SLRD}
					\bit{1}{2}
					\bit{0}{1}
					\bit{1}{1}
					\bit{0}{1}
					\bit{1}{2}
				\newwave{FD[15:0]}
					\bitvector{Z}{1}
					\bitvector{N-1}{2}
					\bitvector{N}{2}
					\bitvector{X}{1}
					\bitvector{Z}{1}
			\end{scope}
			\begin{scope}[on background layer]
				\foreach \x in {1,2,...,7}{
					\draw[dashed,black!20] (\x.3,0) -- (\x.3,\value{wavecount}+1);}
			\end{scope}
		\end{tikzpicture}
		\caption{Diagrama temporal de la lectura de datos desde la memoria FIFO por un FPGA}
		\label{fpga:lecfifo}
	\end{figure}

	Para efectuar una operación de lectura en régimen asíncrono, la que se muestra en la Figura \ref{fpga:lecfifo}, en primer lugar, el FPGA debe colocar en los puertos FIFOADR la dirección de la memoria sobre la que desea efectuar esta operación. Luego, debe ser activada la señal SLOE, la cual coloca en los puertos FD[15:0] los datos almacenados en la memoria FIFO activa por FIFOADR. El dato disponible en la salida de la memoria FIFO siempre será el más antiguo. En el cambio de asertiva a negada de la señal SLRD , la memoria FIFO aumenta un contador que selecciona la dirección del próximo dato y lo coloca en el puerto FD[15:0].
	
	Una vez que todos los datos fueron leídos, la memoria FIFO activa la señal FLAG Vacío. Mientras SLOE no es acertivo, el puerto FD[15:0] permanece en estado de alta impedancia. En la Figura \ref{fpga:lecfifo} se puede observar también que tanto las señales FLAG, SLOE y SLRD son asertivas en $'0'$.
	
\subsection{Escritura de datos en la memoria FIFO}

%TODO el diagrama temporal
	\begin{figure}[ht]
		\centering
		\begin{tikzpicture}[scale=1.4\textwidth/\paperwidth]
			\begin{scope}[transform shape,node distance=1,text width=60]
				\setcounter{wavecount}{0}
				\newwave{FIFOADR[1]}
					\bit{0}{9}
				\newwave{FIFOADR[0]}
					\bit{0}{9}
				\newwave{FLAG Lleno}
					\bit{1}{7}
					\bit{0}{2}
				\newwave{SLWR}
					\bit{1}{3}
					\bit{0}{1}
					\bit{1}{1}
					\bit{0}{1}
					\bit{1}{1}
					\bit{0}{1}
					\bit{1}{1}
				\newwave{FD[15:0]}
					\bitvector{N-2}{3}
					\bitvector{N-1}{2}
					\bitvector{N}{2}
					\graybitvector{No leído}{2}
			\end{scope}
			\begin{scope}[on background layer]
				\foreach \x in {1,2,...,9}{
					\draw[dashed,black!20] (\x.3,0) -- (\x.3,\value{wavecount}+1);}
			\end{scope}
		\end{tikzpicture}
		\caption{Diagrama temporal de la escritura de datos en la memoria FIFO desde un FPGA}
		\label{fpga:escfifo}
	\end{figure}

	Las señales que intervienen en el proceso de escritra de datos en la memoria FIFO, se encuentran detalladas en el diagrama temporal de la Figura \ref{fpga:escfifo}. Para escribir datos en una memoria FIFO, el FPGA debe seleccionar la memoria a través de FIFOADR en primer lugar. Luego, se coloca en el bus de datos, donde se encuentran conectados los puertos FD[15:0], el dato a escribir. Se debe tener en cuenta que SLOE debe estar no asertivo, de modo tal que el bus FD se encuentre en modo de alta impedancia y no interfiera con la escritura.
	
	Una vez colocado el dato en el bus, se debe activar la señal SLWR. En el flanco asertivo de SLWR, el controlador incrementa el contador que indica la dirección de memoria en donde será almacenado el dato siguiente y deja guardado el dato que leyó en los puertos del bus FD. Como se observa en el diagrama de la Figura \ref{fpga:escfifo}, SLWR es activo en bajo.