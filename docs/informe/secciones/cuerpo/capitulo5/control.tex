En el Capitulo \ref{cap:cy} del presente informe se describen los puertos de la interfaz entre un maestro esterno y las memorias FIFO del controlador de Cypress.\\

Al abordar, ahora, el control del mismo, es necesario desarrollar con mayor detalle cada una de estas señales. Así, se podrá comprender, con mayor detalle, las señales que debe enviar y recibir una MEA que controle esta interfaz.\\

Si bien el modo de funcionamiento de la interfaz puede ser síncorno o asíncorno, se detalla sólo este último, es decir, el implementado, ya que debido a errores de diseño de quien escribe, no es posible implementar, hasta el momento de la escritura, un funcionamiento sincrónico del sistema.\\

Por lo anterior, no se detalla en ningún momento la señal del puerto IFCLK.\\

\subsection{Lectura de datos desde la interfaz}

	\begin{figure}
		\centering
		\begin{tikzpicture}[scale=1.4\textwidth/\paperwidth]
			\begin{scope}[transform shape,node distance=1,text width=60]
				\setcounter{wavecount}{0}
				\newwave{FIFOADR[0]}
					\bit{1}{7}
				\newwave{FIFOADR[1]}
					\bit{1}{7}
				\newwave{FLAG Vac\'io}
					\bit{1}{5}
					\bit{0}{2}
				\newwave{SLOE}
					\bit{1}{1}
					\bit{0}{5}
					\bit{1}{1}
				\newwave{SLRD}
					\bit{1}{2}
					\bit{0}{1}
					\bit{1}{1}
					\bit{0}{1}
					\bit{1}{2}
				\newwave{FD[0:15]}
					\bitvector{Z}{1}
					\bitvector{1}{2}
					\bitvector{0}{2}
					\bitvector{X}{1}
					\bitvector{Z}{1}
			\end{scope}
			\begin{scope}[on background layer]
				\foreach \x in {1,2,...,7}{
					\draw[dashed,black!20] (\x.3,0) -- (\x.3,\value{wavecount}+1);}
			\end{scope}
		\end{tikzpicture}
		\caption{este texto}
		\label{if:label}
	\end{figure}
