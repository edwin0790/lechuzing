El desarrollo expuesto en el presente trabajo puede ser de gran utilidad en el desarrollo de aplicaciones científicas. Se espera que el mismo sea utilizado en la lectura de imágenes adquiridas a través de sensores \acrshort{cmos} para aplicaciones tales como neutrografía y detección de radiación, entre otros.

Se plantea para trabajos futuros la posibilidad de implementar una nueva prueba de funcionamiento a través del envío ininterrumpido de tramas conocidas de datos, generados en forma local desde el \acrshort{fpga}, lo que permitirá conocer en mayor detalle la tasa máxima de bit posible con la configuración desarrollada en este trabajo.

Otra mejora a implementar en el sistema desarrollado consta de la realización sincrónica de la comunicación entre el \acrshort{fpga} y la Interfaz \acrshort{usb}. Para ello, es necesario reconfigurar el controlador FX2LP, indicando que el funcionamiento de la memoria \acrshort{fifo} será de modo sincrónico. A su vez se debe rediseñar la placa de interconexión de forma tal que tanto el controlador FX2LP como el \acrshort{fpga} puedan compartir el reloj. Se espera que esta mejora brinde mayor velocidad al sistema, como así también se releve al \acrshort{fpga} de utilizar un \acrshort{pll} para brindar la señal de reloj necesaria, liberando recursos para su utilización en otro tipo de desarrollos implementados dentro de este.