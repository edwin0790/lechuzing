%TODO Programación y configuración de la interfaz Cypress 
\section{Programación y Configuyración de la Interface}

	%TODO Elección de la configuración de la interfaz Cypress para cumplir con los requerimientos del sistema
	El objetivo que persigue el diseño de sensores es transformar alguna variable física en una señal diferente que luego pueda ser medida y registrada.Esto dá lugar a lo que se conoce
como dato. Los datos son, entonces, el valor de una medida que representa la variable física que se desea conocer.\\
	Sin embargo, los datos aislados por sí mismos dicen muy poco con respecto a la variable medida. Para que un dato sea realmente útil, al mismo se le debe aplicar una serie de metodos
y procedimientos que otorguen información.\\
	La información es el resultado de la comparación de un conjunto de datos entre sí, de manera que permita encontrar relaciones entre ellos. Con la información, un repector aumenta el
conocimiento que posee en relación al sistema medido.\\
	En otras palabras para adquirir conocimiento, es necesario obtener información; para adquirir información es necesario contar con un conjunto de datos que permitan ser procesados;
a su vez, los datos son obtenidos por sensores.\\
	Una vez adquirido un dato, el mismo debe ser almacenado en forma segura, de forma tal que, una vez obtenida una cantidad suficiente que permita extraer información, pueda ser procesada.
	El proceso de los datos puede swer realizado in situ o a posteriori.
	Cada uno de ellos presenta ventajas e inconvenientes
	entre las ventajas de el procesamiento de datos in situ, es que la información se obtiene de una forma mucho más veloz. La desventaja es que el sistema desarrollado 
Este proceso puede ser in situ, es decir, en el mismo
momento en que los datos son adquiridos, o bien, puede hacerse a posteriori, ya sea de forma automática o manual. 
	%TODO Descripción del Framework Cypress
	%TODO Adaptación del framework para su utilización en sistemas de código abierto
	%TODO Configuración del microprocesador 8051

