\chapter{Introducción}
  \section{Motivación}

%SOBRE_USB
En el mes de Noviembre de 1994, un comité conformado por seis de los más grandes
fabricantes de computadoras emitieron los primeros lineamientos de un protocolo
de comunicación de arquitectura abierta que permitiese la Comunicación entre
dispositivos de diferentes vendedores.\\

Sobre finales del mes de Abril del 2000, Compaq, Hewlett-Packard, Intel, Lucent,
Microsoft, NEC y Philips lanzaron las especificaciones del {\it Universal Serial
Bus 2.0}, ampliamente conocido como USB 2.0 con los objetivos específicos de ser
una forma de comunicar teléfonos y computadoras personales, fácil de usar, de
puerto expansible y que tenga la velocidad suficiente para operar los
dispositivos disponibles hasta ese año y que bastó durante mucho tiempo
después.\\%REFERENCIA {ref usb2.0espec}\\

El peso de mercado que tuvieron estas compañías, e incluso algunas aún tienen,
hizo que esta norma sea ampliamente aceptada y con el paso de los años se
transformó en una forma de comunicación estándar de cualquier producto
electrónico de consumo masivo que requiriese interacción con una PC.\\


%SOBRE_LA_PC
La evolución de la compútadora personal, así como el desarrollo de lenguajes de
programación más simples, desde el punto de vista del desarrollador; software y
harware cada vez más potente, la han posicionado como la herramienta fundamental
en cualquier área. Hoy poy hoy, es posible encontrar sistemas tan diversos y
específicos como uno desee, desde programas que realizan operaciones de cálculo
muy básicas hasta sistemas operativos completos dedicados al control de redes de
sensores y datos sumamente complejos.\\

En este sentido, las posibilidades son casi infinitas, y se encuentran en el
mercado miles de herramientas de diseño personalizado en cualquier área, ya sea
gráfica, audiovisual, mecánica, inmobiliaria, médica, de infraestructura y un
sinnúmero de etcéteras.\\


De esto no han quedado exenta la industria electrónica y actualmente es posible
obtener programas que permiten calcular componentes, dibujar circuitos impresos,
compilar códigos para memorias, controladores y/o procesadores e incluso
permiten diseñar circuitos integrados. También es facible hallar software que
realiza adquisición de señales, gráfica de variables en tiempo real, o no;
simulación de sistemas, entre una gama muy amplia que hacen de la informática en
general y de la computadora en particular, un equipo indispensable para
cualquier desarrollo.\\

%SOBRE_FPGA
A su vez, el avance tecnológico que ha logrado la industria del semiconductor
sobre el desarrollo de circuitos integrados cada vez más potentes, rápidos y
eficientes ha logrado que lleguen a manos de cualquier persona, con ánimos de
trabajar, circuitos integrados programables de bajo costo que permiten avanzar
sobre el desarrollo de sistemas a la medida de cualquier cliente que esté
disponible. En este sentido, es posible adquirir micro-controladores
(\SI{}{\micro C}), dispositivos lógicos programables (PLD's), arreglos de
compuertas lógicas (GAL's), arreglos de compuertas programables
en campo (FPGA's), entre otros dispositivos, a bajo costo y que posibilita el
desarrollo sistemas de baja o alta complejidad a precios competitivos y tiempos
de entrega acordes.\\

Uno de los más interesantes y desarrollados de estos sistemas programables
disponibles en el mercado es, sin dudas, el FPGA, dado que brinda la
flexibilidad necesaria al desarrollador de diseñar, implementar, probar y
depurar sistema digitales de alta velocidad y/o elevada complejidad a medida.
Sin embargo, este tipo de chips son relativamente caros, si se los compara
con cualquier otro tipo de circuito programable, y se torna de suma importancia
aprovechar al máximo su capacidad.\\

Debido a la versatilidad del FPGA y su creciente adopción es de suma utilidad
realizar una comunicación fluida del sistema en desarrollo con un PC a través de
un puerto USB. El presente informe da cuenta de lo realizado por el autor de
este trabajo por lograr esta tarea.\\

  \section{Objetivos}
    \subsection{Objetivo General}

    El objetivo del presente trabajo es realizar una correcta comunicación de
    datos entre un sistema basado en FPGA y una computadora personal a través de
    un puerto USB.

    \subsection{Objetivos Particulares}

    Para la concreción del Objetivo General descripto anteriormente, se procedió
    a lograr los siguientes objetivos específicos.

      \begin{itemize}
        \item Adaptación del framework provisto por Cypress Semiconductors para
        istribuciones de sistemas operativos Linux basados en Debian.
        \item Configuración del kit de desarrollo CY3684 de Cypress.
        \item Configuración de un FPGA de la familia Spartan de Xilinx.
        \item Interconexion entre las distintas placas de desarrollo y la PC.
        \item Validación de la conexión
      \end{itemize}

  \section{Estructura del Informe}
  %ESTRUCTURA
