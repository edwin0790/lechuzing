Durante el transcurso del presente trabajo se logro establecer una comunicación entre una PC y un FPGA a través de una comunciación USB 2.0 de alta velocidad. Sin embargo, el rendimiento esperado para las pruebas resultó demasiado bajo.

Se plantea para un trabajo futuro la posibilidad de implementar una nueva prueba de funcionamiento a través del envío ininterrumpido de tramas conocidas de datos, generados en forma local desde el FPGA. Esto permitirá conocer en mayor detalle la tasa máxima de bit lograda con la configuración desarrollada en este trabajo.

Otra mejora a implementar en el sistema desarrollado consta de la realización sincrónica de la comunicación entre el FPGA y la Interfaz USB. Para ello, es necesario reconfigurar el controlador FX2LP, indicando que el funcionamiento de la memoria FIFO será de modo sincrónico. A su vez se debe rediseñar la placa de interconexión de forma tal que tanto el controlador FX2LP como el FPGA puedan compartir el reloj. Se espera que esta mejora brinde mayor velocidad al sistema, como así también se releva al FPGA de utilizar un PLL para brindar la señal de reloj necesaria, liberando recursos para su utilización en otro tipo de desarrollos implementados dentro de este.