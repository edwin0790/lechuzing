%El sistema desarrollado pretende ser una implementación que transmita datos desde una PC hacia un FPGA y viceversa. Una vez configurada la interfaz y diseñada y sintetizada la maquina de estados, se posee un sistema capaz de establecer la comunicación.
%
%Sin embargo, aún resta verificar y validar que dicha comunicación es funcional, es decir, que efectivamente puede transmitir datos de un punto a otro y cumple con las especificaciones de velocidad.
%
%Durante el proceso, se encontraron varios problemas relativos tanto al desarrollo de la configuración de la interfaz como al de la maqui

Configurada la interfaz, desarrollada la máquina de estado y conectados los circuitos integrados a través del circuito impreso, resta probar el sistema, transfiriendo datos desde la PC hacia el FPGA y viceversa.

Para lograr esto, se probó en primer lugar la conexión USB desde la interfaz hacia la PC y luego se incorporó el FPGA. Este capítulo tratará cada una de estas etapas de desarrollo. Es decir, en primer lugar se hablará sobre las pruebas y depuración necesaria de la configuración de la interfaz. Seguidamente, se abordarán los test bench realizados a la MEF y los módulos utilizados para verificar su funcionamiento.

Luego, se procede a explicar las pruebas realizadas al sistema desde la PC en un software específico, escrito en C.

Finalmente, se detallarán los resultados y conclusiones obtenidas durante el desarrollo del trabajo que se informa en este documento.