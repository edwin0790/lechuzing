%INTENTO 1
%El sistema desarrollado pretende ser una implementación que transmita datos desde una PC hacia un FPGA y viceversa. Una vez configurada la interfaz y diseñada y sintetizada la maquina de estados, se posee un sistema capaz de establecer la comunicación.
%
%Sin embargo, aún resta verificar y validar que dicha comunicación es funcional, es decir, que efectivamente puede transmitir datos de un punto a otro y cumple con las especificaciones de velocidad.
%
%Durante el proceso, se encontraron varios problemas relativos tanto al desarrollo de la configuración de la interfaz como al de la maqui

%INTENTO 2
%Configurada la interfaz, desarrollada la máquina de estado y conectados los circuitos integrados a través del circuito impreso, resta probar el sistema, transfiriendo datos desde la PC hacia el FPGA y viceversa.
%
%Para lograr esto,fue neceasario montar el sistema emulando las condiciones esperadas de uso. Es decir, se procedió a conectar las placas de desarrollo entre sí y la interfaz con la PC. También se incorporó un sistema genérico dentro del FPGA que almacene los datos ingresados a través de la PC y que los escriba, luego, en la interfaz para su posterior recepción por el Host. Para ello, se implemento un sistema tipo eco a través de una memoria FIFO sintetizada en el FPGA.
%
%Además, se realizó un código en C para enviar y recibir datos en forma automática. De esta manera se pudo comprobar el desempeño del sistema desarrollado.
%
%A lo largo de este capítulo se desarrolla la implementación de la memoria FIFO en VHDL, se detalla el código desarrollado para las pruebas y, finalmente, se expondrán los resultados y conclusiones obtenidas durante el desarrollo del trabajo que se informa en este documento.


% Otra vez INTENTO 3

%Hasta el momento, se explicaron aspectos varios de la norma USB, se seleccionaron las herramientas a utilizar para cumplir con los objetivos del trabajo, se configuraron las distintas partes que componen el sistema y se realizó un circuito que conecte a cada una de ellas.

%Con lo anterior, podría esperarse que el sistema desarrollado sea funcional. Sin embargo, esto no puede aseverarse hasta probar que la comunicación entre una PC y el FPGA es efectiva. Por lo tanto es necesario realizar una verificación del sistema.

%Para ello, es necesario implementar el sistema como se espera que funcione. Es decir, se debe implementar un sistema en FPGA que pueda enviar y recibir datos desde una PC. A continuación, se desarrollará la implementación sistema de tipo eco, que recibe datos desde una PC y los transmite, luego, en el sentido inverso.

%Con este propósito se explicarán los sistemas implementados dentro del FPGA, como así también el programa de computadora realizado con el fin de probar esto y se expondrán los resultados y las conclusiones del trabajo realizado.

% INTENTO 4

El sistema de comunicación desarrollado fue sometido a una prueba de desempeño a fin de verificar que sea capaz de enviar y recibir datos en forma efectiva y que, a su vez, cumpla con el objetivo de establecer un enlace capaz de enviar datos a una tasa de \SI{480}{\mega\bit\per\second}. La prueba realizada consistió en el envío de un conjunto de datos desde una \acrshort{pc}, que fue almacenado en el \acrshort{fpga} para luego ser retransmitidos hacia la \acrshort{pc} y fue repetida en forma automática durante 24 horas. Para efectuar dicha prueba, se elaboró un sistema en \acrshort{fpga}, dotado con la capacidad de memoria necesaria para recibir los datos que luego enviará la \acrshort{mef}, presentada en el Capítulo~\ref{cap:fpga}, que provee lo necesario para la comunicación con el controlador FX2LP y una señal de reloj que permite sincronizar la \acrshort{mef} y la memoria implementada.

Además, se realizó un programa de computadora, escrito en lenguaje C++, que permite enviar una trama de datos aleatorios hacia el dispositivo desarrollado en el presente trabajo, a través del protocolo \acrshort{usb}. A lo largo de este Capítulo, se describirán los componentes utilizados para realizar las pruebas de desempeño del sistema de comunicación desarrollado y se expondrán los resultados obtenidos.