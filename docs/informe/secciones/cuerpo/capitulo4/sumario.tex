A lo largo del presente capítulo se presentaron las pruebas de desempeño del sistema realizado y los resultados de estas. Para la realización de las pruebas, los sistemas configurados en los capítulos precedentes fueron conectados. A su vez, se expuso la elaboración de un sistema tipo Eco en el FPGA con cada uno de los módulos que lo componen. Debido a la incorporación de estos componentes, se realizó una nueva verificación funcional de la síntesis del circuito con resultados positivos.

Luego se detalló el desarrollo de un software para PC, utilizando la biblioteca \verb|libusb-1.0|, encargado de generar paquetes de datos, transmitirlos y comparar los resultados a fin de detectar errores. La prueba de desempeño fue repetida en forma ininterrumpida durante más de 24 horas logrando transmitir $1,08\times10^{12}\,bit$ sin errores con el enlace USB funcionando a alta velocidad. 