El protocolo USB es un sistema de comunicación diseñado durante los años 90 por los seis fabricantes de la industria de las computadoras, Compaq, Intel, Microsoft, Hewlett-Packard, Lucent, NEC y Philips, con la idea de proveer a su industria de un sistema que permita la conexión entre las PC's y los periféricos con un formato estandart, de forma tal que permita la compatibilidad entre los distintos fabricantes.\\

Hasta ese momento, el gran ecosistema de periféricos, sumado a los nuevos avances y desarrollos, hacia muy compleja la conectividad de todos ellos. Cada uno de los fabricantes desarrollaba componmentes con fichas, niveles de tensión, velocidades, drivers y un sinnumero de etc diferentes, lo cuál dificultaba al usuario estar al día y poder utilizar cada componente que compraba. Lo más probable era encontrarque cada vez que uno comparaba una PC, debía cambiar el teclado, el mouse o algún periferico específico. Esto también complicaba a las mismas empresas productoras, por que la introducción de un nuevo sistema requería un mucho soporte extra para poder conectar todo lo ya existente.\\

Todo esto, quedó saldado con el standar USB, que debido a la gran cuota de mercado de sus desarrolladores, rápidamente fue introducido y se transformó en la norma a la hora de seleccionar un protocolo. Al punto tal esto se cumplió que hoy, más de 20 años después, es muy dificil encontrar PC's con otro tipo de puertos, salvo que en el momento de su compra uno requiera un puerto específicamente. Sin embargo, por norma, cualquier PC nueva dispónible en le mercado debe poseer puertos USB para la conexión de los periféricos.\\

Desde el punto de vista técnico, el protocolo USB es un sistema del tipo maestro-esclavo, donde el maestro, denominado HOST, debe ser necesariamente una PC y cualquier periférico a ella acoplada será un esclavo.\\

Para describirlo es conveniente tal vez separar el protocolo en tres partes. Una parte física, en donde se define la conexión mecánica y eléctrica del protocolo y los componentes que intervienen, una capa de protocolo, en donde se define el formato y el marco en el que son enviados los paquetes, como se direccionana y como se comunican entre sí, y una parte lógica, en donde cada componente es visto solamente como un extremo y define como fluyen los datos desde un extremo hacia la PC y viceversa.\\

\subsection{Capa física}
	\subsubsection{Conexión mecánica}
		Mecánicamente, la norma USB 2.0 establece dos tipos de conectores básicos, que con los adelantos tecnológicos posteriores, fueron extendidos a 6 tipos de enchufes y 7 tipos de zócalos.
		En primer lugar, los encufes y zócalos originales fueron el tipo A y el tipo B. El enchufe tipo A es rectangular y posee una lengueta en donde descansan, en forma de tiras, 4 contactos. Estos 4 contactos son los que llevan la señal de alimentación y la señal de datos.\\
		El enchufe tipo B es un trapecio y posee los contactos en sus bordes, distribuidos, dos en la parte superio y dos en la parete inferiro. A su vez, para evitar conexiiones incorrectas, la parte superior del conector se encuentran achatadas. En la Figura \ref{usbstandarplugs} se observan los dos tipos de enchufes que se describen anteriormente.\\
		
		\begin{figure}[h]
			\centering
			\includegraphics[width=0.4\textwidth]{usb_standard_plugs}
			\caption{Conectores USB tipo A y B standard. Agregar referencia}
			\label{usbstandardplugs}
		\end{figure}
		
		Luego, con los adelantos tecnológicos y productos más exigentes en cuanto a dimensiones, tales como smartphones y productos que el usuario acarrea siempre consigo, se vió la necesidad de achicar los conectores, por cuanto los diseñadores de USB lanzaron nuevos tipos de conectores y estos fueron los conectores mini-USB y micro-USB, los cuales también poseen su versión A y B. Es de destacar que estos conectores poseen cables más pequeños que no cumplen con las especificaciones de diseño por ellos mismos. Es por esto que la versión micro-USB dispone de 8 alambres, en lugar de los 4 originales que contenía la norma