Considerando las señales descriptas la Sección \ref{fpga:sigs} y el diseño de la MEF cuyo diagrama de flujo se observa en la Figura \ref{fpga:mef}, se procedió a describir el comportamiento del sistema en VHDL.

% a un sistema que cumple con las especificaciones. Sin embargo, a fin de no dejar señales provistas por la interfaz al aire, se agregan todos FLAGS que brinda el controlador FX2LP en la entidad descripta. Además, se incorporan tres constantes para modificar a criterio del desarrollador las direcciones de entrada y salida, y el ancho del bus de datos, que puede ser de 8 o 16 bits. Por defecto, se utilizan las direcciones y ancho de bus. especificados en el Capitulo \ref{cap:cy}, es decir $''11''$ y $''00''$ como puertos de entrada y salida respectivamente y 16 bits de ancho de bus.
%
%De lo anterior, podemos declarar la entidad que tiene los puertos detallados a continuación:
%
%%\begin{lstlisting}[language=VHDL,backgroundcolor=\color{gray!30}]
%%entity fx2lp_interfaz is
%%	generic(
%%		constant in_ep_addr:  std_logic_vector(1 downto 0) := "00";
%%		constant out_ep_addr: std_logic_vector(1 downto 0) := "11";
%%		constant port_width:  integer := 16
%%	);
%%	port(
%%		reloj: in std_logic;
%%		reset: in std_logic;
%%	-- desde y hacia la interfaz
%%	fdata:    inout std_logic_vector(port_width-1 downto 0);
%%	fifoaddr: out	std_logic_vector(1 downto 0);
%%	sloe: 	  out	std_logic;
%%	slrd:     out	std_logic;
%%	slwr:     out	std_logic;
%%	pktend:   out	std_logic;
%%	-- EP2 isoc in (hacia pc)
%%	-- EP8 bulk out (desde pc)
%%	flaga: in	std_logic;   -- EP2_full--->FLAG_Lleno
%%	flagb: in	std_logic;   -- EP8_empty-->FLAG_Vacio
%%	flagc: in	std_logic;   -- EP8_full--->sin uso
%%	flagd: in	std_logic;   -- EP2_empty-->sin uso
%%	-- desde y hacia el sistema
%%	enviar_dato: in  std_logic;
%%	d_recivido:  out std_logic_vector(port_width-1 downto 0);
%%	d_a_enviar:  in  std_logic_vector(port_width-1 downto 0)
%%);
%%end fx2lp_interfaz;
%%\end{lstlisting}

Conceptualmente, una MEF se compone tres partes, el estado actual, la función del próximo estado y la función de salida. Cada uno de estas partes puede ser descripta en VHDL todo junto en un mismo proceso, o bien en procesos separados. Este trabajo fue implementado mediante un proceso para la función de próximo estado y otro para actualizar el registro del estado actual. Las funciones de salida se implementaron mediante estados combinacionales.

\begin{figure}[t]
	\centering
	\begin{tikzpicture}[ask/.style = {diamond,text width=70,draw=black,align=center,aspect=2},
	scale=.7]
		\begin{scope}[transform shape,node distance=(1 and 2),>=latex,]
			\node[moore,text width=80,align=center] (inicio) {Inicio};
		
			\node[ask] (vacio1) [below=of inicio]{FLAG\_Vacío};
				\draw[->] (inicio.south) -| (vacio1);
		%			\node[moore,text width=100] (lecdir) [below=of vacio1,label=above right:dirección] {FIFOADR=entrada\\SLOE=$'0'$\\SLRD=$'1'$\\SLWR=$'1'$};
		%			\draw[o->](vacio1.east) -- ($(vacio1.east)+(1,0)$);
		
			\node[moore,text width=80,align=center] (lecoe) [right=of vacio1]{lec\_direccion};
				\draw[o->] (vacio1.east) -- ($(vacio1.east)!0.5!(lecoe.west)$);
				\draw[->]($(vacio1.east)!0.5!(lecoe.west)$) |- ($(lecoe.north)+(0,0.5)$) -- (lecoe.north);
		
			\node[moore,text width=80,align=center](lecrd)[below=of lecoe]{lectura};
				\draw[->](lecoe) -- (lecrd);
		
			\node[ask] (vacio2)[below=of lecrd]{FLAG\_Vacío};
				\draw[->](lecrd) -- (vacio2);
				
				\draw[->](vacio2.west) -| ($(lecoe.west)!0.5!(vacio1.east)$);
				\draw[o->](vacio2.east) -- ++(1.2,0) |- ($(inicio.north)+(0,.6)$);
				\draw[->] ($(inicio.north)+(0,.6)$) -- (inicio.north);
		
			\node[ask](enviar1)[below=of vacio1]{Enviar\_datos};
				\draw[->](vacio1.south) --(enviar1.north);
		
		
			\node[ask] (lleno1) [below=of enviar1]{FLAG\_Lleno};
				\draw[->](enviar1) -- (lleno1);
		
			\node[moore,text width=80,align=center](escdir)[below=of lleno1]{esc\_direccion};
				\draw[o->](lleno1) -- ($(lleno1.south)!0.5!(escdir.north)$);
				\draw[->]($(lleno1.south)!0.5!(escdir.north)$) -- (escdir);
		
		
			\node[ask](vacio3)[left=2.2 of vacio1]{FLAG\_Vacío};
				\draw[->](escdir.south)--($(escdir.south)+(0,-.5)$) -| ($(vacio3.east)+(.6,0)$) |- ($(vacio3.north)+(0,.5)$)-|(vacio3.north);
		
			\node[ask](enviar2)[below=of vacio3]{Enviar\_datos};
				\draw[->](vacio3) -- (enviar2);
		
				\draw[o->] (enviar1.west) -- ($(enviar1.west)+(-.7,0)$);
				\draw[->] ($(enviar1.west)-(.7,0)$) -- ($(inicio.north -| enviar1.west)+(-.7,.6)$);
				\draw[->] ($(inicio.north -| enviar1.west)+(-.7,.6)$)--($(inicio.north)+(0,.6)$);
				\draw[->] ($(inicio.north)+(0,.6)$) -- (inicio.north);
				\draw[->](lleno1.west) -| ($(enviar1.west)-(.7,0)$);
				\draw[o->](vacio3.west) -- ($(vacio3.west)+(-.9,0)$);
				\draw[->]($(vacio3.west)+(-.9,0)$) |- ($(inicio.north -| enviar1.west)+(-.7,.6)$);
		
			\node[ask](lleno2)[below=of enviar2]{FLAG\_Lleno};
			\node[moore,text width=80,align=center](escwr)[below=of lleno2]{escritura};
				\draw[o->](lleno2) -- (escwr);
				\draw[->](lleno2.west) -| ($(enviar2.west)+(-.9,0)$);
				\draw[->](enviar2)--(lleno2);
				\draw[o->](enviar2)--($(enviar2.west)+(-.9,0)$);
				\draw[->]($(enviar2.west)+(-.9,0)$) -- ($(vacio3.west)+(-.9,0)$);
				\draw[->](escwr) -- ($(escwr.south)+(0,-1)$) -| ($(escdir.east)+(.5,0)$) |- ($(lleno1.south)!0.5!(escdir.north)$);
		\end{scope}
	\end{tikzpicture}
	\caption{Diagrama de flujo de la máquina de estados desarrollada}
	\label{la figura}
\end{figure}

De esta forma, se presenta a continuación la función de próximo estado. Para su mejor comprensión, se puede utilizar la Figura \ref{la figura} en donde se observa una simplificación de cada uno de los estados del diagrama en bloques de la Figura \ref{fpga:mef}, en donde se quitaron las variables de salida y se incorporó en cada uno de los estados el nombre con el que se lo asigna en el código VHDL desarrollado. Además, para facilitar la lectura del desarrollo, se colocaron las dos señales de entrada {\it FLAG\_Vacío} y {\it FLAG\_Lleno} como activos en alto.
%El estilo elegido para la descripción cuenta con un registro que determina el estado próximo de la MAE a través de un proceso secuencial. El valor de dicho registro, es volcado a otro que indica el estado actual de la MAE, a través de los flancos del reloj, en una secuencia diferente. Las señales de salida son implementadas en forma concurrente, de manera externa a los procesos que comanda la MEA. Las señales de entrada se encuentran incorporadas en el proceso que determina el estado próximo.
%La MEA es descripta a través del código que se muestra a continuación:

\begin{lstlisting}[language=VHDL,backgroundcolor=\color{gray!30}]
architecture Behavioral of fx2lp_interfaz is
	-- maquina de estados de la interfaz
	type estados_mef is
	(
		inicio,
		lec_direccion, lectura,
		esc_direccion, escritura
	);
	signal estado_actual, prox_estado: estados_mea := inicio;
begin
	--implementacion de funcion de proximo estado
	proximo_estado: process(estado_actual, flag_lleno,
	 flag_vacio, enviar_dato)
	begin
		case estado_actual is
			when inicio =>
				if flag_vacio = '0' then
					prox_estado <= lec_direccion;
				elsif enviar_dato = '1' then
					if flag_lleno = '0' then
						prox_estado <= esc_direccion;
					else
						prox_estado <= inicio;
					end if;
				else
					prox_estado <= inicio;
				end if;

			when lec_direccion =>
				prox_estado <= lectura;

			when lectura =>
				if flag_vacio = '0' then
					prox_estado <= lec_direccion;
				else
					prox_estado <= inicio;
				end if;

			when esc_direccion =>
				prox_estado <= escritura;

			when escritura =>
				if enviar_dato = '1' then
					if flag_vacio = '1' and flag_lleno = '0' then
						prox_estado <= esc_direccion;
					else
						prox_estado <= inicio;
					end if;
				else
					prox_estado <= inicio;
				end if;

			when others =>
				prox_estado <= inicio;
		end case;
	end process proximo_estado;
end Behavioral;
\end{lstlisting}

Como se menciona anteriormente, las señales {\it FLAG\_Vacío} y {\it FLAG\_Lleno} se hicieron activos en alto. Sin embargo, las señales que toma la interfaz son activas en bajo. Entonces, se asignaron las señales mencionadas a los puertos {\it flaga} y {\it flagb} mediante un inversor. Todo esto apuntó a facilitar la lectura y el desarrollo de la descripción.

\begin{lstlisting}[language=VHDL,backgroundcolor=\color{gray!30}]
architecture Behavioral of fx2lp_interfaz is
	signal flag_vacio: std_logic;
	signal flag_lleno: std_logic;
begin
	flag_lleno  <= not flaga;
	flag_vacio <= not flagb;
end Behavioral;	
\end{lstlisting}

La función de salida se implementa con lógica combinacional, utilizando el registro de estado actual. Debido a esto, se describe cada una de las salidas por separado. Como lo que modifica estas salidas son los estados de la MEF definidos, se debe recurrir a señales que sirvan como conectores internos desde los puertos hacia los diferentes componentes que se describen.

\begin{lstlisting}[language=VHDL,backgroundcolor=\color{gray!30}]
architecture Behavioral of fx2lp_interfaz is
	signal slwr_int:  	 std_logic := '1';
	signal slrd_int:  	 std_logic := '1';
	signal sloe_int:  	 std_logic := '1';
	signal pktend_int:	 std_logic := '1';
	signal faddr_int:	 std_logic_vector(1 downto 0) := "ZZ";
	signal fdata_sal:	 std_logic_vector(port_width-1 downto 0);
	signal fdata_inent:	 std_logic_vector(port_width-1 downto 0);
	signal reloj_sitema: std_logic;
begin
	reloj_sistema <= reloj;
	slwr   <= slwr_int;
	slrd   <= slrd_int;
	sloe   <= sloe_int;
	faddr  <= faddr_int;
	pktend <= pktend_int;
	d_recibido <= fdata_ent;
	fdata_sal <= d_a_enviar;
	
end Behavioral
\end{lstlisting}

Con todas las señales definidas y asignadas, y la máquina de estados que se detalló anteriormente, se pueden asignar las señales de salida:

\begin{lstlisting}[language=VHDL,backgroundcolor=\color{gray!30}]
architecture Behavioral of fx2lp_interfaz is
	with estado_actual select
		faddr_int <=	out_ep_addr when lec_direccion | lectura,
						in_ep_addr  when esc_direccion | escritura,
						(others => 'Z') when others;
	
	slwr_int <=	'0' when prox_estado = esc_direccion else
				'1';
	
	slrd_int <= '0' when estado_actual = lec_direccion else
	'1';
	
	pktend_int <= ((not falg_vacio) or enviar_dato);
	
	with estado_actual select
	sloe_int <=	'0' when lectura | lec_direccion,
				'1' when others;
	
	with estado_actual select
		fdata <=	fdata_sal        when escritura | esc_direccion,
					(others => 'Z')  when others;
	
	with estado_actual select
		fdata_ent <=	fdata     when lectura | lec_direccion,
						fdata_ent  when others;
end Behavioral
\end{lstlisting}

Finalmente, resta el reloj que hace avanzar la MAE. A este reloj, se le acoplan dos temporizadores de habilitación. Esto se debe a que se espera que el sistema trabaje a 50 MHz. Sin embargo, para respetar los tiempos de establecimiento y ancho de pulso de las distintas señales\cite{Cypress2017}, cuando el próximo estado es esc\_dirección se deben esperar tres ciclos de reloj y en el caso de que el próximo estado sea escritura, lec\_direccion o lectura, se debe esperar dos ciclos de reloj.
Esto se implementa con dos contadores diferentes, los cuales habilitan o no el cambio de estado. Esto se detalla a continuación:

\begin{lstlisting}[language=VHDL,backgroundcolor=\color{gray!30}]
architecture Behavioral of fx2lp_interfaz is
	signal cont3:	 natural range 0 to 4 := 0;
	signal cont2:	 natural range 0 to 3 := 0;
	signal disparo3: std_logic := '0';
	signal disparo2: std_logic := '0';
begin
	contador3: process(reloj_sistema, reset, disparo3)
	begin
		if reset = '0' then
			cont3 <= 0;
		elsif rising_edge(reloj_sistema) then
			if cont3 > 0 then
				cont3 <= cont3 - 1;
			elsif disparo3 = '1' then
				cont3 <= 4;
			end if;
		end if;
	end process contador3;

	disparo3 <= '1' when (prox_estado = esc_direccion) else '0';
	
	counter2: process(reloj_sistema, reset, disparo2)
	begin
		if reset = '0' then
			cont2 <= 0;
		elsif rising_edge(reloj_sistema)then
			if cont2 > 0 then
				cont2 <= count2 - 1;
			elsif disparo2 = '1' then
				cont2 <= 3;
			end if;
		end if;
	end process contador2;
	
	with prox_estado select
	disparo2 <=	'1' when lec_direccion | lectura | esc_direccion,
				'0' when others;

	reloj_mea: process (reloj_sistema, reset)
	begin
		if reset = '0' then
				estado_actual <= idle;
		elsif rising_edge(reloj_sistema) then
			if cont2 = 0 and cont3 = 0 then
				estado_actual <= prox_estado;
			end if;
		end if;
	end process reloj_mea;
end Behavioral
\end{lstlisting}

El código completo se puede encontrar en el Anexo \ref{ap:vhdl}
