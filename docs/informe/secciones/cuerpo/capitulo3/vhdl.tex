Considerando las señales que se mencionaron la Sección \ref{fpga:sigs} y el diseño de la MEF cuyo diagrama de flujo se observa en la Figura \ref{fpga:mef:diag}, se procedió a describir el comportamiento del sistema en VHDL. La Figura~\ref{fpga:mef:concepto} muestra la estructura conceptual de una MEF. Se puede apreciar que una MEF se compone de tres partes: la función del próximo estado, el estado actual y la función de salida. Estas partes pueden ser descriptas en VHDL agrupadas en un mismo proceso, o bien cada una de ellas en un proceso diferente. Este trabajo fue implementado mediante un proceso para la función de próximo estado y otro para actualizar el registro del estado actual. Las funciones de salida se implementaron mediante declaraciones concurrentes.

% a un sistema que cumple con las especificaciones. Sin embargo, a fin de no dejar señales provistas por la interfaz al aire, se agregan todos FLAGS que brinda el controlador FX2LP en la entidad descripta. Además, se incorporan tres constantes para modificar a criterio del desarrollador las direcciones de entrada y salida, y el ancho del bus de datos, que puede ser de 8 o 16 bits. Por defecto, se utilizan las direcciones y ancho de bus. especificados en el Capitulo \ref{cap:cy}, es decir $''11''$ y $''00''$ como puertos de entrada y salida respectivamente y 16 bits de ancho de bus.
%
%De lo anterior, podemos declarar la entidad que tiene los puertos detallados a continuación:
%
%%\begin{lstlisting}[language=VHDL,backgroundcolor=\color{gray!30}]
%%entity fx2lp_interfaz is
%%	generic(
%%		constant in_ep_addr:  std_logic_vector(1 downto 0) := "00";
%%		constant out_ep_addr: std_logic_vector(1 downto 0) := "11";
%%		constant port_width:  integer := 16
%%	);
%%	port(
%%		reloj: in std_logic;
%%		reset: in std_logic;
%%	-- desde y hacia la interfaz
%%	fdata:    inout std_logic_vector(port_width-1 downto 0);
%%	fifoaddr: out	std_logic_vector(1 downto 0);
%%	sloe: 	  out	std_logic;
%%	slrd:     out	std_logic;
%%	slwr:     out	std_logic;
%%	pktend:   out	std_logic;
%%	-- EP2 isoc in (hacia pc)
%%	-- EP8 bulk out (desde pc)
%%	flaga: in	std_logic;   -- EP2_full--->FLAG_Lleno
%%	flagb: in	std_logic;   -- EP8_empty-->FLAG_Vacio
%%	flagc: in	std_logic;   -- EP8_full--->sin uso
%%	flagd: in	std_logic;   -- EP2_empty-->sin uso
%%	-- desde y hacia el sistema
%%	enviar_dato: in  std_logic;
%%	d_recivido:  out std_logic_vector(port_width-1 downto 0);
%%	d_a_enviar:  in  std_logic_vector(port_width-1 downto 0)
%%);
%%end fx2lp_interfaz;
%%\end{lstlisting}

\begin{figure}[t]
	\begin{tikzpicture}[scale=1]
		\begin{scope}[transform shape,>=latex,node distance=1.5]
			\node[bloque](prEs){Función Estado Próximo};
			\node[bloque](acEs)[right=of prEs]{Estado Actual};
			\node[bloque](sal)[right=of acEs]{Función de Salida};
			\node[minimum height=30](ent)[left=of prEs,yshift=15,]{Entradas};
			\draw[decoration={brace},decorate,ultra thick] (ent.north east) -- (ent.south east);
			
			\draw[lineaExt](prEs)to(acEs);
			\draw[lineaExt](acEs)to(sal);
			\draw[lineaExt](acEs) -- ($(acEs)!.45!(sal)$) |- ($(prEs.south) - (0,.5)$) -| ([yshift=-15]$(prEs.west)-(1.3,0)$) |- ([yshift=-15]prEs.west);
			\draw[lineaInt](prEs)to(acEs);
			\draw[lineaInt](acEs) -- ($(acEs)!.45!(sal)$) |- ($(prEs.south) - (0,.5)$) -| ([yshift=-15]$(prEs.west)-(1.3,0)$) |- ([yshift=-15]prEs.west);
			\draw[lineaInt](acEs)to(sal);
			\draw[lineaExt,shorten <= 4](ent) -- ([yshift=15]prEs.west);
			\draw[lineaInt,shorten <= 5.5](ent) -- ([yshift=15]prEs.west);
		\end{scope}
	\end{tikzpicture}
	\caption{Estructura de una máquina de estados finitos}
	\label{fpga:mef:concepto}
\end{figure}


\begin{figure}[!hb]
	\centering
	\begin{tikzpicture}[ask/.style = {diamond,text width=70,draw=black,align=center,aspect=2},
	scale=.7]
		\begin{scope}[transform shape,node distance=(1 and 2),>=latex,]
			\node[moore,text width=80,align=center] (inicio) {Inicio};
		
			\node[ask] (vacio1) [below=of inicio]{FLAG\_Vacío};
				\draw[->] (inicio.south) -| (vacio1);
		%			\node[moore,text width=100] (lecdir) [below=of vacio1,label=above right:dirección] {FIFOADR=entrada\\SLOE=$'0'$\\SLRD=$'1'$\\SLWR=$'1'$};
		%			\draw[o->](vacio1.east) -- ($(vacio1.east)+(1,0)$);
		
			\node[moore,text width=80,align=center] (lecoe) [right=of vacio1]{lec\_direccion};
				\draw[o->] (vacio1.east) -- ($(vacio1.east)!0.5!(lecoe.west)$);
				\draw[->]($(vacio1.east)!0.5!(lecoe.west)$) |- ($(lecoe.north)+(0,0.5)$) -- (lecoe.north);
		
			\node[moore,text width=80,align=center](lecrd)[below=of lecoe]{lectura};
				\draw[->](lecoe) -- (lecrd);
		
			\node[ask] (vacio2)[below=of lecrd]{FLAG\_Vacío};
				\draw[->](lecrd) -- (vacio2);
				
				\draw[->](vacio2.west) -| ($(lecoe.west)!0.5!(vacio1.east)$);
				\draw[o->](vacio2.east) -- ++(1.2,0) |- ($(inicio.north)+(0,.6)$);
				\draw[->] ($(inicio.north)+(0,.6)$) -- (inicio.north);
		
			\node[ask](enviar1)[below=of vacio1]{Enviar\_datos};
				\draw[->](vacio1.south) --(enviar1.north);
		
		
			\node[ask] (lleno1) [below=of enviar1]{FLAG\_Lleno};
				\draw[->](enviar1) -- (lleno1);
		
			\node[moore,text width=80,align=center](escdir)[below=of lleno1]{esc\_direccion};
				\draw[o->](lleno1) -- ($(lleno1.south)!0.5!(escdir.north)$);
				\draw[->]($(lleno1.south)!0.5!(escdir.north)$) -- (escdir);
		
		
			\node[ask](vacio3)[left=2.2 of vacio1]{FLAG\_Vacío};
				\draw[->](escdir.south)--($(escdir.south)+(0,-.5)$) -| ($(vacio3.east)+(.6,0)$) |- ($(vacio3.north)+(0,.5)$)-|(vacio3.north);
		
			\node[ask](enviar2)[below=of vacio3]{Enviar\_datos};
				\draw[->](vacio3) -- (enviar2);
		
				\draw[o->] (enviar1.west) -- ($(enviar1.west)+(-.7,0)$);
				\draw[->] ($(enviar1.west)-(.7,0)$) -- ($(inicio.north -| enviar1.west)+(-.7,.6)$);
				\draw[->] ($(inicio.north -| enviar1.west)+(-.7,.6)$)--($(inicio.north)+(0,.6)$);
				\draw[->] ($(inicio.north)+(0,.6)$) -- (inicio.north);
				\draw[->](lleno1.west) -| ($(enviar1.west)-(.7,0)$);
				\draw[o->](vacio3.west) -- ($(vacio3.west)+(-.9,0)$);
				\draw[->]($(vacio3.west)+(-.9,0)$) |- ($(inicio.north -| enviar1.west)+(-.7,.6)$);
		
			\node[ask](lleno2)[below=of enviar2]{FLAG\_Lleno};
			\node[moore,text width=80,align=center](escwr)[below=of lleno2]{escritura};
				\draw[o->](lleno2) -- (escwr);
				\draw[->](lleno2.west) -| ($(enviar2.west)+(-.9,0)$);
				\draw[->](enviar2)--(lleno2);
				\draw[o->](enviar2)--($(enviar2.west)+(-.9,0)$);
				\draw[->]($(enviar2.west)+(-.9,0)$) -- ($(vacio3.west)+(-.9,0)$);
				\draw[->](escwr) -- ($(escwr.south)+(0,-1)$) -| ($(escdir.east)+(.5,0)$) |- ($(lleno1.south)!0.5!(escdir.north)$);
		\end{scope}
	\end{tikzpicture}
	\caption{Diagrama de simplificado del flujo de la MEF}
	\label{fpga:mef:simple}
\end{figure}

%El desarrollo de las descripciones en VHDL, las verificaciones funcionales y la síntesis realizada a lo largo de este trabajo fueron realizados a través del entorno de desarrollo ISE provisto por Xilinx. Los códigos obtenidos, como así también los resultados arrojados por este entorno de desarrollo se plasman en el Anexo~\ref{ap:vhdl}

Se presentan a continuación la función de próximo estado y la actualización del estado actual debido a que son, a criterio del autor, los procesos más importantes del desarrollo. El código completo, en donde se realiza la declaración de la entidad, la declaración de las señales, las conexiones internas y se asignan las funciones de salida a los estados en forma concurrente, se puede encontrar en el Apéndice \ref{ap:vhdl}.

La función de próximo estado, es un proceso que lee el registro \textit{estado\_actual} y las señales de entrada y, en función de su valor, asigna el estado siguiente al registro \textit{prox\_estado}. Para una mejor comprensión de la descripción de la función de próximo estado, se puede utilizar la Figura \ref{fpga:mef:simple} en donde se observa una simplificación de cada uno de los estados del diagrama en bloques de la Figura \ref{fpga:mef:diag}, en donde se quitaron las variables de salida y se incorporó en cada uno de los estados el nombre con el que se lo asigna en el código VHDL desarrollado. Además, para facilitar la lectura del desarrollo, se colocaron las dos señales de entrada \textit{FLAG\_Vacío} y \textit{FLAG\_Lleno} como activos en alto.
%El estilo elegido para la descripción cuenta con un registro que determina el estado próximo de la MAE a través de un proceso secuencial. El valor de dicho registro, es volcado a otro que indica el estado actual de la MAE, a través de los flancos del reloj, en una secuencia diferente. Las señales de salida son implementadas en forma concurrente, de manera externa a los procesos que comanda la MEA. Las señales de entrada se encuentran incorporadas en el proceso que determina el estado próximo.
%La MEA es descripta a través del código que se muestra a continuación:

\begin{lstlisting}[language=VHDL,backgroundcolor=\color{gray!30}]
architecture Behavioral of fx2lp_interfaz is
	-- maquina de estados de la interfaz
	type estados_mef is
	(
		inicio,
		lec_direccion, lectura,
		esc_direccion, escritura
	);
	signal estado_actual, prox_estado: estados_mef := inicio;
begin
	--implementacion de funcion de proximo estado
	proximo_estado: process(estado_actual, flag_lleno,
	 flag_vacio, enviar_dato)
	begin
		case estado_actual is
			when inicio =>
				if flag_vacio = '0' then
					prox_estado <= lec_direccion;
				elsif enviar_dato = '1' then
					if flag_lleno = '0' then
						prox_estado <= esc_direccion;
					else
						prox_estado <= inicio;
					end if;
				else
					prox_estado <= inicio;
				end if;

			when lec_direccion =>
				prox_estado <= lectura;

			when lectura =>
				if flag_vacio = '0' then
					prox_estado <= lec_direccion;
				else
					prox_estado <= inicio;
				end if;

			when esc_direccion =>
				prox_estado <= escritura;

			when escritura =>
				if enviar_dato = '1' then
					if flag_vacio = '1' and flag_lleno = '0' then
						prox_estado <= esc_direccion;
					else
						prox_estado <= inicio;
					end if;
				else
					prox_estado <= inicio;
				end if;

			when others =>
				prox_estado <= inicio;
		end case;
	end process proximo_estado;
end Behavioral;
\end{lstlisting}

%Como se menciona anteriormente, las señales {\it FLAG\_Vacío} y {\it FLAG\_Lleno} se hicieron activos en alto. Sin embargo, las señales que toma la interfaz son activas en bajo. Entonces, se asignaron las señales mencionadas a los puertos {\it flaga} y {\it flagb} mediante un inversor. Todo esto apuntó a facilitar la lectura y el desarrollo de la descripción.

%\begin{lstlisting}[language=VHDL,backgroundcolor=\color{gray!30}]
%architecture Behavioral of fx2lp_interfaz is
%	signal flag_vacio: std_logic;
%	signal flag_lleno: std_logic;
%begin
%	flag_lleno  <= not flaga;
%	flag_vacio <= not flagb;
%end Behavioral;	
%\end{lstlisting}

%La función de salida se implementa con lógica combinacional, utilizando el registro de estado actual. Debido a esto, se describe cada una de las salidas por separado. Como lo que modifica estas salidas son los estados de la MEF definidos, se debe recurrir a señales que sirvan como conectores internos desde los puertos hacia los diferentes componentes que se describen.

%\begin{lstlisting}[language=VHDL,backgroundcolor=\color{gray!30}]
%architecture Behavioral of fx2lp_interfaz is
%	signal slwr_int:  	 std_logic := '1';
%	signal slrd_int:  	 std_logic := '1';
%	signal sloe_int:  	 std_logic := '1';
%	signal pktend_int:	 std_logic := '1';
%	signal faddr_int:	 std_logic_vector(1 downto 0) := "ZZ";
%	signal fdata_sal:	 std_logic_vector(port_width-1 downto 0);
%	signal fdata_inent:	 std_logic_vector(port_width-1 downto 0);
%	signal reloj_sitema: std_logic;
%begin
%	reloj_sistema <= reloj;
%	slwr   <= slwr_int;
%	slrd   <= slrd_int;
%	sloe   <= sloe_int;
%	faddr  <= faddr_int;
%	pktend <= pktend_int;
%	d_recibido <= fdata_ent;
%	fdata_sal <= d_a_enviar;
%	
%end Behavioral
%\end{lstlisting}

%Con todas las señales definidas y asignadas, y la máquina de estados que se detalló anteriormente, se pueden asignar las señales de salida:
%
%\begin{lstlisting}[language=VHDL,backgroundcolor=\color{gray!30}]
%architecture Behavioral of fx2lp_interfaz is
%	with estado_actual select
%		faddr_int <=	out_ep_addr when lec_direccion | lectura,
%						in_ep_addr  when esc_direccion | escritura,
%						(others => 'Z') when others;
%	
%	slwr_int <=	'0' when prox_estado = esc_direccion else
%				'1';
%	
%	slrd_int <= '0' when estado_actual = lec_direccion else
%	'1';
%	
%	pktend_int <= ((not falg_vacio) or enviar_dato);
%	
%	with estado_actual select
%	sloe_int <=	'0' when lectura | lec_direccion,
%				'1' when others;
%	
%	with estado_actual select
%		fdata <=	fdata_sal        when escritura | esc_direccion,
%					(others => 'Z')  when others;
%	
%	with estado_actual select
%		fdata_ent <=	fdata     when lectura | lec_direccion,
%						fdata_ent  when others;
%end Behavioral
%\end{lstlisting}

La transición entre estados es un proceso que consta de un reloj que  transfiere el valor del registro de \textit{prox\_estado} al registro \textit{estado\_actual}. A este reloj, se le acoplan dos temporizadores de habilitación. Esto se debe a que se algunas señales deben respetar ciertos tiempos de establecimiento y ancho de pulso~\cite{Cypress2017}. Cuando el próximo estado es \textit{esc\_dirección} se deben esperar tres ciclos de reloj y en el caso de que el próximo estado sea escritura, \textit{lec\_direccion} o lectura, se debe esperar dos ciclos de reloj.
Esto se implementa con dos contadores diferentes, los cuales habilitan o no el cambio de estado, lo que se detalla a continuación:

\begin{lstlisting}[language=VHDL,backgroundcolor=\color{gray!30}]
architecture Behavioral of fx2lp_interfaz is
	signal contador3	:natural range 0 to 4 := 0;
	signal contador2	:natural range 0 to 3 := 0;
	signal disp3		:std_logic := '0';
	signal disp2		:std_logic := '0';
begin
	reloj_mef: process (reloj_sist, reset)
	begin
		if reset = '0' then
			estado_actual <= inicio;
		elsif rising_edge(reloj_sist) then
			if contador2 = 0 and contador3 = 0 then
				estado_actual <= prox_estado;
			end if;
		end if;
	end process reloj_mef;
	
	tempo3: process(reloj_sist, reset, disp3)
	begin
		if reset = '0' then
			contador3 <= 0;
		elsif rising_edge(reloj_sist) then
			if contador3 > 0 then
				contador3 <= contador3 - 1;
			elsif disp3 = '1' then
				contador3 <= 4;
			end if;
		end if;
	end process tempo3;

	disp3 <= '1' when (prox_estado = esc_direccion) else '0';
	
	tempo2: process(reloj_sist, reset, disp2)
	begin
		if reset = '0' then
			contador2 <= 0;
		elsif rising_edge(reloj_sist)then
			if contador2 > 0 then
				contador2 <= contador2 - 1;
			elsif disp2 = '1' then
				contador2 <= 3;
			end if;
		end if;
	end process tempo2;
	
	with prox_estado select
	disp2 <=	'1' when lec_direccion | lectura | esc_direccion,
				'0' when others;

end Behavioral
\end{lstlisting}