En el Capítulo anterior, se detalla como se logró intercambiar datos entre una PC y el controlador FX2LP a través del protocolo \acrshort{usb}. Este enlace constituye una primer etapa del desarrollo realizado. En una segunda etapa, se debe lograr la comunicación de datos entre el \acrshort{fpga} y el controlador FX2LP de Cypress. Así, los datos estarían en condiciones de fluir entre el Host y el \acrshort{fpga}, a través de la interfaz \acrshort{usb}, utilizando las memorias \acrshort{fifo} del controlador FX2LP. 

Para diseñar un sistema de comunicación entre el \acrshort{fpga} y la interfaz \acrshort{usb}, es necesario identificar cuales son los protocolos a través de los cuales se leen y escriben datos en el controlador FX2LP, como así también las señales y los puertos con que debe interactuar el \acrshort{fpga}. Una vez identificado el procedimiento, se pudo desarrollar el sistema de comunicación, que fue implementado en una \acrfull{mef}, a través de la cual, se leen las señales que provienen del controlador FX2LP y se generan las señales necesarias para comandar su memoria \acrshort{fifo}.

%la que fue realizada en VHDL e implementada en el FPGA

%\begin{figure}
%	\centering
%	\begin{tikzpicture}[scale=1.35*\textwidth/\paperwidth,>=latex,node distance=3]
%	\begin{scope}[transform shape,node distance=1.5,align=center]
%	\node[interior,text width=60](mef)	{M\'aquina de Estados Finitos};
%	\node	(aux1)	[right=of mef]	{};
%	\node[interior,text width=80](dev)		[right=of aux1]	{Desarrollo implementado en FPGA};
%	\draw[<->] (mef.east) -- node[above]{Flujo de Datos} (dev.west);
%	\node[node distance = .5]	(texto fpga) [below=of aux1] {FPGA};
%	\end{scope}
%	\begin{scope}[on background layer]
%	\node[fit=(mef)(dev)(texto fpga),rectangle,draw](fpga){};
%	\end{scope}
%	\begin{scope}[transform shape,]
%	\node[exterior,text width=70,align=center,minimum height=65] (cy)	[left=of fpga]{Controlador FX2LP};
%	\end{scope}
%	
%	\begin{scope}[transform shape]
%	\draw[<->] (cy.20) to node[above]{Datos} (cy.20 -| fpga.west);
%	\draw[<->] (cy.-20) to node[above] {Control} (cy.-20 -| fpga.west);
%	\end{scope}
%	
%	\end{tikzpicture}
%	\caption{Esquema conceptual del flujo de datos desde la PC hasta el controlador}
%	\label{fpga:concepto}
%\end{figure}

%La Figura \ref{fpga:concepto} muestra un esquema en el cual se observa, como productor y consumidor de datos, un desarrollo genérico, implementado dentro del FPGA. Los datos fluyen desde el FPGA al controlador FX2LP a través de una máquina de estados finitos (MEF), que también provee las señales de control.

En este Capítulo, se justifica la elección del \acrshort{fpga} y la placa de desarrollo utilizados para la implementación del sistema de comunicación. También se detallan las señales que intervienen en el funcionamiento de la interfaz \acrshort{usb} y los protocolos de lectura y escritura de modo asíncrono. A continuación se desarrolla el diseño de la \acrshort{mef} y su descripción en lenguaje \acrshort{vhdl}, para su posterior síntesis en el \acrshort{fpga}. Luego, se exponen la verificación funcional del sistema descripto. Además, se explica el desarrollo de un circuito impreso utilizado para la interconexión entre las distintas placas de desarrollo que se utilizan en este trabajo.
%A lo largo de este capítulo se explica en detalle el protocolo que debe seguir el dispositivo maestro que comanda la lectura y escritura de datos en los buffers disponibles en la memoria FIFO esclava que posee el controlador de Cypress. Luego, se desarrolla la máquina de estados que comanda ese intercambio de datos y el código escrito en VHDL que la describe para ser sintetizada en el FPGA. Finalmente se explica el desarrollo de un circuito desarrollado como interconexión entre las distintas placas de desarrollo que se utilizan en este trabajo.