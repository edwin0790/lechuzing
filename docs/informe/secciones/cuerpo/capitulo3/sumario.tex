Durante el presente capítulo se justificó la selección de la placa Mojo v3, la cual posee incorporado un \acrshort{fpga} Spartan 6 desarrollado por Xilinx para la realización de este trabajo.

Se expuso también cuales son las señales de control que intervienen y como es su funcionamiento durante las operaciones de lectura y escritura externa en las memorias \acrshort{fifo} del controlador FX2LP.

Con base en el mecanismo que siguen las señales de control en las operaciones mencionadas se explicó el diseño de una \acrshort{mef} que intercambia datos entre un sistema implementado en \acrshort{fpga} y el controlador FX2LP. Se realizó una descripción física de la \acrshort{mef}, utilizando \acrshort{vhdl} como lenguaje de descripción, para su posterior síntesis en el \acrshort{fpga} Spartan 6. Se realizó también una verificación funcional de la síntesis realizada, a fin de corroborar su correcto funcionamiento.

Finalmente se detalló la placa de interconexión realizada para la conexión eléctrica de las placas de desarrollo que contienen al controlador FX2LP y al FPGA Spartan 6 y como esta \acrshort{pcb} determina la correlación entre los puertos de cada uno de los circuitos integrados.