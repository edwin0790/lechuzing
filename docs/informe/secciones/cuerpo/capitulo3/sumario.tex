\begin{figure}
	\centering
		\begin{tikzpicture}[]
			\begin{scope}[transform shape,node distance=1,>=latex]
				\node[rectangle, rounded corners,draw=black,minimum size=40](memo){Memoria};
				\node[](aux01)[right=of memo]{};
				\node[align=center,](comFIFO)[above=of aux01]{Comunicacion\\Interfaz - FPGA};
				\node[exterior,align=center](fpga)[right=of aux01]{FPGA\\\bf{Spartan-6}};
				\node[rectangle, rounded corners, draw=black, minimum size=40,align=center](trans)[left=of memo]{Transceptor\\USB};
				\node[node distance=.5](aux02)[left=of memo]{};
				\node[align=center](interfaz)[below=of aux02]{Interfaz\\\bf{EZ-USB FX2LP}};
				\node[](aux03)[left=of trans]{};
				\node[align=center](comPC)[above=of aux03]{Comunicación\\PC - Interfaz};
				\node[exterior,align=center](pc)[left=of aux03]{PC\\\bf{\verb|libusb|}};
				\draw[thick,<->] (fpga) to (memo);
				\draw[thick,<->] (pc) to (trans);
			\end{scope}
			\begin{scope}[]
				\node[rectangle, dashed, draw=black, rounded corners, fit={(fpga)(memo)(comFIFO)}] (parte1) {};
				\node[exterior,fit={(trans)(interfaz)(memo)}](bridge){};
				\node[rectangle, rounded corners, dashed,draw=black, fit={(trans)(pc)(comPC)}](parte3){};
				\node[rectangle, rounded corners, dashed,draw=black, fit={(interfaz)(bridge)}](){};
			\end{scope}
		\end{tikzpicture}
		\caption{Partes en que se desglosa el trabajo con sus componentes}
		\label{fig:comp}
\end{figure}

En este capítulo se resumen las partes y que herramienta compone cada una de ellas. Tal como se observa en la Figura \ref{fig:comp}, el programa de PC que gestionará la comunicación desde ese extremo se realiza en lenguaje C, con la biblioteca \verb|libusb|. Como interfaz, se utiliza el controlador EZ-USB FX2LP, que viene montado en la placa de desarrollo CY3684 EZ-USB FX2LP de Cypress Semiconductor. Finalmente, desde el extremo del FPGA, se utilizará un Spartan-6 de Xilinx, a través de la placa de desarrollo Mojo v3.