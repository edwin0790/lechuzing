Durante el presente capítulo se justificó la selección de la placa Mojo v3, la cuál posee incorporado un FPGA Spartan 6 desarrollado por Xilinx para la realización de este trabajo.

Se expuso también cuáles son las señales de control que intervienen y cómo es su funcionamiento durante las operaciones de lectura y escritura externa en las memorias FIFO del controlador FX2LP.

Con base en el mecanismo que siguen las señales de control en las operaciones mencionadas se explicó el diseño de una MEF que intercambia datos entre un sistema implementado en FPGA y el controlador FX2LP. Se realizó una descripción física de la MEF, utilizando VHDL como lenguaje de descripción, para su posterior síntesis en el FPGA Spartan 6. Se realizó también una verificación funcional de la síntesis realizada, a fin de corroborar su correcto funcionamiento.

Finalmente se detalló la placa de interconexión realizada para la conexión eléctrica de las placas de desarrollo que contienen al controlador FX2LP y al FPGA Spartan 6 y como éste circuito impreso determina la correlación entre los puertos de cada uno de los circuitos integrados.