	libusb es una biblioteca de código abierto, muy bien documentada, escrita en C, que brinda acceso genérico a dispositivos USB. Las características de diseño que persigue el equipo de desarrollo que mantiene la biblioteca es que sea multiplataforma, modo usuario y agnótico de versión.\\
	
	En el sitio web disponible %TODO cita a libusb.info
	se explica lo que significa cada uno de estos conceptos:
	
	\begin{itemize}
		\item{Multiplataforma:} Se apunta a que cualquier software que contenga esta biblioteca pueda ser compilado y ejecutado en la mayo cantidad de plataformas posibles, dotando al software de portabilidad, es decir, ala biblioteca puede ser ejecutada en Windows, Linux, OS X, Android y otras plataformas sin necesidad de realizar cambios en el código.
		\item{Modo usuario:} No se requiere acceso privilegiado de ningún tipo para poder ejecutar programas escritos con esta biblioteca.
		\item{Agnótisco de versión:} Sin importar la versión de la norma USB que se utilice, el programa se podrá comunicar siempre con el dispositivo USB que se requiera.
	\end{itemize}
	
	Otra ventaja que posee la biblioteca libusb es que, al ser de código abierto, posee una gran comunidad que contribuye al crecimiento del proyecto, como así también otros proyectos que utilizan esta biblioteca. Así, existe una gran variedad de ejemplos que facilitan el aprendizaje en su utilización y adaptaciones para diferentes lenguajes de programación, que se adapte a los conocimientos previos de la persona que desarrolla programas.