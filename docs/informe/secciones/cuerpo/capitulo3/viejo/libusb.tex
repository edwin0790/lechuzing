	%TODO falta una introducción para esta sección que la relacione con el objetivo del trabajo
	La tercer parte en la que se divide el trabajo es relativa a la comunicación entre la interfaz y una PC. Ya que la interfaz se encarga en gran medida de lo relativo al empaquetamiento, codificación y decodificación y que las PC, por su parte, vienen equipadas con el hardware necesario, este trabajo debe implementar el software que comande y gestione, desde el sistema operativo el correcto acceso a los datos que se envían y reciben. Para la elaboración de software que permita el manejo de los puertos USB, se utiliza la biblioteca \verb|libusb|.%\\
	
	\verb|libusb| es una biblioteca de código abierto, muy bien documentada, escrita en C, que brinda acceso genérico a dispositivos USB. Las características de diseño que persigue el equipo de desarrollo que mantiene la biblioteca es que sea multiplataforma, modo usuario y agnóstico de versión\cite{libusb}.%\\
		
	\begin{itemize}
		\item{Multiplataforma:} Se apunta a que cualquier software que contenga esta biblioteca pueda ser compilado y ejecutado en la mayor cantidad de plataformas posibles, dotando al software de portabilidad, es decir, esta biblioteca puede ser ejecutada en Windows, Linux, OS X, Android y otras plataformas sin necesidad de realizar cambios en el código.
		\item{Modo usuario:} No se requiere acceso privilegiado de ningún tipo para poder ejecutar programas escritos con esta biblioteca.
		\item{Agnótisco de versión:} Sin importar la versión de la norma USB que se utilice, el programa se podrá comunicar siempre con el dispositivo USB que se requiera.
	\end{itemize}
	
	La biblioteca \verb|libusb| no posee un autor formal. Es decir, no hay una persona, empresa u organización formal que se encargue de la creación y el mantenimiento del software. Existe una comunidad de más de 130 desarrolladores que en forma voluntaria cooperan en el mantenimiento y desarrollo de esta biblioteca. Se garantiza así que el proyecto esté documentado en forma detallada, existiendo amplios ejemplos y tutoriales de su uso.%\\
	
	Se elige esta biblioteca para la realización del software que gestionara el envío y la recepción de datos debido a su amplio soporte, la factibilidad de ejecutarlo en diferentes sistemas operativos y por ser totalmente gratuito.%\\
%	Otra ventaja que posee la biblioteca libusb es que, al ser de código abierto, posee una gran comunidad que contribuye al crecimiento del proyecto, como así también otros proyectos que utilizan esta biblioteca. Así, existe una gran variedad de ejemplos que facilitan el aprendizaje en su utilización y adaptaciones para diferentes lenguajes de programación, que se adapte a los conocimientos previos de la persona que desarrolla programas.