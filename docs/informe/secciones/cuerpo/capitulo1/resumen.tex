En el presente capítulo se expuso la necesidad de elaborar un sistema de comunicación que permita la transferencia de datos entre una PC y un FPGA para ser utilizados por desarrollos científicos implementados con este último dispositivo.

Se propuso utilizar una interfaz comercial como intermediario los dos dispositivos y se brindó una justificación del empleo del protocolo USB 2.0 de alta velocidad para una comunicación óptima que satisfaga los requerimientos.

Además, se repasaron algunos conceptos importantes inherentes a la versión 2.0 de la norma USB.

Se presentaron también los objetivos formales del trabajo y la estructura del presente informe.