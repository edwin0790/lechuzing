El protocolo USB es un sistema de comunicación diseñado durante los años 90 por los seis fabricantes de la industria de las computadoras, Compaq, Intel, Microsoft, Hewlett-Packard, Lucent, NEC y Philips, con la idea de proveer a su industria de un sistema que permita la conexión entre las PC's y los periféricos con un formato estandart, de forma tal que permita la compatibilidad entre los distintos fabricantes.\\

Hasta ese momento, el gran ecosistema de periféricos, sumado a los nuevos avances y desarrollos, hacia muy compleja la conectividad de todos ellos. Cada uno de los fabricantes desarrollaba componmentes con fichas, niveles de tensión, velocidades, drivers y un sinnumero de etc diferentes, lo cuál dificultaba al usuario estar al día y poder utilizar cada componente que compraba. Lo más probable era encontrarque cada vez que uno comparaba una PC, debía cambiar el teclado, el mouse o algún periferico específico. Esto también complicaba a las mismas empresas productoras, por que la introducción de un nuevo sistema requería un mucho soporte extra para poder conectar todo lo ya existente.\\

Todo esto, quedó saldado con el standar USB, que debido a la gran cuota de mercado de sus desarrolladores, rápidamente fue introducido y se transformó en la norma a la hora de seleccionar un protocolo. Al punto tal esto se cumplió que hoy, más de 20 años después, es muy dificil encontrar PC's con otro tipo de puertos, salvo que en el momento de su compra uno requiera un puerto específicamente. Sin embargo, por norma, cualquier PC nueva dispónible en le mercado debe poseer puertos USB para la conexión de los periféricos.\\

Desde el punto de vista técnico, el protocolo USB es un sistema del tipo maestro-esclavo, donde el maestro, denominado HOST, debe ser necesariamente una PC y cualquier periférico a ella acoplada será un esclavo.\\

Para describirlo es conveniente tal vez separar el protocolo en tres partes. Una parte física, en donde se definen los componentes que intervienen, una capa de protocolo, en donde se define el formato y el marco en el que son enviados los paquetes, como se direccionana y como se comunican entre sí, y una parte lógica, en donde cada componente es visto solamente como un extremo y define como fluyen los datos desde un extremo hacia la PC y viceversa.\\

\subsection{Capa física}
	En esta sección no se describirán los detalles de las conexiones eléctricas ni mecánicas a las que se refieren las especificaciones de la norma USB debido a dos cuestiones fundamentales. Una de ellas es que toda esta sección de la norma está resuelta ya por los fabricantes del chip de Cypress. Este chip maneja todas las señales, arma y desarma los paquetes que salen hacia la PC y que llegan de ella respectivamente. Por otro lado, no es el objetivo de este trabajo adentrarse en esos detalles. Gracias a la extensión de este tipo de comunicación existen una gran cantidad de fabricantes en el mercado que fabrican cada uno de los componentes, ya sean, cables, conectores en todas sus versiones, adaptadores de un tipo de enchufe a otro, su costo es despreciable con respecto a cualquier tipo de desarrollo en ese sentido y son de una muy buena calidad, en el sentido que todos cumplen con lo que la norma establece.\\
	
	Si se hará a continuación una descripción de los dispositivos físicos y su categoría, degun la norma, en función del rol que cumplen.\\
	
	La comunicación USB posee una topología maestro-esclavo. Es decir, Existe un dispositivo que dirige todas las transferencias de datos y otros dispositivos que responden luego de que el maestro a emitido una directiva. Por esto, el elemento central de cualquier comunicación USB es el HOST (director o anfitrión, por su traducción de la voz inglesa). Este dispositivo que en la mayoria de los casos es la PC, aunquetambién puede ser algún dispositivo inteligente como un smartphone, es el que posee en Host USB Controller. El HOST se encarga de enviar los tokens a todos los periféricos, con la dirección del dispositivo, el sentido de la comunicación, el tipo de transferencia que se espera y todas las acciones de control que el sistema requiera.\\
	
	En el otro extremo de la comunicación, se encuentran los dispositivos. Los dispositivos son todos los periféricos que actúan como fuente o sumidero de información. Es decir, en caso de periféricos de entrada, serán una fuente de información hacia el Host. Si los periféricos son de salida, serán un sumidero de la información que proporciona la PC. Los casos de perifericos de entrada/salida, se denominarán periféricos compuestos.\\
	
	Existe también, a los fines de la norma,un elemento intermedio, denominado HUB (concentrador o distribuidor, según la traducción del inglés). Este dispositivo se encarga de conectar dos o más dispositivos, ya sea de entrada o salida, de recibir y enviar las direcciones y de regenerar las señales que el host envía y deben ser recibidas por los dispositivos, o bien, los datos que fluyen por el sistema.\\
	
\subsection{Capa de protocolo}
	En la capa de protocolo, las especificaciones detallan como se compone el marco y como los paquetes deben ser armados para que sean efectivamente enviados a través del sistema.\\
	
	Cada mensaje que se intercambia por el bus se denomina paquete. Cada paquete posee en su inicio lo que se denomina PID. el PID (del inglés packet ID o identificador del paquete) puede ser de 4 tipos, definidos por cada uno de los tipos de paquetes que puede haber: 
	\begin{itemize}
		\item en prime lugar se encuentran los paqutes token, a través del cual el host envía las directivas a los distintos periféricos. Estas directivas pueden ser IN, cuando solicita datos a un periferico; OUT, cuando va a enviar datos a un dispositivo; SOF, que indica el inicio de cada cuadro, para que cada dispositivo se sincronice y SETUP, cuando va a enviar un paquete de configuración a algún dispositivo.
		\item el segundo tipo de paquetes es el paquete de datos. Este tipo de PID puede ser emitido por un dispositivo, si es que envía datos al host o bien por el mismo host cuando el flujo de datos es a la inversa.
		\item el tercer tipo de paquets es un paquete de reconocimiento, denominado ACK, (por acknowledge o reconocimiento). Este paquete es enviado por los periféricos y le da idea al host de cual es el estado del dispositivo, es decir, si se encuentra operativo o no, si se encuientra ocupado o si recibió la trasnferencia de forma correcta.
		\item finalmente existen paquetes especiales, a través de los cuales el protocolo se comunica con los hubs, emite directivas intermedias, envia señales de polling para conocer el estado del bus, entre otras.
	\end{itemize}
	
	Como se menciono anteriormente, el host envía un toquen SOF que sirve para sincronizar los dispositivos al bus. En un sistema USB, el host provee la base de tiempo y envía cada \SI{1}{\milli\second} un SOF (Start of frame, o su traducción, inicio de cuadro) seguido de un numero de 11 bits que sirve para contar cada uno de los marcos. Además, en sistemas de alta velocidad, cada cuadro se divide en ocho microcuadros de \SI{125}{\micro\second}, que también son marcados por un SOF, sin embargo, el contador no se actualiza por cada microcuadro.\\
	
	Luego de esto, el sistema puede comenzar con la transferencia de datos. USB dispone 4 tipos de posibles transferencias, que se detallan un poco más adelante, y que pueden ser usadas conforme a los diferentes requerimientos del sistema.\\
	
	Cada transferencia de datos está compuesta por un primer paquete de token, emitido por el host, que posee el tipo de transferencia que se espera, sea de entrada, de salida, de control o especial; la dirección del dispositivo que debe responder o escuchar el mensaje enviado por el bus y un código de detección de errores del tipo CRC (código de checkeo redundante cíclico).\\
	
	Luego, el siguiente paquete posee los datos transferir, precedido por un PID de datos, y otro código CRC para detectar errores. Este paquete es transmitido por el host o el dispositivo dependiendo del sentido de la transferencia.\\
	
	Finalmente, el dispositivo devuelve un paquete de reconocimiento, indicandole al Host si el transferencia fue efectiva o no y por qué esta no fu efectiva, siendo ese el caso.\\
	
	\subsubsection*{Transferencias por paquetes (Bulk transfers)}
		Este tipo de transferencias puede ser dispuesta para trasmitir un gran flujo de datos. No posee perdida de datos gracias a un sistema de chequeo y retransmisión de datos. El inconveniente que presenta este tipo de transferencias es que en un nivel de prioridades se presenta en el final del sistema. Es decir, el bus solo va a ser usado para transferir este tipo de datos siempre que se encuentre desocupado, o bien, se le asignará una proporción ínfima de ancho de banda para poder trasnmitir con este modo. Es comunmente usado para trasmitir datos que no son críticos en tiempo, por ejemplo para scanners e impresoras.\\
	
	\subsubsection*{Transferencias de interrupción}
		Este tipo de trasnferencias sirve para enviar y recibir paquetes de datos que requieren un buen sistema de control de errores, pero que, son más restrictivos en tiempos. El sistema siempre destinará un intervalo fijo de tiempo para transmitir los datos que estén pendientes para trasnferencias de interrupción.\\
	
	\subsubsection*{Transferencias Isocrónicas}
		Este tipo de trasnferencias está destinado a datos que son realmente críticos en tiempo. Es usado, principalmente para enviar datos a chorro, como ser el caso de streaming de audio o video, en donde los datos producidos deben ser rápidamente llevados al usuario.\\
	
		No posee un control de errores muy sofisticado, más que un simple código CRC, pero no existe mecanismo de retrasmisión de datos ni handshaking entre los dispositivos y el host.\\
	
		Como el tiempo es el requerimiento críticoen este tipoo de datos, el controlador le asigna una determinada cantidad de tiempo de bus, o en otras palabras, una determinada cuota de ancho de banda.\\
	
	\subsubsection*{Transferencias de control}
		Este tipo de trasnferencias solo las emite el host y el sistema las utiliza para configurar cada dispositivo. Debido a su criticidad, el controlador dispondra encada cuadro de una fracción de ancho de banda para las trasnmisiones de control. Es el tipo de trasnferencias que posee el sistema de detección de errores más sofisticado, de forma tal de asegurar la integridad de los datos de control.\\
	
		A cambio de esto, solo muy poca información puede ser trasmitida por cada cuadro, de hasta 64 bytes en sistemas de alta velocidad.\\
	
\subsection{Capa lógica}
	Desde el punto de vista lógico, cada dispositivo es visto por el host como un extremo independiente, que posee un módo de comunicación, es decir, con ese dispositivo el protocolo se comunicara solo por un tipo de trasnferencia;y un solo sentido. En otras palabras, USB notará como separado un dipositivo de entrada y otro de salida, independientemente de si físicamente el dispositivo es un perifericode entrada y salida.\\
	
	Esta independencia brinda la posibilidad de configurar cada extremo de forma diferente y obtener el ancho de banda necesario para la subida y bajada de datos, los tiempos de acceso al bus, la dirección y todo lo relacionado a los modos de comunicación conforme a los requerimientos.\\
	
	El protocolo entiende que entre le host y cada uno de los extremos existe un tubo (la norma en ingles habla de {\it pipes}) en donde la información es colocada y transferida. Luego, cada tubo posee la configuración establecida por el controlador del host y se comunica con cada extremo por medio de estos tubos. A los fines del usuario, esto es lo importa, por cuanto uno solicita acceso al bus y define en que buffer va a contener los datos a enviar o transmitir y el protocolo se encarga de el empaquetado, el armado de los cuadros, el acceso el bus y el posterior envío de datos.\\
