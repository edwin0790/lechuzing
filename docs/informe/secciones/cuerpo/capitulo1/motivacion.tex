%\subsubsection*{Las herramientas que adquieren datos}
El grado de avance que ha experimentado la tecnología en general, y la electrónica en particular, gracias a la industria de los semiconductores, permite que la producción científica pueda adquirir una gran cantidad de datos. Esto, requiere el relevamiento y registro de diferentes tipos de magnitudes físicas y/o químicas sobre el objeto o proceso sobre el que se está investigando. Sin embargo, esta no es una tarea simple ya que, en muchos casos, estas magnitudes son difíciles de observar y cuantificar con los sentidos de un operador. Entonces, se vuelve conveniente transformar las variables a medir en otras que resulten más simples.\\

El dispositivo que recibe estímulos energéticos de una condición, situación o fenómeno físico y/o químico y los convierte en una señal asociada y definida de otra forma de energía, se lo denomina transductor\cite{Pallas-Areny2001}\cite{considine1971encyclopedia}. En otras palabras, este tipo de máquinas son conversores de energías\cite{considine1971encyclopedia}\cite{Pallas-Areny2001}\cite{PerezGarcia2008}. Un sensor es un tipo especial de transductores que posee como variable de salida una señal eléctrica que está adaptada para ser ingresada en un circuito electrónico, o adecuada al sistema de medida que se utilice \cite{Fraden2010}\cite{Slawinski2011}\cite{Ogata2002}.\\

Las altas escalas de integración de circuitos alcanzadas (y aún en incremento) posibilitan el diseño de sistemas  sensoriales cada vez más complejos, en los cuales se logra agrupar miles de sensores en áreas reducidas, obteniendo medidas simultáneas y flujos crecientes de datos. Uno de los desarrollos que se encuentra en boga es el de sensores y sistemas que adquieran imágenes de diferentes tipos. Desde el punto de vista digital, una imagen es un arreglo bidimensional de números, los cuales pueden ser exhibidos en una pantalla en forma de intensidad y colores de luz. Por esto, un sensor de imagen puede estar compuesto, bien por un arreglo bidimensional de sensores lumínicos, por un transductor que es simultáneamente desplazado y medido o por una combinación de ambos métodos. En todos los casos, es de suma utilidad que la lectura de imágenes sea realizada en el menor tiempo posible ya que cada imagen conlleva una cantidad no menor de datos.\\

Desde un punto de vista electrónico, para poder mover grandes volúmenes de datos en forma digital, se requiere de circuitos que sean capaces de operar a alta frecuencia de conmutaciones. Esto no es trivial, ya que cuando las longitudes de las ondas que se mueven a través de un circuito son comparables con sus dimensiones físicas, los conductores se comportan como líneas de transmisión, se evidencian capacidades parásitas en diseño que perjudican el desempeño de los sistemas. Por ejemplo, operando a \SI{50}{\mega\hertz}, hablamos de \\

En gran medida, la incorporación y evolución de microcontroladores permite obtener una captura y obtención cada vez superior de datos. Sin embargo, al poseer una estructura rígida y su capacidad de procesamiento está limitada a una instrucción por vez, no es del todo adecuada para el desarrollo de nuevos dispositivos y técnicas de medida especializadas. En ciertos casos, la posibilidad de realizar cientos de operaciones es excesivo para aplicaciones muy específicas. Una solución óptima, sin considerar los costos asociados a esto, sería el desarrollo de un circuito integrado de aplicación específica (ASIC) que resuelva en forma precisa el problema. Sin embargo, cuando sí se considera el costo asociado a este enfoque, se vuelve una solución totalmente ineficiente, del orden de las decenas de miles de dolares.\\

El siguiente enfoque, óptimo al criterio de este trabajo, es la utilización de FPGAs para realizar el desarrollo. Al poseer, este tipo de herramientas, la posibilidad de conectar, adquirir y procesar en paralelo y a alta velocidad, incorpora una gran ventaja con respecto a un procesador. A su vez, al ser programable, se adapta mejor a una solución específica. En cuanto al costo de un FPGA, van desde decenas a centenas de dolar, es decir, dos ordenes de magnitud más económico que un ASIC.\\

Existen diversas publicaciones en donde se observa el uso de FPGAs para la implementación de sistemas que producen imágenes. Por ejemplo, el desarrollo de un sensor de radiación utilizando una cámara CMOS comercial. Para ello, los autores utilizaron un FPGA para configurar y transmitir imagenes a un grabador de video a través de puerto UART. Esto permitió adquirir una imagen accionando un disparador realizado con un pulsador\cite{Perez2017}.\\

Se denomina ultrasonografía a la técnica de adquirir imágenes basandose en rebotes de ultrasonido. Sus aplicaciones son múltiple, en las que se destaca el diagnóstico médico, ya que es una técnica no invasiva y sin riesgos de radiación ionizante sobre el paciente. Un trabajo reciente desarrolló un sistema de ecografía médica con bajo costo utilizando un FPGA\cite{biswas2018embedded}. El autor también presentó un algoritmo realizado y probado en PC. Luego se implementó e en una FPGA.\\

Yanagisawa, entre otros, desarrolló un sistema con telescopios pequeños para explorar objetos de campo cercano con la finalidad de monitorear cuerpos celestes que puedan colisionar con el planeta\cite{Yanagisawa2018}. Los autores utilizan la velocidad de los circuitos implementados en FPGA para minimizar el tiempo de adquisición.

%EXTRAIDA Existen sensores de imagenes de distancia desarrollados mediante FPGA\cite{Cui2018}.\\%BORRAR esta cita!!!

\subsubsection*{El procesamiento de datos}
La obtención de datos por si misma no otorga información. Para ello, es probable que un gran flujo de datos requiera de un procesamiento y análisis exhaustivo de los mismos. La invención y evolución de las computadoras, como así también el desarrollo de nuevos algoritmos, dan lugar a procesamiento de datos cada vez más complejos en tiempos mucho menores.\\

Las primeras ENIAC, computadora de propósito general desarrollada en el año 1946 para el cálculo de tablas balísticas de las fuerzas armadas estadounidenses, podía ejecutar 20 operación cada \SI{10}{\micro\second}, es decir, podía ejecutar instrucciones con una frecuencia máxima de \SI{200}{\kilo\hertz}. A su vez, tuvo un costo aproximado de U\$S 500.000, pesaba 5 t y consumía \SI{175}{\kilo\watt}.\\

En contraste con aquello, es posible conseguir en el mercado actual, computadoras que pesan unidades de kilogramos, pueden ejecutar instrucciones en unidades de nanosegundo, (5 ordenes de magnitud menos), consumen apenas centenas de watts y cuestan algunos cientos de U\$S. A tal punto ha evolucionado esta tecnología que se encuentran presente computadoras muy potentes en casi cualquier laboratorio, oficina u hogar.\\

Esta potencia de cálculo, ayudado por el desarrollo de nuevos métodos y algoritmos de cálculo, permiten a los investigadores procesar miles de datos en tiempos muy reducidos, ayudando al análisis de los mismos y la obtención de información.\\

\subsubsection*{La comunicación entre los sistemas productores y los procesadores de datos}
La generación de datos y el procesamiento de lo mismos, según el enfoque del presente trabajo, se da en sistemas diferentes. Estos sistemas requieren, de una conexión a través de la cual los datos puedan ser llevados de un lugar a otro. Se torna de suma utilidad, entonces, proveer una comunicación efectiva y robusta que permita mover grandes volúmenes de datos en poco tiempo, y de esta forma facilitar los tiempos de desarrollo, las pruebas y depuración de sistemas.\\

Implementar un sistema de comunicación en una FPGA, si bien no es trivial, puede ser resuelto de muchas maneras, quedando a criterio del desarrollador utilizar algún sistema de comunicación estándar, o bien, diseñar uno propio. Sin embargo, desde el punto de vista de una computadora, las comunicaciones se vuelven un poco más estrictas y acotadas a los puertos y señales que puede manejar el equipo, conforme el fabricante haya establecido.\\

Es por eso que este trabajo trata de implementar una comunicación entre una computadora personal y una FPGA, utilizando un protocolo estándar, que esté disponible en cualquier computadora comercial y que posea una tasa de bit suficiente para poder transmitir imagenes.\\



%Es por esto que la PC {\it Personal Computer} se ha transformado en la herramienta indispensable en cualquier ámbito, pero en especial en los entornos en donde se requiere el  manejo, cálculo, procesamiento y análisis de grandes cantidades de datos e información.\\
%
%
%
%Desde la inclusión de la norma USB, en el año 1996 a la fecha, se ha convertido en el elemento que no falta en ningun equipo, al punto tal que ha desplazado a cualquier otro conector. Al punto tal es esto, que para requerir algún puerto adicional que no sea de esta norma, cualquier comprador debe especificar que así sea, mas no es necesario especificar que tiene USB como norma de conexión.\\
%
%El presente trabajo pretende ofrecer una comunicación basada en la versión 2.0 de la norma USB entre una PC y un FPGA, y otorgue una herramienta adicional a científicos que desarrollen o utilicen sistemas basados en FPGA.\\
%
%Este trabajo, pretende elaborar una interfaz entre los dos extremos, es decir, entre la PFGA y la PC, de forma tal que permita a un desarrollador, investigador o usuario en general, obtener una comunicación confiable y con un ancho de banda que permita mover el flujo de datos que genera una sensor que adquiera imágenes.\\
%
%Es cierto que el protocolo puede ser totalmente implementado en una FPGA, sin embargo, esto requeriría un muy alto costo tanto económico como en recursos disponibles del chip programable para una tarea genérica que es mejor elaborar con un circuito integrado diseñado especialmente para tal fin. Es por esto que se utiliza como lazo de interfaz un chip comercial elaborado por Cypress Semiconductor.\\
%
%
%
%[1][https://ieeexplore.ieee.org/abstract/document/8214376]
%[2][http://www.idr.iitkgp.ac.in/jspui/bitstream/123456789/9068/1/NB15975_Abstract.pdf]
%[3][https://ieeexplore.ieee.org/abstract/document/8396725]




%El mundo actual, en el que vivimos inmersos, demanda y consume volumenes cada vez más grandes de información. Con solo hacer una rapida miradad en diarios, incluso no especilizados, se observa la importancia que poseen las ciencias y disciplinas que manejas la informacion, aquellas areas agrupadas dentro del conjunto Técnico de la Información y la Comunicación, o más ocnocido por sus siglas, TIC's.
%
%Internet de las cosas , Big Data, Inteligencia Artificial, Redes Neuronales, Robótica, Domótica, entre otras, son areas en las que los datos y la información es abultada y su correcto manejo es sumamente complejo e importante. Además, ninguna de las actividades científicas, puede escapar de esta gran demanda mundial de información.
%
%La información es un conjunto de datos ordenados y procesados de forma tal que permita al que lo lea que eleve su nivel de conocimiento sobre un determinado tema. Es decir que, para que exista información, en primer lugar tenemos que tener datos, de forma tal que podamos luego procesarlos y obtener realmente información de llos.
%
%Como ejemplo de esto, podemos citar simplemente el acto de medir la presión dentro de un tubo de gas. Sería normal pensar en que, teniendo este objetivo en mente, simplemente coloquemos un manómetro en la salida del tubo. El manómetro es un artefacto que posee una varilla 
%
%
%
%
%
%El mundo, durante la era de la información, nos exige producir y consumir cada vez más y más información. En cada aspecto social dela vida de la persona se pueden encontrar ejemplos claros de esto.\\
%
%Bajo un punto de vista social nos vemos bombardeados por información. Los 'influencers' que se mueven a traves de Facebook, Twiter, Instagram, Snapchat, Linkedin, Youtube, etc, necesitan para estar a la moda, producir constantemente material y que ese material sea consumido por alguien que este demandando esa información.
%
%En el area de econimía y finanzas existen cada vez más robots que extraen información de las diferentes bolsas, periodicos, bancos, industrias y donde se ocurra que pueda haber información útil, para tomar mejores decisiones que, por supuesto, también ejecutan los robots.
%
%En el comercio, desde las tiendas de supermercado hasta las tiendas digitales que trabajan solo a través de internet están constantemente recabando datos y procesandolos a din de diseñar estrategias que permitan ofrecer productos que se adapten mejor a lo que busca el cliente y de esa forma lograr vender mayores volumenes.
%
%La industria cuenta cada vez más con multiplicidad de posibilidades basadas en sensores que adquieren grandes cantidades de datos que son almacenadas y procesadas, muchas vecen 'en línea' o 'al instante', de forma tal de encontrar mejores procesos o ejecutar nuevas tareas, como por ejemplo la industria automotriz, enfocada cada vez más en autos que puedan manipular todo el entorno y ser totalmente autónomos de un conductor.
%La industria espacial y satelital dotan a sus equipos de mayores sensores y mayores flujos de datos. La industria médica brinda cada vez más posibilidades de diagnóstico a traves de nuevas técnicas y formas de adquisición de imágenes.
%
%A todo esto, la ciencia y el desarrollo de nuevas aplicaciones y equipos, no son indiferente. Cada vez se encuentran más innovaciones y desarrollos de nuevos sensores que adquieren flujos de información crecientes. La toma de imagenes se ha vuelto una herramienta clave para la investigación en diversas areas, tales como la biología, la ciencia de materiales, las ciencias nucleares, las ciencias de la tierra, etc.
%
%Las redes sociales son un ejemplo de esto, pero esto se observa en la industria, la economía, las finanzas e incluso hasta en la casa.
%
%
%
%[1] file:///home/lechuzin/Facultad/Trabajo%20Final/lechuzing/docs/bibliografia/The-Rise-of-the-Network-Society-With-a-New-Preface-Volume-I-The-Information-Age-Economy-Society-and-Culture-Information-Age-Series-.pdf