%\subsubsection*{Las herramientas que adquieren datos}
El grado de avance que ha experimentado la tecnología en general, y la electrónica en particular, gracias a la industria de los semiconductores, permite que la producción científica pueda adquirir una gran cantidad de datos. Para llevar a cabo la producción del conocimiento, se requiere el relevamiento y registro de diferentes tipos de magnitudes físicas y/o químicas sobre el objeto o proceso sobre el que se está investigando. Sin embargo, esta no es una tarea simple ya que, en muchos casos, estas magnitudes son difíciles de observar y cuantificar. Entonces, se vuelve conveniente transformar las variables a medir en otras que resulten más simples.\\

El dispositivo que recibe estímulos energéticos de una condición, situación o fenómeno físico y/o químico y los convierte en una señal asociada y definida de otra forma de energía, se lo denomina transductor\cite{Pallas-Areny2001}\cite{considine1971encyclopedia}. En otras palabras, este tipo de máquinas son conversores de energías\cite{considine1971encyclopedia}\cite{Pallas-Areny2001}\cite{PerezGarcia2008}. Un sensor es un tipo especial de transductores que posee como variable de salida una señal eléctrica que está adaptada para ser ingresada en un circuito electrónico, o adecuada al sistema de medida que se utilice \cite{Fraden2010}\cite{Slawinski2011}\cite{Ogata2002}.\\

Las altas escalas de integración de circuitos alcanzadas en la actualidad posibilitan el diseño de sistemas sensoriales cada vez más complejos, en los cuales se logra agrupar miles de sensores en áreas reducidas, obteniendo medidas simultáneas y flujos crecientes de datos. Este trabajo se centrará en la transmisión de datos provenientes de sensores de imágenes, uno de los desarrollos que se encuentra en boga.\\

Desde el punto de vista digital, una imagen es un arreglo bidimensional de números, los cuales pueden ser exhibidos en una pantalla en forma de intensidad y colores de luz. Cada punto del arreglo que se muestra en pantalla se denomina pixel, acrónimo del ingles {\it PIcture ELement}, o elemento de imágen. Por esto, un sensor de imagen puede estar compuesto, bien por un arreglo bidimensional de sensores lumínicos, cómo la cámara de un teléfono celular, por un transductor que es simultáneamente desplazado y medido, método utilizado para la microscopía de fuerza atómica \cite{Binnig1983}, o por una combinación de ambos métodos. Por ejemplo, un scanner posee un arreglo lineal de transductores que son desplazados a través de la hoja para generar una imagen digital. En cualquiera de los casos, es de suma utilidad que la lectura de imágenes sea realizada en el menor tiempo posible, ya que cada imagen conlleva una cantidad no menor de datos.\\

\begin{figure}[t]
	\centering
	\includegraphics[width=0.7\textwidth]{pixel-aps1}
	\caption{Esquema de un pixel activo}
	\label{fig:pix}
\end{figure}

%TODO insertar aquí ejemplos de desarrollo de sensores de imágenes
Uno de los trabajos más importantes en este sentido, fue el desarrollo de los CMOS APS ({\it Active Pixel Array}, o matriz de pixeles activos) \cite{Mendis1994}. La Figura \ref{fig:pix} muestra un dibujo esquemático del funcionamiento de cada uno de los pixeles activos que componen la matriz. Este posee un fotodiodo, que es el elemento transductor. En un primer momento, este diodo es conectado en forma inversa, generando una región de vaciamiento entre sus terminales. Luego, cuando un fotón incide sobre él, se forman pares electrón-hueco y ellos circulan a través del diodo a las terminales, generando una caída de tensión. Finalmente, se mide la tensión del diodo, cuya intensidad será inversamente proporcional a la cantidad de fotones que incidieron sobre el pixel en cuestión.\\

A partir del desarrollo de los APS, se fue perfeccionando el método hasta obtener circuitos integrados con mayor cantidad de pixeles y que pueden tener diversas aplicaciones. Por ejemplo, en los trabajos \cite{Hu-Guo2009} y \cite{Baudot2009} se presentan sensores CMOS basados en la arquitectura MIMOSA (de {\it Minimum Ionizing particule MOS Active pixel Sensor}, Sensor con Pixel activo MOS de particulas ionizantes mínimas). Estos sensores se desarrollaron con el objetivo específico de detección de radiación ionizante. Esta última temática también la tomaron Galimberti \cite{Galimberti2018}, Perez {\it et al.}\cite{Perez2016}, utilizando sensores comerciales y en \cite{Perez2018Thermal} los usan para detectar neutrones térmicos En otro trabajo, Hizawa, {\it el al.} \cite{Hizawa2007} fabricaron un sensor que adquiere imágenes midiendo el pH de cada uno de los pixeles, obteniendo imágenes de fenómenos químicos en tiempo real.\\

La obtención de datos por si misma no otorga información. Para ello, es probable que un gran flujo de datos requiera de un procesamiento y análisis exhaustivo de los mismos. La invención y evolución de las computadoras, como así también el desarrollo de nuevos algoritmos, dan lugar a procesamiento de datos cada vez más complejos en tiempos mucho menores.\\

Las primeras ENIAC, computadora de propósito general desarrollada en el año 1946 para el cálculo de tablas balísticas de las fuerzas armadas estadounidenses, podía ejecutar 20 operaciones cada \SI{10}{\micro\second} \cite{Goldstine1946}, es decir, ejecutaba instrucciones con una frecuencia máxima de \SI{200}{\kilo\hertz}. A su vez, tuvo un costo aproximado de U\$S 500.000, pesaba 5 t y consumía \SI{175}{\kilo\watt}.\\

En contraste con aquello, es posible conseguir en el mercado actual, computadoras cuyas dimensiones y peso hacen que puedan ser levantadas con las manos, ejecutan instrucciones en cuenstión de nanosegundos, (5 ordenes de magnitud menos), consumen menos de \SI{1}{\kilo\watt} y cuestan algunos cientos de U\$S. A tal punto ha evolucionado esta tecnología que se encuentran presente computadoras muy potentes en casi cualquier laboratorio, oficina u hogar. Esta potencia de cálculo, ayudado por el desarrollo de nuevos métodos y algoritmos de cálculo, permiten a los investigadores procesar miles de datos en tiempos muy reducidos, ayudando al análisis de los mismos y la obtención de información.\\

La generación de datos y el procesamiento de lo mismos se da en sistemas diferentes. Ellos requieren, por tanto, de una conexión a través de la cual los datos puedan ser transferidos de un sistema al otro. Se torna de suma utilidad, entonces, proveer una comunicación efectiva y robusta que permita transmitir grandes volúmenes de datos en poco tiempo, y de esta forma facilitar los tiempos de desarrollo, las pruebas y depuración.\\

Desde un punto de vista electrónico, para poder transmitir grandes volúmenes de datos en forma digital, se requiere de circuitos que sean capaces de operar a altas frecuencias de conmutación. Esto no es trivial, ya que cuando las longitudes de onda que se mueven a través de un circuito son comparables con sus dimensiones físicas, se evidencian capacidades e inductancias parásitas no contempladas que perjudican el desempeño de los sistemas. Por ejemplo, si se desea transmitir datos en serie con una variación entre sus datos de \SI{300} {\mega\hertz}, se habla de ondas cuya longitud de onda es de \SI{1}{\meter}. Se dice que las dimensiones son comparables cuando el largo del conductor posee al menos un cuarto de la longitud de onda, es decir, para este caso, \SI{25}{\centi\meter}. Si bien esto puede parecer grande, se debe notar que a mayores frecuencias, este efecto comienza a ser aún más perjudicial.\\

Otro problema que presentan los circuitos de alta velocidad tiene que ver con los tiempos de propagación de las corrientes y tensiones que circulan a través de ellos. Cuando se aplica un impulso en un conductor, la onda viaja a la velocidad de la luz. Esto quiere decir que la tensión no llega al mismo tiempo a todos los puntos del circuito, sino que, mientras más alejada está la fuente, más lejos está se demora en responder. Puede suceder, entonces, que la lógica del circuito se demore más que el pulso de reloj que indica un cambio de estado, obteniendo así un comportamiento no deseado, si no está correctamente diseñado y contemplado este aspecto.\\

Una solución para los circuitos electrónicos que necesitan mover grandes volúmenes de datos, puede ser la incorporación varios conductores a través de los cuales puede fluir la información. Retornando al ejemplo de los datos transmitidos con variaciones de hasta \SI{300}{\mega\hertz}, si son enviados a través de dos conductores iguales, idénticos al anterior, la frecuencia necesaria cae sería la mitad, con 3, se obtiene una reducción de la frecuencia a la tercera parte, con 10, es suficiente con un décimo, etc. La cantidad de conductores a través de los cuales circula la información, se denomina ancho de bus.\\

En gran medida, la incorporación y evolución de microcontroladores permite capturar y procesar volúmenes crecientes de datos. Sin embargo, este tipo de dispositivos posee una estructura rígida, capacidad de procesamiento limitada a una instrucción por vez y ancho de bus definido, la única opción para aumentar los volúmenes de datos que circulan a través de ellos, es un aumento de la frecuencia de funcionamiento, generando los problemas anteriormente detallados. Una solución óptima, sin considerar los costos asociados a esto, sería el desarrollo de un circuito integrado de aplicación específica (ASIC del inglés {\it Application Specific Integrated Circuit}). En este tipo de circuitos, el diseñador elabora un circuito que puede operar a altas velocidades y, a su vez, obtener un ancho de bus sin restricciones, más que las dimensiones físicas del área donde será realizado el circuito. Sin embargo, cuando sí se considera el costo asociado a este enfoque, se vuelve una solución ineficiente en bajas cantidades. La manufactura de este tipo de dispositivos puede tener un costo de miles hasta cientos de miles de dólares, dependiendo del proceso de fabricación utilizado. Gran parte de estos costos son no recurrentes, es decir, solo se pagan una vez por proyecto. En grandes cantidades de dispositivos, este tipo de soluciones se vuelven más convenientes.\\

Otro enfoque, es la utilización de Arreglos de Compuertas Programables por Campo (FPGA, acrónimo del inglés {\it Field-Programmable Gate Array}). Un FPGA es un dispositivo electrónico que posee la capacidad de sintetizar casi cualquier circuito digital. En esencia, es una matriz de bloques lógicos (también llamadas {\it slices} o celdas lógicas, dependiendo del fabricante), que contienen Tablas de Verdad(LUTs o {\it Look-Up-Table}) y flip-flops (ff), entre otras cosas, y pueden ser interconectadas entre sí, según el criterio del usuario. Así, permite implementar una solución digital en un circuito físico, a diferencia del microcontrolador que lo realiza a través de un algoritmo, incorporando la ventaja de definir el ancho de bus necesario para relevar una gran cantidad de datos y transmitirlos a frecuencias de trabajo menores, además de ejecutar tareas en paralelo, disminuyendo los tiempos de procesamiento. A su vez, al ser implementado en un área muy pequeña, debido a la integración del sistema, este tipo de sistemas puede trabajar a frecuencias muy elevadas, lo que implica una mayor tasa de datos aún. A pesar de la gran diversidad de precios existentes en el mercado, una FPGA de costos menores a la centena de dólares suele tener muy buenas prestaciones para la mayor parte de las aplicaciones.\\

Existen diversas publicaciones en donde se observa el uso de FPGAs para la implementación de sistemas que producen imágenes. Por ejemplo, el desarrollo de un detector de radiación ionizante utilizando una sensor CMOS comercial. Para ello, los autores utilizaron una FPGA para configurar diversos parámetros del sensor y transmitir imágenes a una PC a través de un puerto UART. Esto permitió adquirir una imagen accionando un disparador realizado con un pulsador\cite{Perez2017}.\\

Se denomina ultrasonografía a la técnica de adquirir imágenes basandose en reflexiones de ultrasonido. Sus aplicaciones son múltiples, en las que se destaca el diagnóstico médico debido. Un trabajo reciente desarrolló un sistema de ecografía médica con bajo costo utilizando una FPGA\cite{biswas2018embedded}. El autor también presentó un algoritmo realizado y probado en PC. Luego se implementó e en una FPGA.\\

Yanagisawa {\it et al}, desarrollaron un sistema con telescopios pequeños para explorar objetos de campo cercano con la finalidad de monitorear cuerpos celestes que puedan colisionar con el planeta\cite{Yanagisawa2018}. En este trabajo, se aprovechó la velocidad de los circuitos implementados en FPGA para minimizar el tiempo de adquisición.\\

Implementar un sistema de comunicación en una FPGA, si bien no es trivial, puede ser resuelto de muchas maneras, quedando a criterio del desarrollador utilizar algún protocolo de comunicación estándar, o bien, diseñar uno propio. Sin embargo, desde el punto de vista de una computadora, las comunicaciones se vuelven un poco más estrictas y acotadas a los puertos y señales que puede manejar el equipo, conforme el fabricante haya establecido.\\

Es por eso que este trabajo busca implementar una comunicación entre una computadora personal y una FPGA, utilizando un protocolo estándar, que esté disponible en cualquier computadora comercial y que posea una tasa de bit suficiente para poder transmitir imagenes.\\


%Es por esto que la PC {\it Personal Computer} se ha transformado en la herramienta indispensable en cualquier ámbito, pero en especial en los entornos en donde se requiere el  manejo, cálculo, procesamiento y análisis de grandes cantidades de datos e información.\\
%
%
%
%Desde la inclusión de la norma USB, en el año 1996 a la fecha, se ha convertido en el elemento que no falta en ningun equipo, al punto tal que ha desplazado a cualquier otro conector. Al punto tal es esto, que para requerir algún puerto adicional que no sea de esta norma, cualquier comprador debe especificar que así sea, mas no es necesario especificar que tiene USB como norma de conexión.\\
%
%El presente trabajo pretende ofrecer una comunicación basada en la versión 2.0 de la norma USB entre una PC y un FPGA, y otorgue una herramienta adicional a científicos que desarrollen o utilicen sistemas basados en FPGA.\\
%
%Este trabajo, pretende elaborar una interfaz entre los dos extremos, es decir, entre la PFGA y la PC, de forma tal que permita a un desarrollador, investigador o usuario en general, obtener una comunicación confiable y con un ancho de banda que permita mover el flujo de datos que genera una sensor que adquiera imágenes.\\
%
%Es cierto que el protocolo puede ser totalmente implementado en una FPGA, sin embargo, esto requeriría un muy alto costo tanto económico como en recursos disponibles del chip programable para una tarea genérica que es mejor elaborar con un circuito integrado diseñado especialmente para tal fin. Es por esto que se utiliza como lazo de interfaz un chip comercial elaborado por Cypress Semiconductor.\\
%
%
%
%[1][https://ieeexplore.ieee.org/abstract/document/8214376]
%[2][http://www.idr.iitkgp.ac.in/jspui/bitstream/123456789/9068/1/NB15975_Abstract.pdf]
%[3][https://ieeexplore.ieee.org/abstract/document/8396725]




%El mundo actual, en el que vivimos inmersos, demanda y consume volumenes cada vez más grandes de información. Con solo hacer una rapida miradad en diarios, incluso no especilizados, se observa la importancia que poseen las ciencias y disciplinas que manejas la informacion, aquellas areas agrupadas dentro del conjunto Técnico de la Información y la Comunicación, o más ocnocido por sus siglas, TIC's.
%
%Internet de las cosas , Big Data, Inteligencia Artificial, Redes Neuronales, Robótica, Domótica, entre otras, son areas en las que los datos y la información es abultada y su correcto manejo es sumamente complejo e importante. Además, ninguna de las actividades científicas, puede escapar de esta gran demanda mundial de información.
%
%La información es un conjunto de datos ordenados y procesados de forma tal que permita al que lo lea que eleve su nivel de conocimiento sobre un determinado tema. Es decir que, para que exista información, en primer lugar tenemos que tener datos, de forma tal que podamos luego procesarlos y obtener realmente información de llos.
%
%Como ejemplo de esto, podemos citar simplemente el acto de medir la presión dentro de un tubo de gas. Sería normal pensar en que, teniendo este objetivo en mente, simplemente coloquemos un manómetro en la salida del tubo. El manómetro es un artefacto que posee una varilla 
%
%
%
%
%
%El mundo, durante la era de la información, nos exige producir y consumir cada vez más y más información. En cada aspecto social dela vida de la persona se pueden encontrar ejemplos claros de esto.\\
%
%Bajo un punto de vista social nos vemos bombardeados por información. Los 'influencers' que se mueven a traves de Facebook, Twiter, Instagram, Snapchat, Linkedin, Youtube, etc, necesitan para estar a la moda, producir constantemente material y que ese material sea consumido por alguien que este demandando esa información.
%
%En el area de econimía y finanzas existen cada vez más robots que extraen información de las diferentes bolsas, periodicos, bancos, industrias y donde se ocurra que pueda haber información útil, para tomar mejores decisiones que, por supuesto, también ejecutan los robots.
%
%En el comercio, desde las tiendas de supermercado hasta las tiendas digitales que trabajan solo a través de internet están constantemente recabando datos y procesandolos a din de diseñar estrategias que permitan ofrecer productos que se adapten mejor a lo que busca el cliente y de esa forma lograr vender mayores volumenes.
%
%La industria cuenta cada vez más con multiplicidad de posibilidades basadas en sensores que adquieren grandes cantidades de datos que son almacenadas y procesadas, muchas vecen 'en línea' o 'al instante', de forma tal de encontrar mejores procesos o ejecutar nuevas tareas, como por ejemplo la industria automotriz, enfocada cada vez más en autos que puedan manipular todo el entorno y ser totalmente autónomos de un conductor.
%La industria espacial y satelital dotan a sus equipos de mayores sensores y mayores flujos de datos. La industria médica brinda cada vez más posibilidades de diagnóstico a traves de nuevas técnicas y formas de adquisición de imágenes.
%
%A todo esto, la ciencia y el desarrollo de nuevas aplicaciones y equipos, no son indiferente. Cada vez se encuentran más innovaciones y desarrollos de nuevos sensores que adquieren flujos de información crecientes. La toma de imagenes se ha vuelto una herramienta clave para la investigación en diversas areas, tales como la biología, la ciencia de materiales, las ciencias nucleares, las ciencias de la tierra, etc.
%
%Las redes sociales son un ejemplo de esto, pero esto se observa en la industria, la economía, las finanzas e incluso hasta en la casa.
%
%
%
%[1] file:///home/lechuzin/Facultad/Trabajo%20Final/lechuzing/docs/bibliografia/The-Rise-of-the-Network-Society-With-a-New-Preface-Volume-I-The-Information-Age-Economy-Society-and-Culture-Information-Age-Series-.pdf