La información es el resultado de recopilar, ordenar y procesar un conjunto de datos, de forma tal que permitan adquirir mayor conocimiento sobre un asunto determinado y otorguen un significado mayor que cada uno de los datos por separado. Por ejemplo, Si un caprintero quiere medir la distancia de una barra de madera, utilizará una cinta métrica. Se denomina cinta métrica a una lámina metálica que posee marcas graduadas. Esta graduación, se realiza comparando la lamina metal con una barra patrón que indica que la distancia indicada se corresponde con la convención de distancia.\\

El carpintero compara el tamaño de las patas de la mesa con la cinta métrica, que a su vez, posee registrada su distancia en función del patrón. Esto quiere decir que el dato 1, la longitud del patron, junto al dato 2, escala graduada de la cinta, más el dato 3, la longitud de la cinta métrica, permiten al carpintero cambiar su estado de desconocido a conocido, con respecto a la longitud del trozo de madera, a través de la información proporcionada por el conjunto de datos.\\

La Ciencia es un conjunto de técnicas y procedimientos que, a través de un método científico, busca adquirir, descubrir y/o desarrollar nuevo conocimiento. Se puede deducir, entonces, que la ciencia produce, de forma fundamental, datos, que luego de un procesado se transforman en información. A su vez, el estudio detallado de esta información genera conocimiento.\\

Cuando se nombra la palabra ciencia, se hace referencia a un conjunto conformado por diferentes objetos de estudio. El objeto de estudio es el sujeto a través del cuál se da la principal clasificación de las diferente ramas de la Ciencia: las Ciencias Sociales estudian las relaciones humanas y las Ciencias Naturales dedican sus esfuerzos a objetos que se encuentran en la naturaleza. Dentro del último grupo encontramos las Ciencias de la Tierra se enfocan en una rama más particular de la naturaleza, como lo es el estudio de la superficie terrestre; y siguiendo así se puede encontrar muchas más clasificaciones de ciencias, incluso como ramas englobadas por las anteriormente nombradas.\\

Sin embargo, la producción científica necesita adquirir una gran cantidad de datos que luego serán ordenados, procesados, analizados y transformados en información y conocimiento. En este sentido, la incorporación de una herramienta especialmente diseñada para el procesamiento de datos, como lo es la computadora, permite manejar un numero creciente de información. Es por eso que se encuentra en desarrollo un gran número de sensores y dispositivos que permitan obtener flujos crecientes de datos.\\

Uno de los desarrollo que se encuentra actualmente en boga es el de sensores y sistemas que adquieran imágenes de diferentes tipos. La captura de imágenes requiere de sistemas digitales de alta velocidad que tengan la capacidad de acarrear los datos desde el lugar físico en donde se obtienen los datos, es decir, en el transductor mismo, hasta el circuito o el sistema destinado al proceso de los mismos.\\

Para este tipo de aplicaciones, se torna de interés la utilización de un tipo particular de dispositivos electrónicos denominados FPGA ({\it Field-Programmable Gate Array}, o en español, Arreglo de Compuertas Programables en Campo), circuitos integrados programable para sintetizar circuitos digitales de alta velocidad. En otras palabras, este tipo de componentes permite implementar circuitos que ejecutan tareas complejas y resuelve problemas específicos.\\

Existen numerosos trabajos que utilizan FPGA para la producción de imágenes. Por ejemplo, el desarrollo de un sensor de radiación utilizando un sensor CMOS comercial. Para ello, los autores utilizaron un FPGA para configurar y transmitir imagenes a un grabador de video a través de puerto UART. Esto permitió adquirir una imagen a través de un disparador realizado con un pulsador\cite{Perez2017}.\\

Se denomina ultrasonografía a la técnica de adquirir imágenes basandose en rebotes de ultrasonido. Sus aplicaciones son múltiple, en las que se destaca el diagnóstico médico, ya que es una técnica no invasiva y sin riesgos de radiación ionizante sobre el paciente. Un trabajo reciente desarrolló un sistema de ecografía médica con bajo costo utilizando un FPGA\cite{biswas2018embedded}. El autor también presentó un algoritmo realizado y probado en PC. Luego se implementó e en una FPGA.\\

Existen sistemas de telescopía implementados con FPGA. Yanagisawa y otros desarrollaron un sistema con telescopios pequeños para explorar objetos de campo cercano con la finalidad de monitorear cuerpos celestes que puedan colisionar con el planeta\cite{Yanagisawa2018}. Los autores utilizan la velocidad de los circuitos implementados en FPGA para minimizar el tiempo de adquisición.

Existen sensores de imagenes de distancia desarrollados mediante FPGA\cite{Cui2018}.\\%BORRAR esta cita!!!

A diferencia de un microprocesador o un microcontrolador, también muy usados en la industria electrónica, en el cual una unidad logica algoritmica ejecuta un programa cargado secuencial, es decir, linea por línea, un FPGA puede ser programado de forma tal que cada proceso se ejecute en forma independiente y paralela, dotando al sistema de una mayor velocidad en el procesamiento.\\

De esta forma, un diseñador puede manipular un volumen mucho mayor de datos, que a los efectos de la adquisición y medición de imágenes, resulta más adecuado.\\

Pero como ya se mencionó con anterioridad, la obtención de datos por si misma no le otorga al científico la información, y por ende, el conocimiento nuevo que desea. Para ello, es probable que este gran flujo de datos requiera de un procesamiento y análisis más exhaustivo de los mismos.\\

Es por esto que la PC {\it Personal Computer} se ha transformado en la herramienta indispensable en cualquier ámbito, pero en especial en los entornos en donde se requiere el  manejo, cálculo, procesamiento y análisis de grandes cantidades de datos e información.\\

Se torna de suma utilidad, entonces, proveer una comunicación efectiva y robusta que permita mover grandes volúmenes de datos en poco tiempo, y de esta forma facilitar los tiempos de desarrollo, las pruebas y depuración de sistemas.\\

Desde la inclusión de la norma USB, en el año 1996 a la fecha, se ha convertido en el elemento que no falta en ningun equipo, al punto tal que ha desplazado a cualquier otro conector. Al punto tal es esto, que para requerir algún puerto adicional que no sea de esta norma, cualquier comprador debe especificar que así sea, mas no es necesario especificar que tiene USB como norma de conexión.\\

El presente trabajo pretende ofrecer una comunicación basada en la versión 2.0 de la norma USB entre una PC y un FPGA, y otorgue una herramienta adicional a científicos que desarrollen o utilicen sistemas basados en FPGA.\\
%
%Este trabajo, pretende elaborar una interfaz entre los dos extremos, es decir, entre la PFGA y la PC, de forma tal que permita a un desarrollador, investigador o usuario en general, obtener una comunicación confiable y con un ancho de banda que permita mover el flujo de datos que genera una sensor que adquiera imágenes.\\
%
%Es cierto que el protocolo puede ser totalmente implementado en una FPGA, sin embargo, esto requeriría un muy alto costo tanto económico como en recursos disponibles del chip programable para una tarea genérica que es mejor elaborar con un circuito integrado diseñado especialmente para tal fin. Es por esto que se utiliza como lazo de interfaz un chip comercial elaborado por Cypress Semiconductor.\\
%
%
%
%[1][https://ieeexplore.ieee.org/abstract/document/8214376]
%[2][http://www.idr.iitkgp.ac.in/jspui/bitstream/123456789/9068/1/NB15975_Abstract.pdf]
%[3][https://ieeexplore.ieee.org/abstract/document/8396725]




%El mundo actual, en el que vivimos inmersos, demanda y consume volumenes cada vez más grandes de información. Con solo hacer una rapida miradad en diarios, incluso no especilizados, se observa la importancia que poseen las ciencias y disciplinas que manejas la informacion, aquellas areas agrupadas dentro del conjunto Técnico de la Información y la Comunicación, o más ocnocido por sus siglas, TIC's.
%
%Internet de las cosas , Big Data, Inteligencia Artificial, Redes Neuronales, Robótica, Domótica, entre otras, son areas en las que los datos y la información es abultada y su correcto manejo es sumamente complejo e importante. Además, ninguna de las actividades científicas, puede escapar de esta gran demanda mundial de información.
%
%La información es un conjunto de datos ordenados y procesados de forma tal que permita al que lo lea que eleve su nivel de conocimiento sobre un determinado tema. Es decir que, para que exista información, en primer lugar tenemos que tener datos, de forma tal que podamos luego procesarlos y obtener realmente información de llos.
%
%Como ejemplo de esto, podemos citar simplemente el acto de medir la presión dentro de un tubo de gas. Sería normal pensar en que, teniendo este objetivo en mente, simplemente coloquemos un manómetro en la salida del tubo. El manómetro es un artefacto que posee una varilla 
%
%
%
%
%
%El mundo, durante la era de la información, nos exige producir y consumir cada vez más y más información. En cada aspecto social dela vida de la persona se pueden encontrar ejemplos claros de esto.\\
%
%Bajo un punto de vista social nos vemos bombardeados por información. Los 'influencers' que se mueven a traves de Facebook, Twiter, Instagram, Snapchat, Linkedin, Youtube, etc, necesitan para estar a la moda, producir constantemente material y que ese material sea consumido por alguien que este demandando esa información.
%
%En el area de econimía y finanzas existen cada vez más robots que extraen información de las diferentes bolsas, periodicos, bancos, industrias y donde se ocurra que pueda haber información útil, para tomar mejores decisiones que, por supuesto, también ejecutan los robots.
%
%En el comercio, desde las tiendas de supermercado hasta las tiendas digitales que trabajan solo a través de internet están constantemente recabando datos y procesandolos a din de diseñar estrategias que permitan ofrecer productos que se adapten mejor a lo que busca el cliente y de esa forma lograr vender mayores volumenes.
%
%La industria cuenta cada vez más con multiplicidad de posibilidades basadas en sensores que adquieren grandes cantidades de datos que son almacenadas y procesadas, muchas vecen 'en línea' o 'al instante', de forma tal de encontrar mejores procesos o ejecutar nuevas tareas, como por ejemplo la industria automotriz, enfocada cada vez más en autos que puedan manipular todo el entorno y ser totalmente autónomos de un conductor.
%La industria espacial y satelital dotan a sus equipos de mayores sensores y mayores flujos de datos. La industria médica brinda cada vez más posibilidades de diagnóstico a traves de nuevas técnicas y formas de adquisición de imágenes.
%
%A todo esto, la ciencia y el desarrollo de nuevas aplicaciones y equipos, no son indiferente. Cada vez se encuentran más innovaciones y desarrollos de nuevos sensores que adquieren flujos de información crecientes. La toma de imagenes se ha vuelto una herramienta clave para la investigación en diversas areas, tales como la biología, la ciencia de materiales, las ciencias nucleares, las ciencias de la tierra, etc.
%
%Las redes sociales son un ejemplo de esto, pero esto se observa en la industria, la economía, las finanzas e incluso hasta en la casa.
%
%
%
%[1] file:///home/lechuzin/Facultad/Trabajo%20Final/lechuzing/docs/bibliografia/The-Rise-of-the-Network-Society-With-a-New-Preface-Volume-I-The-Information-Age-Economy-Society-and-Culture-Information-Age-Series-.pdf