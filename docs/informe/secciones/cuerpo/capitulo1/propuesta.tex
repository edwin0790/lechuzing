Dadas la motivación del presente trabajo, se podría decir a priori que el objetivo del presente trabajo es la implementación de una efectiva comunicación entre una Computadora Personal (PC) y una FPGA. Este objetivo, se plasmará de forma más concreta en la Sección \ref{int:obj} del presente Capitulo.\\

El dato de diseño más relavante es el poder transmitir imagenes por la comunicación a implemetar. ¿Pero cuantos datos son suficiente para transmitir imagenes? Para nuestro diseño, basaremos como sensor al que utiliza Perez en su Tesis de Maestría \cite{Perez2018}, un sensor de imagenes monocromáticas MT9M001C12STM, comercializado por Actina Imaging \cite{Micron Technology2014} que transmite las imagenes a una tasa de 48 Mbps.\\

Además, como especifición impuesta por la motivación, la implementación debe posee\%TODO Atención con estos parrafos!!!!
 una tasa de bit que permita transmitir imagenes y que los puertos sean facilmente accesibles en PCs comerciales, resaltan tres protocolos que permitirían lograr este cometido: Ethernet, USB y Wi-Fi. Estos protocolos, son los que actualmente se encuentran presente en cualquier aparato nuevo. Estas normas, entre otras, han dejado de lado a estandares que antes eran muy comunes y que algunos periféricos aún cuentan, como ser RS-232 o PS/2, entre otras.\\ 

En una primera aproximación, la que mayor tasa de datos puede proveer, sin dudas es el estandar Ethernet. Estas comunicaciones pueden alcanzar hasta \SI{400}{\giga bp\second}. Sin embargo, la norma Ethernet está principalmente pensada para redes de computadoras, por lo general se dipone de un solo puerto, el cual puede estar conectado a una red de internet y un periférico que tenga este puerto como conexión requerirá de alguna infrastructura adicional con cables más o menos extensos para lograr la comunicación.\\

En el caso de tratar de utilizar una comunicación via Wi-Fi, es posible que se necesite algún enrutador adicional a la hora de conectarse. A su vez, la tecnología inalámbrica con mayor ancho de banda está disponible hace unos pocos años y no todos los equipos cuentan con esta posibilidad, ofreciendo en esos casos una tasa máxima de \SI{54}{\mega bp \second}. La tasa de transmisión real máxima, descontando todos los encabezados y las colas que posee la norma, es de \SI{19}{\mega bp\second}.