El presente informe se divide en 2 bloques principales: uno referido al desarrollo del sistema y el siguiente a su forma de uso y verificación.

Dentro del bloque referido al desarrollo del sistema, se encuentran los primeros 5 capítulos:

\begin{enumerate}
	\item {\bf \nameref{cap:int}:} En este capítulo se intenta exponer lo que motiva el presente trabajo, la propuesta que da solución a la motivación, el objetivo y alcance que el trabajo busca y la estructura del mismo. Se brindan, además, conceptos importantes de la norma USB que son significativos para los objetivos de este trabajo.
%	\item {\bf \nameref{cap:mats}:} Se describe aquí todas las herramientas de las que se vale este trabajo para cumplir con os objetivos propuestos.
	\item {\bf \nameref{cap:cy}:} Se justifica la elección de la interfaz USB, compuesta por el controlador FX2LP EZ-USB comercializado por Cypress Semiconductor. Se presenta la arquitectura, configuración y código desarrollado para el presente trabajo. 
	\item {\bf \nameref{cap:fpga}:} Este capítulo detalla el desarrollo de una Máquina de Estados Finita para implementar la comunicación entre la FPGA y la interfaz USB. Se justifica la elección del FPGA Spartan 6 fabricado por Xilinx Inc., utilizado en este trabajo.La Máquina de Estados Finita es descripta en VHDL y sintetizada en un FPGA. También se describe un circuito impreso realizado para conectar el FPGA con la Interfaz USB.
	\item {\bf \nameref{cap:verif}:} Se detalla el desarrollo de un sistema de pruebas que conecta el FPGA con la interfaz, y a través de una memoria FIFO sintetizada en FPGA, se envían y se reciben paquetes en y desde una PC, respectivamente. Los paquetes son generados y transmitidos desde la PC a través de un programa elaborado para este trabajo. Finalmente se exponen las pruebas, los resultados de ellas y las conclusiones de este trabajo.
\end{enumerate}