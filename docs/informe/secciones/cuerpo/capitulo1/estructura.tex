El presente trabajo puede ser descompuesto en tres etapas bien diferenciados. La primera de ellas está referida a la selección y configuración de la interfaz. La segunda es dedicada al \acrshort{fpga}, su elección y la elaboración de un módulo que permita comunicar el \acrshort{fpga} a la interfaz. Luego, en la tercer etapa, se elabora un sistema que ensamble las dos etapas anteriores.
Este esquema se expone, a lo largo de este trabajo, en cinco capítulos:

\begin{enumerate}
	\item {\textbf{\nameref{cap:int}:}} En este capítulo se intenta exponer lo que motiva el presente trabajo, la propuesta que da solución a la motivación, el objetivo y alcance que el trabajo busca y la estructura del mismo. Se brindan, además, conceptos importantes de la norma \acrshort{usb} que son significativos para los objetivos de este trabajo.
%	\item {\bf \nameref{cap:mats}:} Se describe aquí todas las herramientas de las que se vale este trabajo para cumplir con os objetivos propuestos.
	\item \textbf{\nameref{cap:cy}:} Se justifica la elección de la interfaz \acrshort{usb}, compuesta por el controlador FX2LP EZ-USB comercializado por Cypress Semiconductor. Se presenta la arquitectura, configuración y código desarrollado para el presente trabajo. 
	\item \textbf{\nameref{cap:fpga}:} Este capítulo detalla el desarrollo de una Máquina de Estados Finita para implementar la comunicación entre el \acrshort{fpga} y la interfaz \acrshort{usb}. Se justifica la elección del \acrshort{fpga} Spartan 6 fabricado por Xilinx Inc., utilizado en este trabajo. La Máquina de Estados Finita es descripta en \acrshort{vhdl} y sintetizada en un \acrshort{fpga}. También se describe un circuito impreso realizado para conectar el \acrshort{fpga} con la Interfaz \acrshort{usb}.
	\item \textbf{\nameref{cap:verif}:} Se detalla el desarrollo de un sistema de pruebas que conecta el \acrshort{fpga} con la interfaz, y a través de una memoria \acrshort{fifo} sintetizada en \acrshort{fpga}, se envían y se reciben paquetes en y desde una \acrshort{pc}, respectivamente. Los paquetes son generados y transmitidos desde la \acrshort{pc} a través de un programa elaborado para este trabajo. Finalmente se exponen las pruebas, los resultados de ellas.
	\item \textbf{\nameref{cap:fin}:} Se presentan las conclusiones obtenidas a lo  largo del trabajo, y se presentan algunas líneas de trabajo futuro para profundizar el aprendizaje y mejorar el desarrollo realizado.
\end{enumerate}

	Al finalizar este trabajo, se agregan cuatro apéndices en donde se encuentran los códigos y esquemáticos elaborados durante la realización del este trabajo final.