El presente informe se divide en 2 bloques principales: uno referido al desarrollo del sistema y el siguiente a su forma de uso y verificación.\\

Dentro del bloque referido al desarrollo del sistema, se encuentran los primeros 5 capítulos:

\begin{enumerate}
	\item {\bf \nameref{cap:int}:} En este capítulo se intenta exponer lo que motiva el presente trabajo, la propuesta que da solución a la motivación, el objetivo y alcance que el trabajo busca y la estructura del mismo. Se brindan, además, conceptos importantes de la norma USB que son significativos para los objetivos de este trabajo.
	\item {\bf \nameref{cap:mats}:} Se describe aquí todas las herramientas de las que se vale este trabajo para cumplir con os objetivos propuestos.
	\item {\bf \nameref{cap:cy}:} Se presenta la arquitectura, configuración y código desarrollado para el presente trabajo, como así también las herramientas específicas provistas por el fabricante, que facilitan el desarrollo. 
	\item {\bf \nameref{cap:fpga}:} Este capítulo detalla lo desarrollado para implementar la comunicación entre la FPGA y la interfaz. Se expone una maquina de estados descrita en VHDL y sintetizada en FPGA. También se describe un circuito impreso realizado para conectar ambas partes.\\
	\item {\bf \nameref{cap:verif}:} Se desarrolla las tareas desarrolladas a fin de realizar las depuraciones del sistema y la verificación del cumplimiento de las especificaciones.\\
\end{enumerate}