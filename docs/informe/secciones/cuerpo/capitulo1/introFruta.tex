Un carpintero desea medir la distancia de una barra de madera que luego será, tal vez, la altura de las patas de una futura mesa. Para ello, utiliza una cinta métrica, compuesta de una cinta metálica que posee una escala graduada. Sabe entonces que la barra mide la distancia que coincide con la distancia de la cinta graduada.\\

Un panadero desea medir cuanto pesa la harina que debe para poder amasar. Entonces, la coloca en una balanza y observa cuanto marca su indicador. Así conoce que la masa de la harina es equivalente a la fracción de medida que indica la balanza.\\

Un atleta desea conocer cuanto demora en correr un trayecto que posee \SI{1.00}{\kilo\meter}. Por esto, registra el valor que indica su reloj al principio del recorrido y cuando alcanza el final observa nuevamente el artefacto. Luego de esto, calcula la diferencia entre el valor final y el inicial, conociendo cuanto tiempo le tomó realizar su travesía.\\

En los tres casos anteriores, tanto el carpintero, como el panadero y el atleta desconocen algo y necesitan cambiar su estado con respecto a esa incertidumbre. Por ello recurren a diferentes objetos, a fin de obtener conocimiento a partir de ellos. Sin embargo, estos objetos, por si mismos, no otorgan información, sino más bien otorgan un dato, que comparado y contrastado con otros datos, se traducen en conocimiento.\\

La información es el resultado de ordenar y procesar un conjunto de datos, de forma tal que permitan cambiar el estado de conocimiento sobre un asunto determinado. En el caso del carpintero, compara el tamaño de las patas de la mesa con una cinta metlálica, que a su vez, posee registrada su distancia en función de algún patrón de metrología, establecido por convención. Esto quiere decir que el dato 1, la longitud del patron, junto al dato 2, escala graduada de la cinta, más el dato 3, la longitud de la cinta métrica, permiten al carpintero cambiar su estado de desconocido a conocido, con respecto a la longitud del trozo de madera, a través de la información proporcionada por el conjunto de datos.\\

Se puede realizar el mismo análisis con respecto a la balanza del panadero, considerando un peso patrón, un desplazamiento y una escala graduada o una señal eléctrica emitida por una celda de carga deformada un porcentaje de su capacidad, registrada previamente por su fabricante conforme a pesos patrones, y un circuito adaptador que transforma esa señal electrica en un valor numérico mostrado en un indicador.\\

El atleta compara las posiciones y los desplazamientos de las agujas de su reloj, previamente calibrado para que dé una vuelta por cada minuto en una aguja, otra aguja que dé una vuelta por hora y la tercera una vez cada 12 horas. Además, es probable que él haya ajustado la hora que indica el reloj para que otorgue un horario idéntico al de referencia, establecido por convención.\\

En todos los casos, se posee una gran cantidad de datos que, ordenados, procesados y comparados otorgan al usuario un valor útil, ya sea una longitud, una masa, un tiempo o cualquiera sea la variable física que se desee conocer.\\

La ciencia es un conjunto de técnicas y procedimientos que, a través del método científico, busca adquirir, descubrir y/o desarrollar nuevo conocimiento. Se desprende entonces, que la ciencia produce, de forma fundamental, información que luego es transformada en conocimiento. Cuando hablamos de ciencia, hablamos de una gran gama de objetos de estudio, sujeto a través del cuál se clasifican, en la mayoría de los casos, las ciencias: las Ciencias Sociales estudian las relaciones humanas, las Ciencias Naturales estudian objetos que se encuentran en la naturaleza, las Ciencias de la Tierra se enfocan en una rama más particular de la naturaleza, como lo son los minerales, la superficie terrestre, etc; y siguiendo así se puede encontrar un sinnuméro se ciencias. Sin embargo, toda ciencia necesita, para su correcta producción científica, adquirir una gran cantidad de datos que luego será nordenados, procesados y transformados en información y conocimiento.\\

La incorporación de una herramienta especialmente diseñada para el procesamiento de datos, como lo es la computadora, permite manejar un numero cada vez creciente de información.
Es por eso que se encuentra en desarrollo un gran número de sensores y dispositivos que permitan obtener cada vez más datos.\\

En este sentido, uno de los desarrollo que se encuentran en boga es el sensores que adquieran imágenes. Como ejemplos podemos encontrar, entre muchos otros, el desarrollo de sensores de radiación[1], ultrasonografía[2], telescopía de objetos cercanos[3], imagenes de distancia[4].

La captura de imágenes, fundamentalmente en el desarrollo de sensores nuevos, requiere de sistemas digitales de alta velocidad que tengan la capacidad de acarrear los datos desde el lugar físico en donde se obtienen los datos, es decir, en el transductor mismo, hasta el circuito o el sistema destinado al proceso de los mismos. De esta forma, toma particular interés la utilización de FPGA's, circuitos integrados diseñados para que un diseñador pueda sintetizar un circuito digital de alta velocidad reprogramable en el cual se puede implementar, con ciertas restricciones, circuitos desarrollados para una tarea muy especifica que resuelva la tarea que el diseñador necesite.\\

A diferencia de un microprocesador o un microcontrolador, también muy usados en la industria electrónica, en el cual una unidad logica algoritmica ejecuta un programa cargado secuencial, es decir, linea por línea, un FPGA puede ser programado de forma tal que cada proceso se ejecute en forma independiente y paralela, dotando al sistema de una mayor velocidad en el procesamiento.\\

De esta forma, un diseñador puede manipular un volumen mucho mayor de datos, que a los efectos de la adquisición y medición de imágenes, resulta más adecuado.\\

Pero como ya se mencionó con anterioridad, la obtención de datos por si misma no le otorga al científico la información, y por ende, el conocimiento nuevo que desea. Para ello, es probable que este gran flujo de datos requiera de un procesamiento y análisis más exhaustivo de los mismos.

Es por esto que la PC {\it Personal Computer} se ha transformado en la herramienta indispensable en cualquier ámbito, pero en especial en los entornos en donde se requiere el  manejo, calculo, procesamiento y análisis de grandes cantidades de informacion de diferentes índoles.\\

Desde la inclusión de la norma USB, en el año 1996 a la fecha, se ha convertido en el elemento que no falta en ningun equipo, al punto tal que ha desplazado a cualquier otro conector. Al punto tal es esto, que para requerir algún puerto adicional que no sea de esta norma, cualquier comprador debe especificar que así sea, mas o es necesario especificar que tiene USB como norma de conexión.\\

Este trabajo, pretende elaborar una interfaz entre los dos extremos, es decir, entre la PFGA y la PC, de forma tal que permita a un desarrollador, investigador o usuario en general, obtener una comunicación confiable y con un ancho de banda que permita mover el flujo de datos que genera una sensor que adquiera imágenes.\\

Es cierto que el protocólo puede ser totalmente implementado en una FPGA, sin embargo, esto requeriría un muy alto costo tanto económico como en recursos disponibles del chip programable para una tarea genérica que es mejor elaborar con un circuito integrado diseñado especialmente para tal fin. Es por esto que se utiliza como lazo de interfaz un chip comercial elaborado por Cypress Semiconductor.\\



%[1][https://ieeexplore.ieee.org/abstract/document/8214376]
%[2][http://www.idr.iitkgp.ac.in/jspui/bitstream/123456789/9068/1/NB15975_Abstract.pdf]
%[3][https://ieeexplore.ieee.org/abstract/document/8396725]
%[4][https://www.sciencedirect.com/science/article/pii/S0030402617316029]




%El mundo actual, en el que vivimos inmersos, demanda y consume volumenes cada vez más grandes de información. Con solo hacer una rapida miradad en diarios, incluso no especilizados, se observa la importancia que poseen las ciencias y disciplinas que manejas la informacion, aquellas areas agrupadas dentro del conjunto Técnico de la Información y la Comunicación, o más ocnocido por sus siglas, TIC's.
%
%Internet de las cosas , Big Data, Inteligencia Artificial, Redes Neuronales, Robótica, Domótica, entre otras, son areas en las que los datos y la información es abultada y su correcto manejo es sumamente complejo e importante. Además, ninguna de las actividades científicas, puede escapar de esta gran demanda mundial de información.
%
%La información es un conjunto de datos ordenados y procesados de forma tal que permita al que lo lea que eleve su nivel de conocimiento sobre un determinado tema. Es decir que, para que exista información, en primer lugar tenemos que tener datos, de forma tal que podamos luego procesarlos y obtener realmente información de llos.
%
%Como ejemplo de esto, podemos citar simplemente el acto de medir la presión dentro de un tubo de gas. Sería normal pensar en que, teniendo este objetivo en mente, simplemente coloquemos un manómetro en la salida del tubo. El manómetro es un artefacto que posee una varilla 
%
%
%
%
%
%El mundo, durante la era de la información, nos exige producir y consumir cada vez más y más información. En cada aspecto social dela vida de la persona se pueden encontrar ejemplos claros de esto.\\
%
%Bajo un punto de vista social nos vemos bombardeados por información. Los 'influencers' que se mueven a traves de Facebook, Twiter, Instagram, Snapchat, Linkedin, Youtube, etc, necesitan para estar a la moda, producir constantemente material y que ese material sea consumido por alguien que este demandando esa información.
%
%En el area de econimía y finanzas existen cada vez más robots que extraen información de las diferentes bolsas, periodicos, bancos, industrias y donde se ocurra que pueda haber información útil, para tomar mejores decisiones que, por supuesto, también ejecutan los robots.
%
%En el comercio, desde las tiendas de supermercado hasta las tiendas digitales que trabajan solo a través de internet están constantemente recabando datos y procesandolos a din de diseñar estrategias que permitan ofrecer productos que se adapten mejor a lo que busca el cliente y de esa forma lograr vender mayores volumenes.
%
%La industria cuenta cada vez más con multiplicidad de posibilidades basadas en sensores que adquieren grandes cantidades de datos que son almacenadas y procesadas, muchas vecen 'en línea' o 'al instante', de forma tal de encontrar mejores procesos o ejecutar nuevas tareas, como por ejemplo la industria automotriz, enfocada cada vez más en autos que puedan manipular todo el entorno y ser totalmente autónomos de un conductor.
%La industria espacial y satelital dotan a sus equipos de mayores sensores y mayores flujos de datos. La industria médica brinda cada vez más posibilidades de diagnóstico a traves de nuevas técnicas y formas de adquisición de imágenes.
%
%A todo esto, la ciencia y el desarrollo de nuevas aplicaciones y equipos, no son indiferente. Cada vez se encuentran más innovaciones y desarrollos de nuevos sensores que adquieren flujos de información crecientes. La toma de imagenes se ha vuelto una herramienta clave para la investigación en diversas areas, tales como la biología, la ciencia de materiales, las ciencias nucleares, las ciencias de la tierra, etc.
%
%Las redes sociales son un ejemplo de esto, pero esto se observa en la industria, la economía, las finanzas e incluso hasta en la casa.
%
%
%
%[1] file:///home/lechuzin/Facultad/Trabajo%20Final/lechuzing/docs/bibliografia/The-Rise-of-the-Network-Society-With-a-New-Preface-Volume-I-The-Information-Age-Economy-Society-and-Culture-Information-Age-Series-.pdf