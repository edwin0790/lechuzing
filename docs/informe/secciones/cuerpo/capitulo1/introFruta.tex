El presente informe busca dar a conocer al lector las tareas y actividades desarrolladas por el autor, a fin de dar cuenta sobre lo realizado en el marco del Trabajo Final de la carrera Ingeniería Electrónica, dictada en la Facultad de Ingeniería de la Universidad Nacional de San Juan.\\

El objetivo principal de este trabajo es la implementación de un sistema de comunicación mediante la norma USB 2.0 para sistemas científicos y/o tecnológicos desarrollados en FPGA.\\

A lo largo de este trabajo, se busca que el lector comprenda el problema que se pretende resolver, la configuración del sistema propuesto, los fundamentos que dan base a la dicha configuración y el modo de uso del sistema.\\

Para ello, este Capítulo explica, en primer lugar, la motivación, es decir, se busca aclarar la problemática que se busca resolver. Luego, se detallan los objetivos que persigue este trabajo. Seguido a esto, se otorga un esquema que describela solución planteada y se justifica el protocolo elegido. A continuación, el lector puede conocer la organización del presente informe. Finalmente, se repasan algunos conceptos importantes de la norma USB que luego se utilizan en el trabajo desarrollado.\\ 