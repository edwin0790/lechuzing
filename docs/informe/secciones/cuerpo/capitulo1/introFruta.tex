El presente informe busca dar a conocer al lector las tareas y actividades desarrolladas por el autor, a fin de dar cuenta sobre lo realizado en el marco del Trabajo Final de la carrera Ingeniería Electrónica, dictada en la Facultad de Ingeniería de la Universidad Nacional de San Juan.\\

El objetivo principal de este trabajo, tal como se mencionará más adelante, es la implementación de un sistema de comunicación mediante la norma USB 2.0 para sistemas científicos desarrollados en FPGA.\\

A lo largo de este trabajo, se busca que el lector pueda entender el problema que se pretende resolver, la configuración del sistema propuesto, los fundamentos que dan base a la dicha configuración y el modo de uso del sistema.\\

Para ello, este Capítulo explica, en primer lugar, los objetivos que persigue este trabajo. Luego, el lector puede conocer la organización del presente informe. A continuación, se detalla la motivación, es decir, se busca aclarar la problemática que se pretende resolver. Lo siguiente es un modelo esquemático de la solución planteada. Además, se comparan diferentes protocolos de comunicación. Con esto se busca justificar la elección del protocolo a implementar y finalmente, se brinda un repaso de algunos conceptos importantes de la norma USB que luego se utilizan en el trabajo desarrollado.\\ 