El presente informe busca dar a conocer al lector las tareas y actividades desarrolladas por el autor, a fin de dar cuenta sobre lo realizado en el marco del Trabajo Final de la carrera Ingeniería Electrónica, dictada en la Facultad de Ingeniería de la Universidad Nacional de San Juan. A lo largo de él, se busca que el lector comprenda la problemática que se pretende resolver y la configuración, fundamentos y modo de uso del sistema propuesto. El objetivo es diseñar e implementar una interfaz USB para la transmisión de datos a alta velocidad, adquiridos por sistemas desarrollados en FPGA para aplicaciones científicas.\\


En este Capítulo, se explica, en primer lugar, la motivación, es decir, se busca aclarar la problemática que se busca resolver. Luego, se detallan los objetivos que persigue este trabajo. Seguido a esto, se otorga un esquema que describe la solución planteada y se justifica el protocolo elegido. A continuación, el lector puede conocer la organización del presente informe. Finalmente, se repasan algunos conceptos importantes de la norma USB que luego se utilizan en el trabajo desarrollado.\\ 