En el presente capítulo se abordaron las partes más relevantes a los fines de este trabajo que corresponden a la norma USB. Resulta importante comprender la norma a fin de lograr una correcta configuración de las partes involucradas en cada etapa del desarrollo.%\\

A partir de la descripción global de un sistema, se pudo comprender cómo identificar los diferentes puertos que intervienen en la norma, las velocidades de operación de un sistema USB, las tensiones que intervienen, la codificación usada y la topología del bus. Se identificaron a su vez, los dispositivos que intervienen y el rol que ocupan dentro del sistema.%\\

Se desarrollaron conceptos sobre los componentes más importantes del protocolo, la forma en que los mensajes se empaquetan, los campos y etiquetas que componen un paquete. Se detallaron los tipos de transferencias USB y sus características. También, se abordaron los descriptores a través de los cuales los dispositivos comunican sus características al Host.