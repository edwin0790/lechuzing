Para implementar un sistema de comunicación entre la \acrshort{pc} y el \acrshort{fpga} a través del protocolo \acrshort{usb}, se propuso utilizar un dispositivo intermedio que cumpla el rol de interfaz. El dispositivo utilizado para este propósito es el controlador EZ-USB FX2LP, diseñado por la empresa Cypress Semiconductor, el cual viene incorporado en el kit de desarrollo CY3684 \cite{CypressSemiconductor2014cy3684}, fabricado por la misma empresa.

El kit de desarrollo CY3684 puede ser descompuesto en dos partes: una de hardware, que posibilita la conexión eléctrica entre los componentes y una parte de software que facilita al desarrollador tanto la elaboración del programa que es cargado y ejecutado por el microcontrolador (denominado firmware), como las pruebas del sistema en desarrollo.

En este capítulo se justifica la selección de la interfaz, se presenta la configuración que mejor se adapta a los objetivos, se detalla el firmware elaborado y se abordan algunos aspectos conceptuales sobre la estructura y arquitectura del circuito integrado seleccionado como interfaz y las herramientas utilizadas.%Un puerto USB posee una estructura fija: 4 contactos; dos de alimentación y dos datos. A su vez, el Protocolo USB es muy específico en cuanto a los niveles de tensión, la frecuencia de comunicación, las cadenas de bits clave que deben enviarse, entre muchas otras cosas que detalla la {Norma USB 2.0}~\cite{Compaq2000}.%\\

%Un FPGA, en contraste con lo anterior, es un dispositivo electrónico que posee en su interior una gran cantidad de elemento lógicos programables. Esto permite la síntesis de casi cualquier circuito lógico. Dicha síntesis puede poseer la cantidad y disposicion de puertos como pines posea el FPGA, exceptuando algunas entradas específicas para el funcionamiento del FPGA. Esto configura una gran versatilidad que permite ajustar un sistema desarrollado dentro del FPGA a la medida necesaria.%\\

%El nexo entre los datos que se producen en sistema en el que se desenvuelve un FPGA y un puerto USB, se lo denomina interfaz. Esta última, en el presente trabajo, está compuesta por un circuito integrado particular, cuya codificación comercial es CY7C68013A, fabricado por Cypress Semiconductor. Este componente electrónico pertenece a la familia de controladores FX2LP del catálogo de integrados EZ-USB. El chip CY7C68013A se encuentra incorporado el kit de desarrollo CY3684 EZ-USB Development Kit, provisto por el fabricante.%\\

%Un kit de desarrollo es un conjunto de herramientas que permiten elaborar soluciones electrónicas que requieren un componente en particular. Además, cuenta con algunos dispositivos genéricos que posibilitan emular el sistema que utilizará el componente central del kit. En el caso del kit que se utiliza en este trabajo se descompone en dos grandes grupos: hardware y software. En la parte de hardware se incluye de un circuito impreso (PCB) que posee soldado, el integrado que nos servira de interfaz y otros componentes que se describen en el Capítulo \ref{cap:mats}.%\\

%En lo que a software se refiere, Cypress incorpora en el kit de desarrollo un código marco que posee escritas todas las funciones y registros que ejecutan las tareas que el sistema necesita para llevar a cabo la comunicación USB. Además se cuenta con herramientas de software que permite escribri, compilar y cargar el programa que se ejecuta en el controlador FX2LP.%\\

%En este capítulo se desarrolla la arquitectura de los integrados FX2LP y se abordan los módulos más relevantes para este trabajo. También se explican las herramientas del kit que fueron utilizadas para la elaboración del firmware y, finalmente, los aspectos más sobresalientes del firmware que ejecuta el CY7C68013A, y establece el funcionamiento de la comunicación USB.