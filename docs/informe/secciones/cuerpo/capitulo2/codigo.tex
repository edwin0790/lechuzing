%A continuación se desarrollan los aspectos más relevantes del código final elaborado, compuestos por la función de inicialización y los descriptores del dispositivo USB.%\\
%
%Al establecer el modo automático de funcionamiento, la función \verb|TD_Poll()| se encuentra vacía.%\\
%
%Luego, en el Capitulo 5 se abordará nuevamente algunos detalles realizados para la depuración del presente código, tales como la conexión del puerto serie y la utilización de algunas interrupciones específicas.
%
%\subsection{Inicialización del dispositivo}
%	La inicialización del dispositivo se realiza a través de la función \verb|TD_Init()|. Ésta es invocada solo una vez en el código, antes de ejecutar el loop principal, donde el programa ejecuta tareas específicas una y otra vez.%\\
%	
%	En primer lugar, se debe configurar la frecuencia a la que corre el reloj principal. Esta puede ser de \SI{12}{\mega\hertz}, \SI{24}{\mega\hertz} o de \SI{48}{\mega\hertz}. Para ello se deben colocar los registros CPUCS.4=1 y CPUCS.3=0. A través del framwork de Cypress, esto puede ser escrito de la siguiente forma:
%	
%	\begin{lstlisting}[language=C,backgroundcolor=\color{gray!30}]
%	CPUCS = ((CPUCS & ~bmCLKSPD) | bmCLKSPD1); // 48 MHz
%	\end{lstlisting}
%
%	\begin{figure}[ht]
%		\centering
%		\begin{tikzpicture}[circuit logic US, node distance=1,shape border uses incircle]
%			\begin{scope}[transform shape]
%				\node[mux, inputs={nn},info={[text width=10]center:0\\1}] (ifcfg4){};
%				\node[buffer gate] (ifcfg5) [right=of ifcfg4] {};
%				\node[not gate]	(a)		[left=of ifcfg4.input 2]	{};
%				\node[node distance=2](aux a)	[left=of ifcfg4.input 1]	{};
%				\node[] (aux 1) [left=of aux a] {};
%				\node[mux, inputs={nn},info={[text width=10]center:0\\1}] (ifcfg6) [left=of aux 1]{};
%				\node[mux, inputs={nn},info={[text width=10]center:1\\0},point left,node distance=4] (ifcfg7) [below=of ifcfg6]{};
%				\node[mux,inputs={nn},info={[text width=10]center:0\\1},node distance=2,point left] (ifcfg4bis) [right=of ifcfg7.input 1]{};
%				\node[not gate,node distance=1.7] (b) [right=of ifcfg4bis.input 1,point left] {};
%				\node[node distance=3] (aux b) [right=of ifcfg4bis.input 2] {};
%				\node[buffer gate,point left] (c)	[right=of aux b]	{};
%				
%				\node[]	(sel4)	[below=of ifcfg4]	{IFCFG.4};
%				%	\node[node distance=1.5] (sel4bis) [above=of ifcfg4bis] {IFCFG.4};
%				\node[]	(sel5)	[above=of ifcfg5]	{IFCFG.5};
%				\node[] (sel6)	[below=of ifcfg6]	{IFCFG.6};
%				\node[node distance=1.5]	(sel7)	[above=of ifcfg7]	{IFCFG.7};
%				\node[] (aux 2) [right=of ifcfg5]		{};
%				\node[node distance=1.5] (aux 3)  [below=of aux 2] {};
%				\node[align=center] (output) [right=of aux 3] {Pin\\IFCLK};
%				\node[] (freq1)	[left=of ifcfg6.input 1]	{30 MHz};
%				\node[]	(freq2) [left=of ifcfg6.input 2]	{48 MHz};
%				\node[align=center] (clkin) [left=of ifcfg7]	{Entrada\\IFCLK\\Interna};
%				
%				\draw (ifcfg4) to (ifcfg5);
%				\draw (aux a.base) to (ifcfg4.input 1);
%				\draw (aux a.base) |- (a.input);
%				\draw (ifcfg6) to (aux a.base);
%				\draw (aux 1.base) |- (ifcfg7.input 2);
%				\draw (ifcfg4bis) to (ifcfg7.input 1);
%				\draw (c) to (ifcfg4bis.input 2);
%				\draw (b) to (ifcfg4bis.input 1);
%				\draw (aux b.base) |- (b.input);
%				\draw (a) to (ifcfg4.input 2);
%				\draw (ifcfg5) to (aux 2.base);
%				\draw (aux 2.base) |- (c);
%				\draw (aux 3.base) -- (output);
%				
%				\draw (clkin) to (ifcfg7);
%				\draw (freq1) to (ifcfg6.input 1);
%				\draw (freq2) to (ifcfg6.input 2);
%				\draw (sel5) to (ifcfg5.north);
%				\draw (sel4) to (ifcfg4.select);
%				\draw (sel4) -| (ifcfg4bis.select);
%				\draw (sel6) to (ifcfg6.select);
%				\draw (sel7) to (ifcfg7.select);	
%			\end{scope}
%		\end{tikzpicture}
%		\caption{Esquema funcional para la entrada de reloj de la Interfaz}
%		\label{ifclk};
%	\end{figure}
%
%	
%	Luego se debe configurar el funcionamiento del reloj de la interfaz. El esquema de la Figura \ref{ifclk} muestra la configuración del hardware. Cómo se observa, La interfaz puede funcionar con frecuencias de \SI{48}{\mega\hertz} o \SI{30}{\mega\hertz} provistas por el FX2LP. Además, la señal puede ser dirigida al exterior, o bien, ser provista por un periférico. Los inversores se programan de forma tal que la interfaz sea activa en el flanco positivo o negativo del reloj fuente. En este trabajo, el registro se programa para tomar como reloj la señal de \SI{48}{\mega\hertz} provisto por el mismo chip, la polaridad es de flanco ascendente y no se posee señal de reloj en el pin externo IFCKL. Además, se programa de modo asíncrono y en modo FIFO esclavo, a cuya configuración se accede por este puerto.%\\
%	
%	El autor entiende sobre la posibilidad y conveniencia de utilizar el modo sincrónico para conectar la interfaz. Más aún, es factible proveer la señal necesaria de reloj desde la FPGA y evitar así problemas de desajustes y fallas de sincronismo. Sin embargo, por error del alumno en el diseño del impreso de interconexión, la entrada de reloj no quedó conectada al chip de la FPGA. Durante la escritura de este informe, se encuentra en viaje el impreso con las correcciones pertinentes y espera ser implementado en trabajos futuros.%\\
%	
%	La configuración, entonces, queda definida por la sentencia de código:
%	
%%	\lstinputlisting[language=C,firstline=91,lastline=95]{\bridge}
%	\begin{lstlisting}[language=C,backgroundcolor=\color{gray!30}]
%	//colocar Interfaz FIFO esclava a 48MHz
%	//clk interno, no_salida, no_invertir clk, no_asincr
%	//fifoesclava(11) = 0xC3
%	//0xCB; para asincr.
%	IFCONFIG = 0xCB;
%	SYNCDELAY;
%	\end{lstlisting}
%		
%	A continuación, se debe establecer el funcionamiento de los pines bandera. Estos avisan cuando un EP está vacío, completo o a un nivel programable. Para este trabajo, son necesarios solo una bandera que señale el nivel vacío del puerto por el que entran los datos a la FPGA y otra que señale el nivel completo del EP donde escribe los datos a enviar. Si bien no son leídos en ningún momento, para evitar problemas se configuran las banderas vacías y completas para ambos EP's:
%	
%%	\lstinputlisting[language=C,firstline=108,lastline=113]{\bridge}
%	\begin{lstlisting}[language=C,backgroundcolor=\color{gray!30}]
%	//Pin Flags Configuratión
%	PINFLAGSAB = 0xBC;	// FLAGA <- EP2 Full Flag
%						// FLAGD <- EP2 Empty Flag
%	SYNCDELAY;
%	PINFLAGSCD = 0x8F;	// FLAGC <- EP8 Full Flag
%						// FLAGB <- EP8 Empty Flag
%	\end{lstlisting}
%	
%	Luego, se deben programar los EP's. La configuración por defecto define a todos los EP como transferencias por bultos con doble buffer de \SI{512}{\byte}. El EP2 y EP4 son salidas (desde la PC). El EP6 y EP8 son entradas (hacia la PC).%\\
%	
%	La programación de este trabajo es con una entrada isócrona de dos buffers con \SI{512}{\byte} de capacidad, cada uno, definida en el EP2 y una salida por bultos de dos buffers de \SI{512}{\byte} configurada en el EP8. Los otros EP's son deshabilitados. Sin embargo, EP1IN y EP1OUT se dejan configurados por defecto, ya que no interfieren en nada con el presente trabajo.
%	
%%	\lstinputlisting[language=C,firstline=117,lastline=128]{\bridge}
%	\begin{lstlisting}[language=C,backgroundcolor=\color{gray!30}]
%	EP1OUTCFG = 0xA0;
%	SYNCDELAY;
%	EP1INCFG = 0xA0;
%	SYNCDELAY;
%	EP4CFG &= 0x7F;
%	SYNCDELAY;
%	EP6CFG &= 0x7F;
%	SYNCDELAY;
%	EP8CFG = 0xA0; //EP8 is DIR=OUT, TYPE=BULK, SIZE=512, BUF=2x
%	SYNCDELAY;
%	EP2CFG = 0xD2;  // EP2 is DIR=IN, TYPE=ISOC, SIZE=512, BUF=2x
%	SYNCDELAY;
%	\end{lstlisting}
%	
%	Como siguiente paso, se limpian las memorias FIFO de cualquier dato espúreo que contengan al momento del inicio del programa.
%	
%	\begin{lstlisting}[language=C,backgroundcolor=\color{gray!30}]
%	FIFORESET = 0x80;
%	SYNCDELAY;
%	FIFORESET = 0x02;
%	SYNCDELAY;
%	FIFORESET = 0x04;
%	SYNCDELAY;
%	FIFORESET = 0x06;
%	SYNCDELAY;
%	FIFORESET = 0x08;
%	SYNCDELAY;
%	FIFORESET = 0x00;
%	SYNCDELAY;
%	\end{lstlisting}
%	
%	De esta forma, el controlador se encuentra listo para configurar las memorias asignadas a cada uno de los EP's. Se debe notar que en primer lugar se coloca 0x00 y luego el valor estipulado. Esto se basa en que el modo automático se prepara para transmitir ante un flanco ascendente del registro que lo habilita.
%	
%%	\lstinputlisting[language=C,firstline=143,lastline=152]{\bridge}
%	\begin{lstlisting}[language=C,backgroundcolor=\color{gray!30}]
%	EP8FIFOCFG = 0x00;
%	SYNCDELAY;
%	EP2FIFOCFG = 0x00;
%	SYNCDELAY;
%	
%	//setting on auto mode. rising edge is necessary
%	EP8FIFOCFG = 0x11;
%	SYNCDELAY;
%	EP2FIFOCFG = 0x0D;
%	SYNCDELAY;
%	\end{lstlisting}
%
%	Finalmente, se habilitan las interrupciones necesarias.
%	
%%	\lstinputlisting[language=C,firstline=179,lastline=183]{\bridge}
%	\begin{lstlisting}[language=C,backgroundcolor=\color{gray!30}]
%	USBIE |= bmSOF;
%	\end{lstlisting}
%
%	Así, queda completa la inicialización y el dispositivo listo para enviar y recibir datos de forma automática.
	
\subsection{Encabezado y declaraciones importantes}
	Para el correcto funcionamiento del código que se describe a lo largo de esta sección, es necesario incorporar el encabezado que se observa a continuación.
	
	\begin{lstlisting}[language=C,backgroundcolor=\color{gray!30}]
	#pragma noiv			// No generar vectores de interrupción
	#include "fx2.h"
	#include "fx2regs.h"
	#include "syncdly.h"	// SYNCDELAY macro
	#include "leds.h"
	\end{lstlisting}
	
	Las primeras 4 líneas de encabezados son provistas por Cypress, a través de su framework. En ellas, la directiva de ensamblador noiv(identificada con \verb|#pragma|), le indica al compilador que no debe habilitar las interrupciones vectorizadas. Estas, en cambio, serán manejadas y direccionadas a través de un archivo objeto provisto en el framework de Cypress Semiconductorm, denominado {\it usbjmptb.obj}.%\\
	
	El encabezado {\it leds.h} cuyo código se muestra a continuación, sirve para encender y apagar las luces de la placa de desarrollo. Los LED fueron utilizado para realizar las tareas de prueba. Se explicará su funcionamiento más adelante.%Estos leds se encuentran conectados a través de un decodificador y su funcionamiento se da con la sola lectura de direcciones específicas.
	
%	\begin{lstlisting}[language=C,backgroundcolor=\color{gray!30}]
%	xdata volatile const BYTE D2ON	_at_ 0x8800;
%	xdata volatile const BYTE D2OFF	_at_ 0x8000;
%	xdata volatile const BYTE D3ON	_at_ 0x9800;
%	xdata volatile const BYTE D3OFF	_at_ 0x9000;
%	xdata volatile const BYTE D4ON	_at_ 0xA800;
%	xdata volatile const BYTE D4OFF	_at_ 0xA000;
%	xdata volatile const BYTE D5ON	_at_ 0xB800;
%	xdata volatile const BYTE D5OFF	_at_ 0xB000;	\end{lstlisting}
	
	Luego, el framework define algunas variables globales que utiliza en las funciones implementadas para el manejo de las tareas relacionadas al protocolo USB. Se listan estas variables a continuación.
	
	\begin{lstlisting}[language=C,backgroundcolor=\color{gray!30}]
	extern BOOL   GotSUD;			// Received setup data flag
	extern BOOL   Sleep;
	extern BOOL   Rwuen;
	extern BOOL   Selfpwr;
	
	BYTE    Configuration;      	// Current configuration
	BYTE    AlternateSetting = 0;   // Alternate settings
	
	//--------------------------------------------------------------
	// Task Dispatcher hooks
	//   The following hooks are called by the task dispatcher.
	//--------------------------------------------------------------
	
	WORD blinktime = 0;
	BYTE inblink = 0x00;
	BYTE outblink = 0x00;
	WORD blinkmask = 0;			// HS/FS blink rate
	\end{lstlisting}
	
\subsection{Descriptores USB}	
	Los descriptores son una estructura definida de datos. A través de ellos, el dispositivo USB le comunica al anfitrión sus atributos, tales como velocidad de trabajo, cantidad de configuraciones e interfaces posibles, número de EPs, dirección de cada uno de ellos, tamaño máximo en bytes de paquetes que puede enviar en una comunicación, entre otros.%\\
	
	El framework de Cypress coloca toda la información sobre los descriptores del dispositivo desarrollado en un archivo escrito en lenguaje de ensamblador, denominado {\it dscr.a51}. Este archivo contiene una plantilla en donde el desarrollador coloca la descripción de la configuración realizada en el firmware. Luego, el archivo es enviado por el dispositivo al Host comunicando las características del sistema desarrollado.%\\
	
	En primer lugar, posee un encabezado en donde se establecen etiquetas que hacen más legible el código para el programador:
	\begin{lstlisting}[language={[x86masm]Assembler},backgroundcolor=\color{gray!30},numbers=left]
	DSCR_DEVICE   equ   1   ;; Descriptor type: Device
	DSCR_CONFIG   equ   2   ;; Descriptor type: Configuration
	DSCR_STRING   equ   3   ;; Descriptor type: String
	DSCR_INTRFC   equ   4   ;; Descriptor type: Interface
	DSCR_ENDPNT   equ   5   ;; Descriptor type: Endpoint
	DSCR_DEVQUAL  equ   6   ;; Descriptor type: Device Qualifier
	
	DSCR_DEVICE_LEN   equ   18
	DSCR_CONFIG_LEN   equ    9
	DSCR_INTRFC_LEN   equ    9
	DSCR_ENDPNT_LEN   equ    7
	DSCR_DEVQUAL_LEN  equ   10
	
	ET_CONTROL   equ   0   ;; Endpoint type: Control
	ET_ISO       equ   1   ;; Endpoint type: Isochronous
	ET_BULK      equ   2   ;; Endpoint type: Bulk
	ET_INT       equ   3   ;; Endpoint type: Interrupt
	\end{lstlisting}

	Luego, el programador se debe asegurar que el código se guarda en un lugar de memoria adecuado.
	
	\begin{lstlisting}[language={[x86masm]Assembler},backgroundcolor=\color{gray!30},numbers=left]
DSCR   SEGMENT   CODE PAGE

;;-----------------------------------------------------------------------------
;; Global Variables
;;-----------------------------------------------------------------------------
	rseg DSCR      ;; locate the descriptor table in on-part memory.
	\end{lstlisting}
	
	Debido a la estructura rígida del formato de los descriptores, impuesto por la norma USB y por la implementación de Cypress, la memoria debe contener en primer lugar el descriptor DEVICE, el cual se muestra a continuación.
	
	\begin{lstlisting}[language={[x86masm]Assembler},backgroundcolor=\color{gray!30},numbers=left]
DeviceDscr:
	db   DSCR_DEVICE_LEN      ;; Largo del descriptor
	db   DSCR_DEVICE   ;; Typo de descriptor
	dw   0002H      ;; Versión de la norma (BCD)
	db   00H        ;; Clase de Dispositivo
	db   00H         ;; Sub-Clase de Dispositivo
	db   00H         ;; Sub-sub-Clase de dispositivo
	db   64         ;; Tamaño máximo de paquete
	dw   0B404H      ;; Identificador de Vendedor (Cypress) 
	dw   0310H      ;; Identificador de Producto (Sample Device)
	dw   0000H      ;; Identificador de versión del producto
	db   1         ;; Indice de Fabricante en string
	db   2         ;; Indice de Producto en string
	db   0         ;; Indice de número de serie en string
	db   1         ;; Número de configuraciones
	\end{lstlisting}
	
	El descriptor de tamaño máximo de paquete está referido especialmente al EP0, es decir, al extremo que el Host y el dispositivo utilizan para intercambiar mensajes de control.
	
	Se debe notar que los penúltimos tres parámetros mostrados corresponden a una descripción realizada en cadena de caracteres al final del archivo, a fin de poder mostrar un mensaje que pueda ser leído por un usuario humano.
	
	El último parámetro está relacionado con el número de configuraciones que posee este dispositivo. Esto determina también la cantidad de descriptores de tipo CONFIGURATION que debe tener el archivo de descripción. No obstante, antes de comenzar con los descriptores CONFIGURATION, se debe especificar el descriptor DEVICE\_QUALIFIER, el cual dá información sobre otras velocidades de operación. Este descriptor es necesario debido a que el sistema implementado cumple con la versión 2.0 de la norma USB.
	
	\begin{lstlisting}[language={[x86masm]Assembler},backgroundcolor=\color{gray!30},numbers=left]
org (($ / 2) +1) * 2
DeviceQualDscr:
	db   DSCR_DEVQUAL_LEN   ;; Largo del descriptor
	db   DSCR_DEVQUAL   ;; Tipo de descriptor
	dw   0002H      ;; Versión de la norma (BCD)
	db   00H        ;; Clase de dispositivo
	db   00H         ;; Sub-clase de dispositivo
	db   00H         ;; Sub-sub-clase de dispositivo
	db   64         ;; Tamaño máximo de paquetes
	db   1         ;; Número de comunicaciones
	db   0         ;; Reservado
	\end{lstlisting}
	
	La norma USB especifica que la información que poseen los descriptores debe estar alineada por palabra (dos bytes), es decir, agrupada cada dos bytes y almacenada en direcciones de memoria par. Para asegurar esta condición, se recurre a la sentencia \verb|org (($/2)+1)*2|, la que se puede leer en la primera líneas del código mostrado anteriormente. La directiva \verb|org| le ordena al ensamblador la dirección específica de memoria en la que debe colocar las instrucciones precedentes. El símbolo (\$) representa al puntero que contiene el valor de memoria de la última instrucción ensamblada. Considerando que las direcciones de memoria son enteros y que las operaciones de ensamblador truncan decimales (no los redondean), la ecuación indicada entrega, como resultado, el entero par siguiente a la dirección actual. 

	Luego de esta directiva, se especifica el descriptor de tipo DEVICE\_QUALIFIER asegurando que se encontrará en una dirección par de memoria.
	
	Seguidamente, la norma indica que se debe detallar cada una de las configuraciones y las interfaces que se indican en los descriptores DEVICE y DEVICE\_QUALIFIER. En este caso, se especifican dos configuraciones: una de alta velocidad, indicada en el descriptor DEVICE y otra de máxima velocidad ({\it Full-Speed}), que se especifica en el descriptor DEVICE\_QUALIFIER. Se expone a continuación el descriptor CONFIGURATION de alta velocidad, unido al descriptor INTERFACE (línea 16 en adelante) y los descriptores ENDPOINT (a partir de la línea 27), que determinan en forma completa la configuración de Alta Velocidad.
	
	\begin{lstlisting}[language={[x86masm]Assembler},backgroundcolor=\color{gray!30},numbers=left]
org (($ / 2) +1) * 2
HighSpeedConfigDscr:
	db   DSCR_CONFIG_LEN	;; Largo del descriptor
	db   DSCR_CONFIG        ;; Tipo de descriptor
	db   (HighSpeedConfigDscr_End-HighSpeedConfigDscr) mod 256
							;; Largo total (LSB)
	db   (HighSpeedConfigDscr_End-HighSpeedConfigDscr)  /  256 
							;; Largo total (MSB)
	db   1      ;; Número de interfaces
	db   1      ;; Indice de configuración
	db   0		;; String de configuración
	db   80H	;; Atributos (b7->1, b6 - selfpwr, b5 - rwu)
	db   50		;; Consumo de potencia (div 2 ma)
	
	;; Alt Interface 0 Descriptor - Bulk IN
	db   DSCR_INTRFC_LEN   ;; Largo del descriptor
	db   DSCR_INTRFC       ;; Tipo de decriptor
	db   0                 ;; Indice de interfaz
	db   0                 ;; Indice de ajuste alternativo
	db   2                 ;; Número de extremos
	db   0ffH              ;; Clase de interfaz
	db   00H               ;; Sub-clsae de interfaz
	db   00H               ;; Sub-sub-clase de interfaz
	db   0                 ;; Indice de string descriptr de interfaz
	
	;; Isoc IN Endpoint Descriptor 
	db   DSCR_ENDPNT_LEN   ;; Largo del descriptor
	db   DSCR_ENDPNT       ;; Tipo de descriptor
	db   82H               ;; Extremo de entrada EP2
						   ;; b7 -> IN/OUT, b[4:0] -> dir
	db   ET_ISO            ;; Tipo de trasferencia
	db   00H               ;; Tamaño máximo de paquete (LSB)
	db   02H               ;; Tamaño máximo de paquete (MSB) 
	db   01H               ;; Intervalo de consulta
	
	;; Bulk OUT Endpoint Descriptor
	db   DSCR_ENDPNT_LEN   ;; Largo del descriptor
	db   DSCR_ENDPNT       ;; Tipo de descriptor
	db   08H               ;; Extremo de salida EP8
	db   ET_BULK           ;; Tipo de transferencia
	db   00H               ;; Tamaño máximo de paquete (LSB)
	db   02H               ;; Tamaño máximo de paquete (MSB)
	db   00H               ;; Intervalo de consulta
	
HighSpeedConfigDscr_End:
	\end{lstlisting}
	
	En la línea 12 del código anterior se debe colocar cuál es la fuente de la potencia que consume el dispositivo, es decir, de donde proviene la energía utilizada para el funcionamiento. El bit 7 debe estar siempre establecido a 1 por razones históricas de la norma USB~\cite{USBspec}. El bit 6 en 1 define que el dispositivo está energizado por una fuente propia. En el caso contrario, toma potencia del bus. El bit 5, por su parte, señala que el dispositivo tiene modo de baja energía y que es posible establecer el modo de funcionamiento de mayor consumo con un comando del Host. La línea 13 se informa cuánta potencia consume, lo que le brinda al Host la posibilidad de establecer un control de la potencia suministrada en el bus.
	
	El último campo del descriptor ENDPOINT, correspondiente al intervalo de consulta, se utiliza para establecer cada cuanto tiempo el Host debe asignar ancho de banda para transferencias isócronas.
	
	Luego de enviar la configuración de Alta Velocidad, se informa de la misma manera los descriptores que detallan la configuración del dispositivo, cuando trabaja con Velocidad Completa, cuyo código se observa a continuación.
	
	\begin{lstlisting}[language={[x86masm]Assembler},backgroundcolor=\color{gray!30},numbers=left]
org (($ / 2) +1) * 2
FullSpeedConfigDscr:
	db   DSCR_CONFIG_LEN	;; Largo del descriptor
	db   DSCR_CONFIG        ;; Tipo de descriptor
	db   (FullSpeedConfigDscr_End-FullSpeedConfigDscr) mod 256
							;; Largo total (LSB)
	db   (FullSpeedConfigDscr_End-FullSpeedConfigDscr)  /  256 
							;; Largo total (MSB)
	db   1      ;; Número de interface
	db   1      ;; Numero de configuraciones
	db   0      ;; Indice de string de configuración
	db   80H    ;; Atributos (b7 -<'1', b6 - selfpwr, b5 - rwu)
	db   50     ;; Requerimiento de pontencia(div 2 ma)
	
	;; Interface Descriptor
	db   DSCR_INTRFC_LEN ;; Largo del descriptor
	db   DSCR_INTRFC     ;; tipo de descriptor
	db   0               ;; Indice de interfaz
	db   0               ;; Indice de ajuste alternativo
	db   2               ;; Número de extremos
	db   0ffH            ;; Clase de interfaz
	db   00H             ;; Sub-clase de interfaz
	db   00H             ;; Sub-sub-clase de interfaz
	db   0               ;; Indice de string de interfaz
	
	;; Endpoint Descriptor
	db   DSCR_ENDPNT_LEN   ;; Largo del descriptor
	db   DSCR_ENDPNT       ;; Tipo de descriptor
	db   82H               ;; Dirección y sentido de extremo
	db   ET_ISO            ;; Tipo de extremo
	db   0FFH              ;; Tamaño máximo de paquete(LSB)
	db   03H               ;; Tamaño máximo de paquete(MSB)
	db   01H               ;; Intervalo de consulta
	
	;; Endpoint Descriptor
	db   DSCR_ENDPNT_LEN   ;; Largo del descriptor
	db   DSCR_ENDPNT       ;; tipo de descriptor
	db   08H               ;; Dirección y sentido de extremo
	db   ET_BULK           ;; Tipo de extremo
	db   040H              ;; Tamaño máximo de paquete (LSB)
	db   00H               ;; Tamaño máximo de paquete (MSB)
	db   01H               ;; Intervalo de consulta
	
FullSpeedConfigDscr_End:
	\end{lstlisting}

	Finalmente, el diseñador puede escribir mensajes en formato de cadena de caracteres, para lectura del usuario. En este trabajo solo se usan dos que sirven como ejemplo, pero no se profundizó más en el estudio de estos mensajes debido a que no son relevantes para los objetivos de este trabajo.
	
	\begin{lstlisting}[language={[x86masm]Assembler},backgroundcolor=\color{gray!30},numbers=left]
org (($ / 2) +1) * 2	
StringDscr:
	
StringDscr0:
	db   StringDscr0_End-StringDscr0    ;; Largo del descriptor
	db   DSCR_STRING					;; Tipo de descriptor
	db   09H,04H
StringDscr0_End:
	
StringDscr1:
	db   StringDscr1_End-StringDscr1    ;; Largo del descriptor
	db   DSCR_STRING					;; Tipo de descriptor
	db   'E',00							;; Mensaje
	db   'd',00
	db   'w',00
	db   'i',00
	db   'n',00
	db   ' ',00
	db   'B',00
	db   'a',00	
	db   'r',00
	db   'r',00
	db   'a',00
	db   'g',00
	db   'a',00
	db   'n',00
StringDscr1_End:
	
StringDscr2:
	db   StringDscr2_End-StringDscr2    ;; Largo del descriptor
	db   DSCR_STRING					;; Tipo de descriptor
	db   'L',00							;; Mensaje
	db   'a',00
	db   '-',00
	db   'T',00
	db   'e',00
	db   's',00
	db   'i',00
	db   's',00
StringDscr2_End:
	\end{lstlisting}
	
	
%	\linenumbers*
%	\lstinputlisting[language=asm,lastline=186]{\bridge}
%	\nolinenumbers