%TODO descripcion de paquetes y campos de lso paquetes
Cada vez que un dispositivo transmite información a través del bus, lo hace con un formato particular, establecido por el protocolo que dicta la norma USB. Cada uno de los fragmentos de información que envía un dispositivo, se llama paquete.\\

Un paquete está compuesto por diferentes campos. El sistema reconoce cada campo, decodifica su información e identifica cada paquete, con su emisor, el tipo de datos que envía, el sentido de circulación de la misma y puede, a su vez, corroborar que los datos transmitidos llegaron a destino en forma satisfactoria.\\

No existe un número infinito de campos y todos pueden resumirse en el presente documento. Sin embargo, solo se detallan a continuación los que el autor considera más relevantes para el objetivo del presente trabajo, quedando de lado algunos comandos inherentes a los hubs que conectan dispositivos de diferentes velocidades.\\

\subsection{Campos de paquetes}

	\subsubsection*{Identificador de paquete}
		El campo Identificador de paquete (PID del ingles {\it Packet Identifier})le da a conocer a los distintos dispositivos el tipo de información que va a contener el paquete. Por ejemplo, indica si el host está solicitando envío o recibo de datos, si está por enviar algun tipo de comunicación o si un dispositivo está transmitiendo los datos.\\
	
		Se compone de un campo de 8 bits, de los cuales 4 corresponden al identificador propiamente dicho y los otros cuatro es el complemento a uno de los mismos datos, permitiendo corroborar que no hubo perdida de datos.\\
		
		Existen 4 tipos de PID: Token, que antecede a cualquier transmisión y es emitido por el host; Data, indica paquetes que contienen datos transmitidos; Handshake, a través del cual los componentes del sistema se enteran si la comunicación fue efectiva o no y Special, cuya función no es de interés para este trabajo.\\
	
		A su vez, dentro de los PID Token, existe 4 tipos: IN, indica que se va a realizar una envío de datos desde un extremo al host; OUT, antecede a un envío de datos en el sentido contrario, es decir del host a un extremo; SETUP, que indica que se transmite un paquete con información de control y SOF (del inglés {Start of Frame)} que da una señal de sincronismo y control.\\
	
		Dentro de los PIDs Data, solo existen diferentes etiquetas que se usan dependiendo del tipo de transmisión. Los PIDs de Handshake contiene 4 mensajes diferentes: ACK para indicar que el mensaje fue recibido satisfactoriamente y NAK señala que no se pudo enviar o recibir, STALL significa que el extremo se detuvo y NYET de cuenta sobre demoras en la respuesta del receptor.\\
	
	\subsubsection*{Dirección}
		El campo de Dirección señala cuál es la función que debe responder o recibir alguna directiva emitida por el host. A su vez, se divide en dos subcampos: uno que indica un dispositivo y la segunda que señala el extremo específico con el cual desea comunicarse.\\

	\subsubsection*{Datos}
		Es el campo que contiene la información a transferir. Puede tener un largo de hasta 1024 bytes. Cada byte enviado se ordena con el bit menos significativo (LSb del inglés {\it Less Significative bit}) primero y el bit mas significativo (MSb por sus siglas en inglés) al final.

	\subsubsection*{Chequeos de redundancia cíclica}
		El campo de chequeo de redundancia cíclica (CRC) contiene verificadores para corroborar que no hubo perdida de información. Dependiendo de que tipo de paquete se esté transmitiendo, el CRC puede tener 5 o 16 bits. Los códigos generadores son representados por las ecuaciones 2.1 y 2.2 respectivamente:
	
		\begin{center}
			\begin{align}
				G(X)&=X^5+X^2+1\\
				G(X)&=X^16+X^15+X^2+1
			\end{align}
		\end{center}
	
\subsection{Formato de paquetes}
	
	\subsubsection*{Paquetes Token}
		Un paquete Token está conformado por los campos PID, Dirección y CRC-5 (CRC de 5 bits).\\
		
	\subsubsection*{Paquetes SOF}
		Un tipo especial de paquete de tipo Token es aquel cuyo PID indica SOF. Cada paquete contiene, en lugar de la dirección, un numero de 11 bits que indica el número de cuadros que han transcurrido, seguido de un código CRC-5.\\
		
		Cada \SI{1}{\milli\second} el host transmite un SOF e incrementar el numero de cuadro una unidad. En sistemas USB 2.0 de Alta velocidad, además, se transmiten 8 subcuadros de \SI{125}{\micro\second} por cada cuadro. Cada uno de estos subcuadros inicia con un paquete SOF. Sin embargo, el host no actualizará el numero de cuadros hasta pasado \SI{1}{\milli\second}.\\
		
	\subsubsection*{Paquetes Data}
		El paquete de datos iniciará con un PID que indique que es un paquete de este tipo, luego enviará los datos desde el LSb hasa el MSb y, finalmente, enviará un código CRC-16 (CRC de 16 bits de longitud).\\
		
	\subsubsection*{Paquetes Handshake}
		El paquete de Handshake solo envía un PID con información sobre si el mensaje fue recibido en forma correcta o no.\\