USB posee un esquema de bus maestro-esclavo, en forma de árbol cuyo nodo principal es el host. Es decir, la comunicación se realiza siempre a través de una sola línea de comunicación a la que se conectan todos los dispositivos que se necesite (dada el campo de direcciones provisto por la norma, 128 dispositivos como máximo). De esta manera, solo puede transmitir un dispositivo a la vez.\\

El acceso al bus, es decir, el acceso a la línea única compartida de comunicación, es administrado por un maestro. El maestro se encarga de solicitar a cada uno de los dispositivos su intervención. Posteriormente, el dispositivo debe responder al pedido del maestro. Este esquema es lo que se conoce como maestro-esclavo.\\

En un sistema USB no cualquier dispositivo puede ser maestro. Este rol lo cumple solo uno: una PC, o cualquier dispositivo con capacidad de llevar a cabo las tareas asignadas (que se detallan más adelante); denominado Host por la norma. La palabra {\it HOST} proviene del habla inglesa y se traduce como anfitrión, aunque en la jerga se conoce comunmente por su nombre en inglés.\\

\begin{figure}[]
	\centering
	\begin{tikzpicture}[scale=.87,>=latex,level 1/.append style={level distance = 2ex},level 2/.append style={level distance = 40}]
		\begin{scope}
			\begin{scope}[transform shape,grow = down]
				\node[] (host) {\it HOST} [
				sibling distance=60,
%				growth parent anchor=south, 
%				edge from parent fork down,
				]
				child{node[](l1r){Raíz}edge from parent[draw=none]
					child{node[](l2h1){Hub}
						child{node[](l3f1){Función}
						}
						child{node[](l3h1){Hub}
							child{node[](l4h1){Hub}
								child{node[](l5h1){Hub}
									child{node[](l6f1){Función}
									}
									child{node[](l6h1){Hub}
										child{node[](l7f1){Función}
										}
									}
								}
								child{node[](l5f1){Función}
								}
							}
						}
						child{node[] (l3f2) {Función}}
					}
					child{node[](l2f1){Función}}
					child{node[](l2h2){Hub}
						child{node[](l3h2){Hub}
							child{node[](l4f1){Función}
							}
							child{node[](l4f2){Función}
							}
						}
					}
				};
				\node[](l6)[left=of l6f1]{Grada 6};
				\node[](l7)at(l6|-l7f1){Grada 7};
				\node[](l5)at(l6|-l5h1){Grada 5};
				\node[](l4)at(l5|-l4h1){Grada 4};
				\node[](l3)at(l4|-l3f1){Grada 3};
				\node[](l2)at(l3|-l2h1){Grada 2};
				\node[](l1)at(l2|-l1r){Grada 1};
			\end{scope}
			\begin{scope}[dashed]
				\draw (l7) -| (l7f1.west);
				\draw (l6) -| (l6f1.west);
%					\draw(l6f1)--(l6h1);
				\draw (l5) -| (l5h1.west);
%					\draw(l5h1)--(l5h1);
				\draw (l4) -| (l4h1.west);
%					\draw(l4h1)--(l4f1);
%					\draw(l4f1)--(l4f2);
				\draw (l3) -| (l3f1.west);
%					\draw(l3f1)--(l3h1);
%					\draw(l3h1)--(l3f2);
%					\draw(l3f2)--(l3h2);
				\draw (l2) -| (l2h1.west);
%					\draw(l2h1)--(l2f1);
%					\draw(l2f1)--(l2h2);
				\draw (l1) -| (l1r.west);
			\end{scope}
		\end{scope}
	\end{tikzpicture}	
	\caption{Topología de un sistema USB}
	\label{fig:top}
\end{figure}

La topología del bus, cómo se observa en la Figura \ref{fig:top}, posee forma de árbol, es decir, puede ser pensada como una comunicación vertical, donde en el punto más alto se encuentra el Host. Siguiendo hacia abajo, el bus puede encontrar dos tipos diferentes de dispositivos: Funciones, cuyo rol es el de proveer una utilidad al sistema, como ser la de captura de imagen, reproducción de audio o el ingreso de comandos; y Hubs (concentradores o distribuidores), que se encargan de conectar una o más funciones al sistema. La norma USB establece gradas, en donde cada Hub introduce una nueva grada que contiene a las Funciones conectadas. Por cuestiones de restricciones temporales y tiempos de propagación en los cables, no se permiten más de 7 gradas, incluyendo al Host en la primera. Es decir, no se puede conectar más de 5 Hubs en cascada. La grada 7 sólo puede contener Funciones\cite{USBspec}.\\

\begin{figure}[b]
	\centering
	\includegraphics[width=0.28\textwidth]{usbconector}
	\caption{Tipos de conectores USB. Los tipo A deben ser usados en el extremo del Host y los tipo B hacia los periféricos\cite{USBHardwareWiki}}
	\label{fig:con}
\end{figure}

Cada uno de estos dispositivos diferentes, se inteconectan entre sí a través de cables y conductores específicos, diseñados en forma tal que no sea posible conectarlos en forma equivocada. Para cumplir con la norma, el Host debe tener siempre un zócalo compatible con conectores tipo A y los periféricos para enchufes de tipo B. Se observan las diferencias entre uno y otro en la Figura \ref{fig:con}