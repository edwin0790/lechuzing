Cada extremo presente en un dispositivo USB, puede estar configurado, en simultaneo, con un solo tipo de transferencias. Es importante, para el diseñador del dispositivo, entender y seleccionar el tipo de transferencia adecuada para cada uso debido a que, de ello depende las características que poseerán las comunicaciones que se efectúen.%\\

Existen cuatro tipos de transferencias definidas por la norma USB: Transferencias de Control, transferencias en masa, transferencias isocrónicas y transferencias de interrupción. Cada una de ellas tiene un propósito y características diferentes, las que se detallan a continuación.

\subsection{Transferencias de control}
	
	Las transferencias de control son utilizadas por el Host para configurar, emitir comandos y conocer el estado de los distintos dispositivos acoplados al bus. Se caracteriza por ser una comunicación de ráfagas, es decir, de corta duración, tener alta prioridad y ser no periódica. Habitualmente, son utilizadas para emitir comandos hacia los dispositivos, o bien, para conocer su estado. Sin embargo, esto no quiere decir que pueda ser empleada para transmitir mensajes que no sean específicamente de comando. Debido a la sensibilidad que los mensajes de control poseen para el sistema USB, estos están dotados del protocolo más estricto de chequeo, corrección y/o retransmisión de datos.%\\
	
	Las transferencias de control poseen dos o tres etapas en su ejecución. En la primera de ellas, el Host debe enviar un Paquete Token que indique SETUP, luego envía un paquete Data con \SI{8}{\byte} y esto es respondido por el dispositivo con un paquete Handshake indicando la recepción. Si hiciese falta enviar información extra, en una segunda etapa, el Host transmitirá un paquete Token indicando la necesidad de información. Luego, dependiendo del sentido de los datos solicitado, se enviará un paquete Data con hasta \SI{64}{\byte} más y el receptor responderá con un paquete Handshake. Finalmente, en la última etapa, se le permite al dispositivo informar su estado. Para ello, el Host le envía un paquete Token de solicitud de datos, luego la Función responderá con un paquete Data y el Host emitirá con un paquete Handshake, indicando si recibió o no la información.
	
\subsection{Transferencias en masa}
	Las transferencias en masa son usadas para transferir paquetes grandes en forma de ráfagas, en forma no periódica. Su utilidad consiste en que aprovecha al máximo cualquier espacio de ancho de banda disponible. Gracias al sistema de chequeo de errores, es posible solicitar retransmisiones, asegurando la integridad de la comunicación. Esta transferencia es ideal para comunicar cantidades relativamente grandes de datos que requieren una comunicación fidedigna a costa de sacrificar velocidad en los tiempos de entrega, por ejemplo, una impresora. En un bus que no posee un gran uso, los mensajes alcanzarán el destino en tiempos cortos. Sin embrago, cuando exista una gran cantidad de dispositivos conectados y el ancho del bus se encuentre congestionado, un mensaje largo puede verse demorado.%\\

	Cuando se lleva a cabo una operación de este tipo, el Host envía un paquete Token de tipo OUT cuando desea transmitir datos o IN si desea recibirlos, la dirección de la Función y su extremo. Luego, el emisor comunica un paquete Data, y finalmente, el receptor de la transferencia responde con un paquete Handshake. Una transferencia en masa ({\it bulk transfer}) puede poseer un tamaño máximo de \SI{512}{\byte} de datos por paquete transmitido.
	
\subsection{Transferencias isocrónicas}
	El término isócrono o isocrónico está referido a sistemas digitales sincrónicos con la particularidad de que se supone que sucede una cantidad determinada de sucesos en intervalos regulares de tiempo. Esto puede ser logrado compartiendo la misma fuente de sincronismo, o bien, sincronizando los relojes de cada componente. 
	
	En un sistema USB, el Host envía una señal SOF por cada cuadro de\SI{1}{\milli\second} y por cada subcuadro de  \SI{125}{\micro\second}, en los sistemas de alta velocidad. Es posible sincronizar sistemas que poseen fuentes de reloj diferentes a través de la captura de esta señal. Esto permite tener este comunicaciones de tipo isocrónico, aún con señales de reloj provenientes de fuentes diferentes.%\\
	
	La principal característica de las transferencias isocrónicas es que son periódicas y continuas entre el Host y las Funciones. Se utiliza este tipo de transferencias para comunicar datos que pierden validez cuando no son entregados en un tiempo establecido. Para lograr esto, el Host asigna una porción fija de ancho de banda por cada cuadro (\SI{1}{\milli\second}) a cada dispositivo que se comunique por transferencias de tipo isocrónicas. Gracias a que los datos pierden su validez a lo largo del tiempo, también los errores la pierden, por lo que no se prevé una retransmisión de los datos enviados por este sistema.%\\
	
	La ejecución de una transferencia isocrónica se da cuando el host envía un paquete Token con la dirección de un extremo de este tipo de transferencias. Luego, el emisor envía un paquete Data cuyo campo de datos puede poseer hasta \SI{1024}{\byte} y un CRC-16. Finalmente el receptor envía un paquete Handshake. Si, dado el caso, el receptor envía un Handshake indicando que el paquete no pudo ser recibido en forma correcta, el mensaje es descartado, sin existir una retransmisión posterior del mismo paquete.
	
\subsection{Transferencias de interrupción}
	Cuando se requiere de una comunicación cuya demora en la entrega de datos sea menor que un tiempo máximo y que, a su vez, posea una baja probabilidad de ocurrencia, el tipo de transferencia óptimo para utilizar, son las transferencias de interrupción. En este tipo de trasferencias, el Host consulta cada un periodo de tiempo determinado el estado de los extremos que se encuentran configurados para efectuar este tipo de transferencias. Para ello, envía un paquete Token, luego el emisor transmite un paquete de datos con hasta \SI{64}{\byte}, si se trata de dispositivos de velocidad completa, y \SI{1024}{\byte}, en el caso de una comunicación de alta velocidad. Finalmente, el receptor responderá con un paquete Handshake.
