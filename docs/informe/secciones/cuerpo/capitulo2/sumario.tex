En el presente capítulo se desarrolló y justificó la elección del controlador FX2LP como nexo entre la FPGA y la PC, brindando la conexión USB necesaria. Luego, se explicaron algunos componentes de la arquitectura implementada por Cypress a fin de proveer la comunicación USB. Finalmente, se detalló paso a paso cada uno de los componentes configurados, como así también el código desarrollado para dicho fin.

Además, se mostraron algunos detalles del framework provisto por Cypress y los encabezados necesarios para su utilización y se explicitaron los descriptores a través de los cuales se le informa al sistema las características de la comunicación que se implementa.

Finalmente se explicaron problemas identificados durante las pruebas y la depuración del programa de configuración, su solución y las herramientas utilizadas para hallarlas.