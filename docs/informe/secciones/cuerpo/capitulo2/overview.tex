El diseño de la norma USB busca resolver tres problemáticas interreacionadas, que son: La conexión de teléfonos con las PC, la facilidad de uso, es decir, que el usuario solo conecte su dispositvo y pueda utilizarlo, y la expansión de puertos disponibles para conectar periféricos. \cite{USBspec}\\

Para satisfacer estas tres demandas, la norma USB 2.0 busca alcanzar un conjunto de metas:

\begin{itemize}
	\item Expandir los puertos de PC destinados a periféricos y que posean facilidad de uso.
	\item Brindar una solución de bajo costo que permita tasas de transferencias de hasta 480 Mbps.
	\item Ser totalmente compatible con datos de voz, audio y video en tiempo real, es decir, que pueda trasmitir una conversación o video-llamada sin intermitencias.
	\item Poseer un protocolo flexible en el que convivan transferencias isocrónicas y mensajes asincrónicos.
	\item Integrar la norma en las tecnologías de dispositivos básicos.
	\item Comprender diferentes configuraciones de Pc y factores de forma.
	\item Proveer una interfaz estandar capaz de difundir rápidamente en los productos existentes en el mercado.
	\item Habilitar nuevas clases de dispositivos que aumenten las posibilidades de las PCs.
	\item Compatibilizar completamente con los dispositivos fabricados con versiones anteriores de la misma especificación.
\end{itemize}

%La consecución de estas metas, hizo que USB se transformara rápidamente en un estandar ampliamente difundido, unificando la multiplicidad de formatos de conexión existentes.\\

