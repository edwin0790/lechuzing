Dentro de un sistema USB existen tres tipos diferentes de dispositivos: Host, Hubs y Funciones. Cada uno de ellos tiene asignado un rol específico dentro de la comunicación. Se detallan a continuación las tareas pertinentes a cada uno de ellos.

\subsection{Host USB}
	El Host es quien comanda las comunicaciones. Este dispositivo debe tener capacidades de memoria y procesamiento necesarias para almacenar y ejecutar el software de control. A su vez, necesita de hardware que le permita llevar un monitoreo y control de los eventos que suceden en el bus. Entre las tareas que debe llevar a cabo, se encuentran:
	
	\begin{itemize}
		\item Detectar la conexión y desconexión de dispositivos.
		\item Administrar el flujo de los comandos de control con los diferentes dispositivos.
		\item Administrar el flujo de la información entre él (Host) y los diferentes dispositivos.
		\item Llevar estadísticas de actividad y estado del bus.
		\item Proveer potencia a los dispositivos conectados, cuando estos así lo requieran.
	\end{itemize}
	
	Debido a que las tareas que ejecuta el Host requiere una cantidad de recursos de almacenamiento y procesamiento, es usual que el sea una PC la que lleve el rol. El Host es quien inicia la comunicación con las Funciones. Las Funciones, a su vez, responden a lo que fue solicitado por el Host, cuando él lo indique.%\\
	
\subsection{Hubs USB}
	Un Hub USB tiene la función de proveer puertos al bus. El primer Hub esta incorporado en el Host y cada vez que se requiere más puertos a los cuales incorporar periféricos, se puede ir agregando a través de Hubs. Otra función importante es la de servir como interfaz entre dispositivos con diferentes velocidades, optimizando así el ancho de banda disponible para la comunicación.%\\
	
\subsection{Funciones USB}
	La norma define como Función a todo aquel dispositivo que se conecta al bus y brinda al Host la capacidad de realizar una nueva tarea. Por ejemplo, un teclado otorga un método de entrada adicional, un mouse permite manejar un puntero de la interfaz gráfica, un parlante y un micrófono posibilitan la emisión y recepción de sonidos, respectivamente. Cada una de estas utilidades, compone una Función USB. A su vez, un dispositivo que brinda más de una capacidad es visto por el Host como Funciones separadas conectadas a través de un Hub. Por ejemplo, si se piensa en unos auriculares con micrófono, aunque se presenten integrados en un mismo producto y tengan un único puerto de conexión al bus, el Host los considera como dos Funciones separadas. Las Funciones, desde un punto de vista de software, son independientes unas de otras, por lo que cuando un programa, llamado cliente, necesita utilizar una de ellas, puede acceder a ésta directamente sin conocer cuantas y cuales funciones diferentes existen en el bus.%\\
	
	Cada Función se compone de un conjunto de extremos. Un extremo es una porción de dispositivo identificable en forma unívoca\cite{USBspec}. Cada extremo tiene características definidas por el diseñador del sistema que deben estar adecuadas a los requerimientos de cada dispositivo. Los extremos tienen un solo sentido de comunicación y un tamaño máximo de mensaje a transmitir o recibir. Cuando se conecta al bus, un dispositivo debe enviar una descripción en donde consten sus extremos y las diferentes formas de configuración de cada uno, con el tipo de mensajes que soporta, el sentido de la comunicación, el tamaño, entre otros parámetros. Esta descripción se lleva a cabo través de lo que la norma llama descriptores.%\\
	
	Todo dispositivo debe contener un extremo con dirección cero dedicado exclusivamente al control de la Función por parte del Host. Debe, como mínimo, poder comunicarse a velocidad completa, es decir, con una señal de \SI{12}{\mega\bit\per\second} y, a su vez, responder a los comandos de control básicos cómo adquirir la dirección, recibir la configuración y enviar los descriptores del dispositivo y sus diferentes configuraciones. Dependiendo de los diferentes requerimientos, el dispositivo puede incorporar otros extremos (15 de entrada y 15 de salida como máximo). Cada extremo no-cero tiene diferente latencia, acceso al bus, ancho de banda, manejo de errores, tamaño máximo de paquete soportado y dirección.
	
	
%	punto de vista lógico, cada periférico posee canales únicos de comunicación con el host, llamados tuberías ({\it pipes} en el idioma inglés de la norma). Existen dos tipos de tuberías, las de control, por donde circulan mensajes propios del protocolo y sirven para la administración, configuración y gestión de las comunicaciones ; y las tuberías de ``chorro'' ({\it stream}) a través de las cuales circulan los mensajes con la información que se desea transmitir de un dispositivo a otro. El final de la tubería se llama extremo ({\it endpoint}) en el periférico y conectan cada extremo a un buffer en el host. Los periféricos poseen uno o más extremos. Cada extremo de un periférico, posee un tipo de transferencia asociado con una dirección de la información determinada. Esto quiere decir que un dispositivo de entrada y salida, debe poseer al menos dos extremos lógicos diferentes, uno para enviar datos al host y otro para recibirlos. Los tipos de transferencia, a su vez, determinan el ancho de bus asignado por el protocolo, la latencia, la tolerancia a errores en los datos enviados y el tamaño de los paquetes a enviar.\\
	