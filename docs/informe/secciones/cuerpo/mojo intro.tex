El principal objetivo de el presente Trabajo Integrador es el de proveer una comunicación USB para desarrollos basados en FPGA. Por esto mismo, es fundamental realizar un interfaz en el FPGA entre el desarrollo y la placa de interfaz.\\

Es por esto que se utilizó la placa MOJO. Dicha placa posee un FPGA Spartan-VI de Xilinx.\\

La placa MOJO es una placa inspirada en el concepto de prototipado rápido. Para ello incorpora una serie de pines a través de los cuales se le puede acoplar los diversos periféricos que uno necesite. Es posible conseguir en el mercado algunas otras placas, que los fabricantes denominan shields ({\it escudo} traducido al castellano), que encajan a la perfección en todos los pines y que contiene algún set de perfiéricos útiles. Estos shields también pueden ser diseñados por uno mismo copn los periféricos exactos que uno necesite. o bien con algunos cables se puede adaptar cuaquier periférico.\\

Al poseer un Spartan-VI, se tiene la posibilidad de elaborar sistemas de muy alta velocidad que permite al desarrollador de sensores y sistemas de adquisición de datos la implementación de circuitos que resuelvan problemas a la medida de los requerimientos.\\