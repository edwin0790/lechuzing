El principal objetivo de el presente Trabajo Integrador es el de proveer una comunicación USB para desarrollos basados en FPGA. Por esto mismo, es fundamental realizar un interfaz en el FPGA entre el desarrollo y la placa de interfaz.\\

Es por esto que se utilizó la placa MOJO. Dicha placa posee un FPGA Spartan-VI de Xilinx. Al poseer un Spartan-VI, se tiene la posibilidad de elaborar sistemas de muy alta velocidad que permite al desarrollador de sensores y sistemas de adquisición de datos la implementación de circuitos que resuelvan problemas a la medida de los requerimientos.\\

La placa MOJO es una placa inspirada en el concepto de prototipado rápido. Para ello incorpora una serie de pines a través de los cuales se le puede acoplar los diversos periféricos que uno necesite. Es posible conseguir en el mercado algunas otras placas, que los fabricantes denominan shields ({\it escudo} traducido al castellano), que encajan a la perfección en todos los pines y que contiene algún set de perfiéricos útiles. Estos shields también pueden ser diseñados por uno mismo con los periféricos exactos que uno necesite. o bien con algunos cables se puede adaptar cuaquier periférico.\\

Además de estos shields, los desarrolladores pensaron en que no sea necesario ninguna herramienta extra a la hora de programar la FPGA. Para ello, dotaron al sistema de un microcontrolador ATmega32U4 de Atmel y cargaron un programa bootloader, que se encarga de transferir la configuración del FPGA cargada desde una memoria flash incorporada al sistema con ese propósito particular y luego entra en modo esclavo, lo que permite al usuario, a posterior poder usar para debug el sistema USB que posee el microcontrolador.\\

Una vez llegado a este punto, el lector podría preguntar con toda razón ¿por qué es necesario realizar un sistema de comunicación USB extra, si ya cuenta con un microcontrolador que se encarga de dicho asunto? La respuesta a esta pregunta es simple y se basa en la capacidad del ancho de banda del sistema de comunicación que dispone la placa.\\

La linea de controladores ATmega posee USB 2.0 full-speed. Esto quiere decir que puede enviar datos a una tasa de 12Mbps. Además, la comunicación entre ambos chips se realiza via SPI (Serial Peripherical Interface, o en español Interface Serie de Periféricos), ofreciendo una velocidad de salida que puede resultar muy lenta a los fines de este trabajo. 