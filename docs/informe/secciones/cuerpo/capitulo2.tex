\chapter{USB 2.0}
  %INTORDUCCION

  \section{Introducción}
  Desde su primera especificación, lanzada en 1996, hasta la fecha la norma USB
  ha ido adquiriendo importancia y relevancia, debido a la aceptación que tuvo
  por parte de los fabricantes y los usuarios.

  USB, acrónimo de Universal Serial Bus (o Bus Serie Universal, en español) es
  una norma de comunicación pensada originalmente para comunicar periféricos con
  la computadora. Sin embargo, hoy en día brinda la posibilidad no solo de
  obtener diferentes dispositivos que se comunican de manera simple, robusta y
  veloz con cualquier PC, sino que, además permite al usuario obtener una fuente
  segura y confiable de energía para cualquier aparato electrónico que así lo
  requiera. Por este motivo, se está transformado en la única de comunicación de
  cualquier dispositivo de consumo masivo.

  Observando las especificaciones, USB es una comunicación que posee una
  complejidad mucho mayor que la que presentan los sistemas serie o paralelo
  tradicionales. Introduce nuevos términos, tales como extremos,
  isocronismo o enumeración, y le da nuevo significado a palabras ya conocidas,
  como configuración, interrupción o interfaz. No obstante esto, está muy bien
  separada por capas que facilitan el trabajo de la implementación del sistema
  y pueden ser vistas como sistemas aislados.

  A continuación se describen algunos detalles técnicos de esta norma, que son
  parte fundamental del trabajo realizado.

  \section{La especificación USB 2.0}
  En el año 2000, el comité conformado por seis de las más grandes compañías de
  la industria informática del momento, Compaq, Intel, Microsoft, NEC, Lucent,
  Hewlett-Packard y Philips; lanzó el documento final que describe las
  especificaciones de la versión USB 2.0. El documento puede ser descargado de
  forma gratuita desde el sitio: \href{http://www.usb.org}{www.usb.org}.

  \section{Conceptos Fundamentales}
  La norma USB 2.0 incorpora y modifica términos y conceptos que es conveniente
  conocer antes de una descripción más profunda de los elementos de interés para
  el presente trabajo.
    \itemize{begin}
    \item {\bf \host}: Es el dispositivo central de la comunicación USB. Solo
    puede existir un \host y es la PC, o el dispositivo que contenga el
    controlador USB. Es quien asigna las direcciones de cada dispositivo y
    solicita alguna acción a los \devices.
    \item {\bf \endpoint}: Es la porción direccionable de \device que sea emisor
    o destino de información respecto al \host.
    \item {\bf \device}: Es toda unidad lógica o física que realiza una función.
    \item {\bf \hub}: Es un \device que permite conectar una o varias funciones,
    aguas abajo
    \item {\bf \functions}
    \item {\bf }
    \itemize{end}
  \section{Topología USB}
  Cualquier sistema de comunicación USB está distribuido de manera centralizada
  con respecto al \host%BUSCAR SINONIMO

  % \section{Reseña histórica}
  %
  % El 11 de Noviembre de 1994, un comité de trabajo compuesto por ingenieros y
  % desarrolladores de siete de las grandes compañías de la industría electrónica
  % e informática del momento: Compaq, Hewlett-Packard, Intel, Lucent, Microsoft,
  % NEC y Philips; lanzó los lineamientos fundamentales de trabajo sobre un nuevo
  % estándar de interconexión entre computadoras personales y dispositivos
  % periféricos. El objetivo principal fué definir las especificaciones que se
  % tendrían en cuenta a fin de realizar una correcta comunicación entre una PC y
  % el teléfono, que sea facil de usar y que brinde una expansión de los puertos
  % que permita a los nuevos dispositivos periféricos que iban naciendo ser
  % conectados sin la necesidad de agrandar en forma infinita el hardware
  % necesario.\\
  %
  % Finalmente, el 15 de Enero de 1996 se publico la version 1.0 de la norma USB.
  % En su primera versión contemplaba dos modos de operación, low-speed de
  % \SI{1.5}{\mega\byte/\second} y full-speed, que opera a
  % \SI{12}{\mega\byte/\second}.\\
  %
  % Sin embargo, la gran evolución en el desempeño de las computadoras que se
  % comercializaban a gran escala, junto con el advenimiento de periféricos cada
  % vez más sofisticados que posibilitaban aplicaciones de mayor ancho de banda,
  % como por ejemplo la imagen digital, puso la necesidad de incorporar una mayor
  % velocidad y eficiencia a la norma USB. De esta forma, el 27 de Abril del 2000,
  % se lanzó la versión USB 2.0, que agregaba un modo más de comunicación,
  % high-speed, que posibilita el envío y recepción de hasta
  % \SI{480}{\mega\byte/\second}
  %
  % \section{Descripción del sistema}
  %
  %   Un sistema USB está compuesta por varios conceptos importantes, como ser la
  %   conexión fisica, potencia, protocolo, robustez de la comunicación, etc. Sin
  %   embargo, debido al interés del presente trabajo, se detallarás solo algunos
  %   aspectos importantes con respecto a las componentes de un sistema USB, su
  %   topología, el flujo de datos y algunos aspectos de la red, descripta
  %   mediante el modelo OSI
