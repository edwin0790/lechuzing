\chapter{Archivos de código para implementación en \acrshort{fpga}}
\label{ap:vhdl}

	El presente apéndice contiene el código desarrollado para la implementación de los sistemas en el \acrshort{fpga} Spartan 6 utilizados en este trabajo. Este código fue escrito en lenguaje \acrshort{vhdl}, salvo por el archivo de restricciones, que posee como lenguaje XST, desarrollado por Xilinx para dicho propósito. A su vez, se expone el resumen de síntesis, en donde constan los recursos del \acrshort{fpga} utilizados.
 	El contenido mostrado en este Apéndice contiene el código de los archivos:
 	\begin{itemize}
		\item \textbf{\nameref{ap:vhdl:iffx2}:} Implementación de la \acrfull{mef} a través de la cual se generan las señales de control que comandan la memoria \acrshort{fifo} del controlador FX2LP 
		
		\item \textbf{\nameref{ap:vhdl:meftb}:} Código utilizado para realizar la verificación funcional de la \acrshort{mef}
		
		\item \textbf{\nameref{ap:vhdl:top}:} Implementación del sistema de pruebas. En él se instancia la \acrshort{mef}, la configuración del \acrshort{pll} y la memoria \acrshort{fifo} generada a través de la herramienta \textit{Core Generator} de Xilinx Inc. 
		
		\item \textbf{\nameref{ap:vhdl:toptb}:} Código de verificación funcional del sistema de pruebas 

		\item \textbf{\nameref{ap:vhdl:ucf}:} Archivo de restricciones en donde se le indica al compilador la frecuencia de la señal de entrada al sistema y la asignación de los puertos de Entrada y Salida del sistema desarrollado a los pines del \acrshort{fpga} Spartan 6.

		\item \textbf{\nameref{ap:vhdl:sum}:} Tabla en donde se indica el consumo de recursos de \acrshort{fpga} por parte de la síntesis realizada a través del entorno de desarrollo ISE de Xilinx Inc.
	\end{itemize}


	\subsection*{fx2lp\_interface.vhd}
		\label{ap:vhdl:iffx2}
		\lstinputlisting[language=VHDL,numbers=left,numberstyle=\tiny,stepnumber=5,language=VHDL,basicstyle=\small,firstnumber=1]{secciones/extras/codigos/VHDL/fx2lp_interface.vhd}
	
	\subsection*{mef\_tb.vhd}
		\label{ap:vhdl:meftb}
		\lstinputlisting[language=VHDL,numbers=left,numberstyle=\tiny,stepnumber=5,firstnumber=1,basicstyle=\small,firstnumber=1]{secciones/extras/codigos/VHDL/mef_tb.vhd}
	
	\subsection*{fx2lp\_interface\_top.vhd}
		\label{ap:vhdl:top}
		\lstinputlisting[language=VHDL,numbers=left,numberstyle=\tiny,stepnumber=5,firstnumber=1,basicstyle=\small,firstnumber=1]{secciones/extras/codigos/VHDL/fx2lp_interface_top.vhd}


	\subsection*{top\_tb.vhd}
		\label{ap:vhdl:toptb}
		\lstinputlisting[language=VHDL,numbers=left,numberstyle=\tiny,stepnumber=5,firstnumber=1,basicstyle=\small,firstnumber=1]{secciones/extras/codigos/VHDL/top_tb.vhd}
	
	\subsection*{fx2lp\_interface\_top.ucf}
		\label{ap:vhdl:ucf}
		\lstinputlisting[numbers=left,numberstyle=\tiny,stepnumber=5,firstnumber=1,basicstyle=\small,firstnumber=1]{secciones/extras/UCF/fx2lp_interface_top.ucf}
	\pagebreak	
	\subsection*{Sumario de síntesis}
		\label{ap:vhdl:sum}
		\begin{table}[ht]
	\centering
	\resizebox{\textwidth}{!}{%
	\begin{tabular}{|l|l|l|l|}
		\toprule
		\rowcolor[rgb]{ .6,  .8,  1} \multicolumn{4}{|c|}{\textbf{fx2lp\_interface\_top Project Status (04/23/2020 - 15:52:42)}} \\
		\midrule
		\rowcolor[rgb]{ 1,  1,  .6} \textbf{Project File:} & \cellcolor[rgb]{ 1,  1,  1}Spartan6.xise & \textbf{Parser Errors:} & \cellcolor[rgb]{ 1,  1,  1}No Errors \\
		\midrule
		\rowcolor[rgb]{ 1,  1,  .6} \textbf{Module Name:} & \cellcolor[rgb]{ 1,  1,  1}fx2lp\_interface\_top & \textbf{Implementation State:} & \cellcolor[rgb]{ 1,  1,  1}Placed and Routed \\
		\midrule
		\rowcolor[rgb]{ 1,  1,  .6} \textbf{Target Device:} & \cellcolor[rgb]{ 1,  1,  1}xc6slx9-2tqg144 & \textbf{Errors:} & \multicolumn{1}{l|}{\cellcolor[rgb]{ 1,  1,  1}} \\
		\midrule
		\rowcolor[rgb]{ 1,  1,  .6} \textbf{Product Version:} & \cellcolor[rgb]{ 1,  1,  1}ISE 14.7 & \textbf{Warnings:} & \multicolumn{1}{l|}{\cellcolor[rgb]{ 1,  1,  1}} \\
		\midrule
		\rowcolor[rgb]{ 1,  1,  .6} \textbf{Design Goal:} & \cellcolor[rgb]{ 1,  1,  1}Balanced & \textbf{Routing Results:} & \cellcolor[rgb]{ 1,  1,  1}All Signals Completely Routed \\
		\midrule
		\rowcolor[rgb]{ 1,  1,  .6} \textbf{Design Strategy:} & \cellcolor[rgb]{ 1,  1,  1}Xilinx Default (unlocked) & \textbf{Timing Constraints:} & \cellcolor[rgb]{ 1,  1,  1}All Constraints Met \\
		\midrule
		\rowcolor[rgb]{ 1,  1,  .6} \textbf{Environment:} & \cellcolor[rgb]{ 1,  1,  1}System Settings & \textbf{Final Timing Score:} & \cellcolor[rgb]{ 1,  1,  1}0  (Timing Report) \\
		\bottomrule
	\end{tabular}}%
	\label{tab:pstatus}%
\end{table}%


{\tiny\tabcolsep=6.5pt
\begin{longtable}{|l|r|r|r|r|}
	\toprule \rowcolor[rgb]{ .6,  .8,  1} \multicolumn{5}{|c|}{\textbf{Device Utilization Summary}}\\
	\midrule
	\rowcolor[rgb]{ 1,  1,  .6} \textbf{Slice Logic Utilization} & \multicolumn{1}{c|}{\textbf{Used}} & \multicolumn{1}{c|}{\textbf{Available}} & \multicolumn{1}{c|}{\textbf{Utilization}} & \multicolumn{1}{c|}{\textbf{Note(s)}} \\
	\midrule
	\endfirsthead
	\toprule \rowcolor[rgb]{ .6,  .8,  1} \multicolumn{5}{|c|}{\textbf{Device Utilization Summary (cont.)}}\\
	\midrule
	\rowcolor[rgb]{ 1,  1,  .6} \textbf{Slice Logic Utilization} & \multicolumn{1}{c|}{\textbf{Used}} & \multicolumn{1}{c|}{\textbf{Available}} & \multicolumn{1}{c|}{\textbf{Utilization}} & \multicolumn{1}{c|}{\textbf{Note(s)}} \\
	\midrule
	\endhead
	\multicolumn{5}{|r|}{Continúa en la siguiente página}\\
	\bottomrule
	\endfoot
	\bottomrule
	\endlastfoot
	Number of Slice Registers & 83    & 11,440 & 1\%   &  \\
	\midrule
	\,\,\,\,\,Number used as Flip Flops & 79    &       &       &  \\
	\midrule
	\,\,\,\,\,Number used as Latches & 4     &       &       &  \\
	\midrule
	\,\,\,\,\,Number used as Latch-thrus & 0     &       &       &  \\
	\midrule
	\,\,\,\,\,Number used as AND/OR logics & 0     &       &       &  \\
	\midrule
	Number of Slice LUTs & 80    & 5,720 & 1\%   &  \\
	\midrule
	\,\,\,\,\,Number used as logic & 79    & 5,720 & 1\%   &  \\
	\midrule
	\,\,\,\,\,\,\,\,\,\,Number using O6 output only & 35    &       &       &  \\
	\midrule
	\,\,\,\,\,\,\,\,\,\,Number using O5 output only & 1     &       &       &  \\
	\midrule
	\,\,\,\,\,\,\,\,\,\,Number using O5 and O6 & 43    &       &       &  \\
	\midrule
	\,\,\,\,\,\,\,\,\,\,Number used as ROM & 0     &       &       &  \\
	\midrule
	\,\,\,\,\,Number used as Memory & 0     & 1,440 & 0\%   &  \\
	\midrule
	\,\,\,\,\,Number used exclusively as route-thrus & 1     &       &       &  \\
	\midrule
	\,\,\,\,\,\,\,\,\,\,Number with same-slice register load & 1     &       &       &  \\
	\midrule
	\,\,\,\,\,\,\,\,\,\,Number with same-slice carry load & 0     &       &       &  \\
	\midrule
	\,\,\,\,\,\,\,\,\,\,Number with other load & 0     &       &       &  \\
	\midrule
	Number of occupied Slices & 39    & 1,430 & 2\%   &  \\
	\midrule
	Number of MUXCYs used & 44    & 2,860 & 1\%   &  \\
	\midrule
	Number of LUT Flip Flop pairs used & 106   &       &       &  \\
	\midrule
	\,\,\,\,\,Number with an unused Flip Flop & 35    & 106   & 33\%  &  \\
	\midrule
	\,\,\,\,\,Number with an unused LUT & 26    & 106   & 24\%  &  \\
	\midrule
	\,\,\,\,\,Number of fully used LUT-FF pairs & 45    & 106   & 42\%  &  \\
	\midrule
	\,\,\,\,\,Number of unique control sets & 14    &       &       &  \\
	\midrule
	\,\,\,\,\,Number of slice register sites lost & \multirow{2}[2]{*}{77} & \multirow{2}[2]{*}{11,440} & \multirow{2}[2]{*}{1\%} & \multirow{2}[2]{*}{} \\
	\,\,\,\,\,\,\,\,\,\,to control set restrictions &       &       &       & \\
	\midrule
	Number of bonded IOBs & 37    & 102   & 36\%  &  \\
	\midrule
	\,\,\,\,\,Number of LOCed IOBs & 36    & 37    & 97\%  &  \\
	\midrule
	\,\,\,\,\,IOB Flip Flops & 1     &       &       &  \\
	\midrule
	\,\,\,\,\,IOB Latches & 16    &       &       &  \\
	\midrule
	Number of RAMB16BWERs & 1     & 32    & 3\%   &  \\
	\midrule
	Number of RAMB8BWERs & 0     & 64    & 0\%   &  \\
	\midrule
	Number of BUFIO2/BUFIO2\_2CLKs & 1     & 32    & 3\%   &  \\
	\midrule
	\,\,\,\,\,Number used as BUFIO2s & 1     &       &       &  \\
	\midrule
	\,\,\,\,\,Number used as BUFIO2\_2CLKs & 0     &       &       &  \\
	\midrule
	Number of BUFIO2FB/BUFIO2FB\_2CLKs & 1     & 32    & 3\%   &  \\
	\midrule
	\,\,\,\,\,Number used as BUFIO2FBs & 1     &       &       &  \\
	\midrule
	\,\,\,\,\,Number used as BUFIO2FB\_2CLKs & 0     &       &       &  \\
	\midrule
	Number of BUFG/BUFGMUXs & 3     & 16    & 18\%  &  \\
	\midrule
	\,\,\,\,\,Number used as BUFGs & 3     &       &       &  \\
	\midrule
	\,\,\,\,\,Number used as BUFGMUX & 0     &       &       &  \\
	\midrule
	Number of DCM/DCM\_CLKGENs & 0     & 4     & 0\%   &  \\
	\midrule
	Number of ILOGIC2/ISERDES2s & 16    & 200   & 8\%   &  \\
	\midrule
	\,\,\,\,\,Number used as ILOGIC2s & 16    &       &       &  \\
	\midrule
	\,\,\,\,\,Number used as ISERDES2s & 0     &       &       &  \\
	\midrule
	Number of IODELAY2/IODRP2/IODRP2\_MCBs & 0     & 200   & 0\%   &  \\
	\midrule
	Number of OLOGIC2/OSERDES2s & 1     & 200   & 1\%   &  \\
	\midrule
	\,\,\,\,\,Number used as OLOGIC2s & 1     &       &       &  \\
	\midrule
	\,\,\,\,\,Number used as OSERDES2s & 0     &       &       &  \\
	\midrule
	Number of BSCANs & 0     & 4     & 0\%   &  \\
	\midrule
	Number of BUFHs & 0     & 128   & 0\%   &  \\
	\midrule
	Number of BUFPLLs & 0     & 8     & 0\%   &  \\
	\midrule
	Number of BUFPLL\_MCBs & 0     & 4     & 0\%   &  \\
	\midrule
	Number of DSP48A1s & 0     & 16    & 0\%   &  \\
	\midrule
	Number of ICAPs & 0     & 1     & 0\%   &  \\
	\midrule
	Number of MCBs & 0     & 2     & 0\%   &  \\
	\midrule
	Number of PCILOGICSEs & 0     & 2     & 0\%   &  \\
	\midrule
	Number of PLL\_ADVs & 1     & 2     & 50\%  &  \\
	\midrule
	Number of PMVs & 0     & 1     & 0\%   &  \\
	\midrule
	Number of STARTUPs & 0     & 1     & 0\%   &  \\
	\midrule
	Number of SUSPEND\_SYNCs & 0     & 1     & 0\%   &  \\
	\midrule
	Average Fanout of Non-Clock Nets & 2.91  &       &       &  \\
\end{longtable}}


\begin{center}
	\centering
	\resizebox{\textwidth}{!}{%
	\begin{tabular}{|l|l|l|l|l|l|}
		\toprule
		\rowcolor[rgb]{ .6,  .8,  1} \multicolumn{6}{|c|}{\textbf{Detailed Reports}}\\
		\midrule
		\rowcolor[rgb]{ 1,  1,  .6} \textbf{Report Name} & \textbf{Status} & \textbf{Generated} & \textbf{Errors} & \textbf{Warnings} & \textbf{Infos} \\
		\midrule
		Synthesis Report & Current & jue 23. abr 15:51:44 2020 & 0     & 26 Warnings (0 new) & 14 Infos (10 new) \\
		\midrule
		Translation Report & Current & jue 23. abr 15:51:53 2020 & 0     & \multicolumn{1}{l|}{0} & 2 Infos (0 new) \\
		\midrule
		Map Report & Current & jue 23. abr 15:52:24 2020 &       & \multicolumn{1}{l|}{} &  \\
		\midrule
		Place and Route Report & Current & jue 23. abr 15:52:34 2020 & 0     & 6 Warnings (0 new) & 0 \\
		\midrule
		Power Report & \multicolumn{1}{l|}{} & \multicolumn{1}{l|}{} &       & \multicolumn{1}{l|}{} &  \\
		\midrule
		Post-PAR Static Timing Report & Current & jue 23. abr 15:52:40 2020 & 0     & \multicolumn{1}{l|}{0} & 3 Infos (0 new) \\
		\midrule
		Bitgen Report & Out of Date & mi\'e 1. abr 14:38:01 2020 & 0     & 4 Warnings (0 new) & 1 Info (0 new) \\
		\bottomrule
	\end{tabular}}%
\end{center}

\begin{center}
	\centering
	\begin{tabular}{c}
		Date Generated: 05/31/2020 - 15:05:40 \\
	\end{tabular}%
\end{center}%

