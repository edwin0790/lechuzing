\chapter{Códigos para implementaciónes en FPGA}
\label{ap:vhdl}
	\section*{Código de implementación de la MEF que controla la interfaz FX2LP}
	\lstinputlisting[language=VHDL,numbers=left,numberstyle=\tiny,stepnumber=5,language=VHDL,basicstyle=\small,firstnumber=1]{secciones/extras/codigos/VHDL/fx2lp_interface.vhd}
	
	\section*{Código de validación de la MEF}
	\lstinputlisting[language=VHDL,numbers=left,numberstyle=\tiny,stepnumber=5,firstnumber=1,basicstyle=\small,firstnumber=1]{secciones/extras/codigos/VHDL/mef_tb.vhd}
	
	\section*{Código de síntesis del sistema de prueba}
	\lstinputlisting[language=VHDL,numbers=left,numberstyle=\tiny,stepnumber=5,firstnumber=1,basicstyle=\small,firstnumber=1]{secciones/extras/codigos/VHDL/fx2lp_interface_top.vhd}


	\section*{Código de validación del sistema de prueba}
	\lstinputlisting[language=VHDL,numbers=left,numberstyle=\tiny,stepnumber=5,firstnumber=1,basicstyle=\small,firstnumber=1]{secciones/extras/codigos/VHDL/top_tb.vhd}
	
	\section*{Archivo de restricciones de usuario}
		Este archivo es en donde se detalla a que pin físico del FPGA se corresponde cada puerto. Este archivo fue utilizado para la implementación del sistema de pruebas.
		
		\lstinputlisting[numbers=left,numberstyle=\tiny,stepnumber=5,firstnumber=1,basicstyle=\small,firstnumber=1]{secciones/extras/UCF/fx2lp_interface_top.ucf}