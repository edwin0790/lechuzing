\usepackage{listings}
\usepackage{listingsutf8}
\lstset{inputencoding=utf8/latin1}
\lstset{tabsize=4}
\usepackage{caption}
\usepackage{subcaption}
\usepackage{enumerate}
\usepackage{lineno}
\usepackage{array}
\usepackage{booktabs}
\usepackage[titletoc]{appendix}
\usepackage{pdfpages}

\usepackage[binary-units]{siunitx}
\usepackage{titling}
\usepackage{soul}
\usepackage{adjustbox}

%change captions name of table using babel
\renewcommand{\spanishtablename}{Tabla}

\usepackage{hyperref}
\hypersetup{
	colorlinks=false,
	linkcolor=black,
	filecolor=black,
	urlcolor=black,
	pdftitle={\@title},
	bookmarks=true,
	hidelinks
}

\usepackage{tikz}
\usetikzlibrary{arrows,backgrounds,decorations.pathreplacing,fit,positioning,petri,babel,shapes,circuits.logic.mux,circuits.logic.US,shapes.gates.logic,trees,patterns}

%\tikzstyle{style} = [definition]
\tikzstyle{interior}=[rectangle,rounded corners,draw=black,minimum size=3em, text width=20,fill=white]
\tikzstyle{exterior} = [rectangle,draw=black,minimum size=40]
\tikzstyle{mode text} = [midway,sloped,text width=200]
\tikzstyle{contenedor} = [rectangle,draw=black]
\tikzstyle{core}=[interior]
\tikzstyle{perif} = [interior,minimum height=20]
\tikzstyle{buf}=[core, text width=53, align=center]
\tikzstyle{obuf} = [buf, node distance=.7]
\tikzstyle{env} = [fill=black!20]
\tikzstyle{simple}=[rectangle, draw=black, minimum height=220, text width=65,align=center]
\tikzstyle{moore} = [rectangle,draw=black, minimum height= 30,text width=80,align=left]
\tikzstyle{mealy} = [rectangle,rounded corners=12, draw=black, text width = 80,align=left,minimum height=40]
\tikzstyle{ask} = [diamond,text width=50,draw=black,align=center,]
\tikzstyle{bloque}=[exterior,align=center,minimum size=60,text width=60]
\tikzstyle{hub}=[draw=black,anchor=west,circle]
\tikzstyle{dev}=[draw=black,anchor=west,rectangle,rounded corners,text width=60,align=center]
\tikzstyle{pid}=[draw=black,align=center,rectangle,rounded corners,minimum width=20,minimum height=75,pattern=north east lines,pattern color=black!35]
\tikzstyle{dir}=[draw,rectangle,rounded corners,minimum width=20,minimum height=75,align=center]
\tikzstyle{data}=[draw=black,align=center,rectangle,rounded corners,minimum width=80,minimum height=75,fill=black!05]
\tikzstyle{crc}=[draw=black,align=center,rectangle,rounded corners,minimum width=20,minimum height=75,pattern=dots,pattern color=black!25]
\tikzstyle{field}=[draw=black,rectangle,rounded corners, minimum height=30,align=center]
\tikzstyle{chart}=[circle,draw=black,text width=50,align=center]
\tikzstyle{lineaExt} = [->,line width=5pt, >=latex, shorten <= 1]
\tikzstyle{lineaInt} = [->,line width=3pt,white, shorten >= 4.8, >=latex, shorten <= 2]


\newcommand{\epg}[3]{
	Buffer {#1}\\
	[38.5pt]EP{#2}\\
	[38.5pt]{#3} Bytes}

\newcommand{\ep}[3]{
	Buffer {#1}\\
	[8pt]EP{#2}\\
	[8pt]{#3} Bytes}

\newcounter{wavecount}

\newcommand{\newwave}[1]{
	\path (0,\value{wavecount}) node[text width=70,anchor=east,align=right]{#1} node[coordinate](t_cur){};
	\draw (0,\value{wavecount}+.3) --++(.2,0);
	\draw (0,\value{wavecount}-.3) --++(.2,0);
	\path (t_cur) --++(.3,0)node[coordinate](t_cur){};
	\addtocounter{wavecount}{-1}}

\newcommand*{\bit}[2]{
	\draw (t_cur) -- ++(0.1,.6*#1-.3) -- ++(#2-.2,0) -- ++(+.1,.3-.6*#1)
	node[coordinate] (t_cur) {};}

\newcommand*{\bitvector}[2]{
	\draw[] (t_cur) -- ++( .1, .3) -- ++(#2-.2,0) -- ++(.1, -.3)
	-- ++(-.1,-.3) -- ++(.2-#2,0) -- cycle;
	\path (t_cur) -- node[align=center] {#1} ++(#2,0) node[coordinate] (t_cur) {};}

\newcommand*{\graybitvector}[2]{
	\draw[fill=black!15] (t_cur) -- ++( .1, .3) -- ++(#2-.2,0) -- ++(.1, -.3)
	-- ++(-.1,-.3) -- ++(.2-#2,0) -- cycle;
	\path (t_cur) -- node[align=center] {#1} ++(#2,0) node[coordinate] (t_cur) {};}

\DeclareSIUnit[number-unit-product = {}]
\byte{B}

\graphicspath{{./imgs/}}

\renewcommand{\chaptermark}[1]{%
    \markboth{\thechapter.\ #1}{}}

\lstset{literate=
	{á}{{\'a}}1	{é}{{\'e}}1	{í}{{\'i}}1	{ó}{{\'o}}1	{ú}{{\'u}}1	{α}{{$\alpha$}}1	{β}{{$\beta$}}1	{γ}{{$\gamma$}}1	{ñ}{{\~n}}1
}

\renewcommand{\appendixpagename}{Apéndice}
\renewcommand{\labelitemi}{$\bullet$}
\renewcommand{\labelitemii}{$-$}

% Las próximas líneas son archivos que conviene que estén por aquí

	%bridge.c
\newcommand{\bridge}{secciones/cuerpo/capitulo3/bridge.c}