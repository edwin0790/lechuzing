\begin{frame}{Conclusiones}
	\begin{itemize}
		\only<1>{
			\item Se desarrollo un sistema de comunicación USB 2.0 que permite conectar una FPGA con una PC. El enlace pudo ser operado a 480 Mbps.
			\item El sistema elaborado permite leer y escribir datos en forma robusta, es decir, sin perder datos.
			\item Se adquirieron y entendieron conceptos importantes de la norma USB 2.0.
			\item Se seleccionó y se configuró el controlador FX2LP como interfaz USB.
			\item Se seleccionó el FPGA Spartan 6 de Xilinx Inc, incorporado en la placa de desarrollo Mojo v3.
		}
		\only<2>{
			\item Se desarrolló una máquina de estados finitos implementada con el FPGA que comanda la comunicación entre este dispositivo y la interfaz USB.
			\item El desarrollo de la máquina de estados ocupa el 1\% de las Slices del FPGA, dejando mucho recurso para cualquier otro sistema que se implemente.
			\item Se desarrolló un programa de computadoras que permite enviar y recibir datos a través del puerto USB.
			\item Se logró transmitir información desde y hacia la PC a una tasa efectiva de 9,12 Mbps.
		}
	\end{itemize}
\end{frame}
