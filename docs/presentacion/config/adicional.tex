\usepackage{siunitx}
%\usepackage{tabulary}
\usepackage{listings}
\usepackage{booktabs}
\usepackage{multirow}
\usepackage{pgfplots}
\usepackage{fancyvrb}
\usepackage{longtable}
\pgfplotsset{compat=1.3}
\usepackage{tikz}
\usetikzlibrary{arrows,babel,backgrounds,circuits.logic.US,decorations.pathreplacing,fit,patterns,petri,positioning,shapes,shapes.gates.logic,trees}

\newcommand\blfootnote[1]{%
	\begingroup
	\renewcommand\thefootnote{}\footnote{#1}%
	\addtocounter{footnote}{-1}%
	\endgroup
}

\tikzset{
	invisible/.style={opacity=0},
	visible on/.style={alt={#1{}{invisible}}},
	alt/.code args={<#1>#2#3}{%
		\alt<#1>{\pgfkeysalso{#2}}{\pgfkeysalso{#3}} % \pgfkeysalso doesn't change the path
	},
}

%\tikzstyle{style} = [definition]
\tikzstyle{interior}=[rectangle,rounded corners,draw=black,minimum size=3em, text width=20,fill=white]
\tikzstyle{exterior} = [rectangle,draw=black,minimum size=40]
\tikzstyle{mode text} = [midway,sloped,text width=200]
\tikzstyle{contenedor} = [rectangle,draw=black]
\tikzstyle{core}=[interior]
\tikzstyle{perif} = [interior,minimum height=20]
\tikzstyle{buf}=[core, text width=53, align=center]
\tikzstyle{obuf} = [buf, node distance=.7]
\tikzstyle{env} = [fill=black!20]
\tikzstyle{simple}=[rectangle, draw=black, minimum height=220, text width=65,align=center]
\tikzstyle{moore} = [rectangle,draw=black, minimum height= 30,text width=80,align=left]
\tikzstyle{mealy} = [rectangle,rounded corners=12, draw=black, text width = 80,align=left,minimum height=40]
\tikzstyle{ask} = [diamond,text width=50,draw=black,align=center,]
\tikzstyle{bloque}=[exterior,align=center,minimum size=60,text width=60]
\tikzstyle{hub}=[draw=black,anchor=west,circle]
\tikzstyle{dev}=[draw=black,anchor=west,rectangle,rounded corners,text width=60,align=center]
\tikzstyle{pid}=[draw=black,align=center,rectangle,rounded corners,minimum width=20,minimum height=110,pattern=north east lines,pattern color=black!35]
\tikzstyle{dir}=[draw,rectangle,rounded corners,minimum width=20,minimum height=110,align=center]
\tikzstyle{data}=[draw=black,align=center,rectangle,rounded corners,minimum width=80,minimum height=110,fill=black!05]
\tikzstyle{crc}=[draw=black,align=center,rectangle,rounded corners,minimum width=20,minimum height=110,pattern=dots,pattern color=black!25]
\tikzstyle{field}=[draw=black,rectangle,rounded corners, minimum height=30,align=center]
\tikzstyle{chart}=[circle,draw=black,text width=50,align=center]
\tikzstyle{lineaExt} = [->,line width=5pt, >=latex, shorten <= 1]
\tikzstyle{lineaInt} = [->,line width=3pt,white, shorten >= 4.8, >=latex, shorten <= 2]

\newcommand{\epg}[3]{
	Buffer {#1}\\
	[44pt]EP{#2}\\
	[44pt]{#3} Bytes}

\newcommand{\ep}[3]{
	Buffer {#1}\\
	[8pt]EP{#2}\\
	[8pt]{#3} Bytes}

\newcommand{\newwave}[1]{
	\path (0,\value{wavecount}) node[text width=45,anchor=east,align=right]{#1} node[coordinate](t_cur){};
	\draw (0,\value{wavecount}+.3) --++(.2,0);
	\draw (0,\value{wavecount}-.3) --++(.2,0);
	\path (t_cur) --++(.3,0)node[coordinate](t_cur){};
	\addtocounter{wavecount}{-1}}

\newcommand*{\bit}[2]{
	\draw (t_cur) -- ++(0.1,.6*#1-.3) -- ++(#2-.2,0) -- ++(+.1,.3-.6*#1)
	node[coordinate] (t_cur) {};}
%diagramas temporales
\newcommand*{\bitvector}[2]{
	\draw[] (t_cur) -- ++( .1, .3) -- ++(#2-.2,0) -- ++(.1, -.3)
	-- ++(-.1,-.3) -- ++(.2-#2,0) -- cycle;
	\path (t_cur) -- node[align=center] {#1} ++(#2,0) node[coordinate] (t_cur) {};}

\newcommand*{\graybitvector}[2]{
	\draw[fill=black!15] (t_cur) -- ++( .1, .3) -- ++(#2-.2,0) -- ++(.1, -.3)
	-- ++(-.1,-.3) -- ++(.2-#2,0) -- cycle;
	\path (t_cur) -- node[align=center] {#1} ++(#2,0) node[coordinate] (t_cur) {};}

\newcounter{wavecount}

