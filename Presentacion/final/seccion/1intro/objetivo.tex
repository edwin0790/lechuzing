\begin{frame}{Objetivos}
	\begin{center}
		\only<1>{\begin{tikzpicture}[>=latex,scale=1.2]
				\begin{scope}[transform shape]
					\node[bloque](pc){PC};
					\node[bloque](adq)[right =3 of pc]{Sistema de adquisicíon basado en FPGA};
					\draw[<->,ultra thick]([yshift=5]pc.east) -- node (data)[above]{Datos} ([yshift=5]adq.west);
					\draw[<->,ultra thick]([yshift=-20]pc.east) -- node (ctrl)[above]{Control} ([yshift=-20]adq.west);
				\end{scope}
				\begin{scope}[on background layer]
					\node[rounded rectangle,fill=blue!20,fit=(pc.east)(adq.west)(data)(ctrl)]{};
				\end{scope}				
			\end{tikzpicture}}

		\only<2>{\begin{tikzpicture}[scale=.7,>=latex]
				\begin{scope}
					\begin{scope}[transform shape,node distance=2]
						\node[bloque] (cy) {Interfaz USB};
%						\only<3->{\node[bloque]	(cy)				{Interfaz\\\alert{Cypress FX2LP}};}
						\node[bloque]	(fpga)	[right=of cy]{FPGA};
%						\only<4->{\node[bloque]	(fpga)	[right=of cy]{FPGA\\\alert{Xilinx Spartan VI}};}
						\node[bloque]	(pc) 	[left=of cy]	{PC};
%						\only<5->{\node[bloque]	(pc) 	[left=of cy]	{PC\\\alert{libusb}};}
						\draw[->,thick] (pc.15) -- node (usbd+) [above]	{D+} (pc.15 -| cy.west);
						\draw[->,thick] (cy.195) -- node (usbd-) [below]	{D-} (cy.195 -| pc.east);
						\draw[<->,thick] (cy.15) -- node (data) [above] {Datos} (cy.15 -| fpga.west);
						\draw[->,thick]  (fpga.195) -- node (ctrl) [below] {Control} (fpga.195 -| cy.east);
						\node[node distance=.4,text=blue] (usb text) [above=of usbd+] {USB};
					\end{scope}
					\begin{scope}
						\node[rectangle,rounded corners,draw=blue,dashed,fit=(usb text)(usbd+)(usbd-)(cy.south west)(pc.east)](usb){};
					\end{scope}
				\end{scope}
			\end{tikzpicture}}
	\end{center}


	\begin{itemize}
		\only<1>{\item Objetivo General
				\begin{itemize}
					\item Realizar una comunicación entre un FPGA y una PC mediante USB 2.0.
				\end{itemize}}
		\only<2>{\item Objetivos Particulares
				\begin{itemize}
					\item Seleccionar la interfaz USB.
					\item Comprender el funcionamiento del kit de desarrollo CY3684 utilizado como interfaz USB y el framework provisto por Cypress.
					\item Realizar la configuración de la interfaz USB.
					\item Sintetizar un circuito en VHDL que sea capaz de interactuar con la interfaz USB.
					\item Sintetizar circuitos de prueba para realizar una verificación funcional.
					\item Validar el funcionamiento.
				\end{itemize}}
	\end{itemize}
\end{frame}
