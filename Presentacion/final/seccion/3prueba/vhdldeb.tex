\begin{frame}{Depuración de la interfaz en VHDL}
	\begin{itemize}
		\item Se utilizó un registro para elaborar un eco de a dos bytes. Permitió corroborar el correcto funcionamiento de la lectura y escritura de las memorias FIFO de la interfaz de Cypress.
		\item Se utilizó una memoria FIFO generada con el software ``Core Generator'' de Xilinx para lograr un correcto funcionamiento de las señales de la interfaz escrita en VHDL.
		\item Además, se utilizó un PLL como generador de reloj, ya que si bien se pensaba hacer sincrónico, por errores de diseño esto no fue posible.
	\end{itemize}
\end{frame}

\begin{frame}{Estructura interna FPGA}
	\centering
%	\begin{tikzpicture}[scale=.58]
%		\begin{scope}[transform shape,node distance=5,>=latex,thick]
%			\node[simple]	(cypress)		[]	 			{FIFO Esclava};
%			\node[simple]	(master)	[right=of cypress]	{Maestro Externo};
%			\node[simple,minimum size=70]	(leer)		[right=of master.north west,anchor=north west] {Leer FIFO};
%			\node[simple,minimum size=70]	(escribir)	[right=of master.south west,anchor=south west]	{Escribir FIFO};
%			\node[simple,node distance=8]	(fifo)		[right=of master]	{FIFO Interna (XLNX core generator)};
%			
%			\draw[<->]	([yshift=4*110/6]cypress.east) --node [above]{IFCLK} ([yshift=4*110/6]master.west);
%			\draw[<->]	([yshift=3*110/6]cypress.east) --node [above]{FD[15:0]} ([yshift=3*110/6]master.west);
%			\draw[<-]	([yshift=2*110/6]cypress.east) --node [above]{FIFOADR[1:0]} ([yshift=2*110/6]master.west);
%			\draw[->]	([yshift=1*110/6]cypress.east) --node [above]{EP2\_EMPTY} ([yshift=1*110/6]master.west);
%			\draw[->]	([yshift=0*110/6]cypress.east) --node [above]{EP8\_FULL} ([yshift=0*110/6]master.west);
%			\draw[<-]	([yshift=-1*110/6]cypress.east) --node [above]{SLOE} ([yshift=-1*110/6]master.west);
%			\draw[<-]	([yshift=-2*110/6]cypress.east) --node [above]{SLWR} ([yshift=-2*110/6]master.west);
%			\draw[<-]	([yshift=-3*110/6]cypress.east) --node [above]{SLRD} ([yshift=-3*110/6]master.west);
%			\draw[<-]	([yshift=-4*110/6]cypress.east) --node [above]{PKTEND} ([yshift=-4*110/6]master.west);
%			
%			\draw[<-] (leer) -- node[above]{SLWR} (master.east |- leer);
%			\draw[<-] ([yshift=-1*80/7]leer.east) -- node[above]{EMPTY}([yshift=-1*80/7]fifo.west |- leer);
%			\draw[->] ([yshift=1*80/7]leer.east) -- node[above]{RD\_EN}([yshift=1*80/7]fifo.west |- leer);
%			
%			\draw[<-] (escribir) -- node[above]{SLRD} (master.east |- escribir);
%			\draw[<-] ([yshift=-1*80/7]escribir.east) -- node[above]{FULL}([yshift=-1*80/7]fifo.west |- escribir);
%			\draw[->] ([yshift=1*80/7]escribir.east) -- node[above]{WR\_EN}([yshift=1*80/7]fifo.west |- escribir);			
%			
%			\draw[<-]	([yshift=1*110/6]fifo.west) --node [above]{DIN[15:0]} ([yshift=1*110/6]master.east);
%			\draw[->]	([yshift=0*110/6]fifo.west) --node [above]{DOUT[15:0]} ([yshift=0*110/6]master.east);
%			\draw[<-]	([yshift=-1*110/6]fifo.west) --node [above]{VALID} ([yshift=-1*110/6]master.east);
%			
%			\node[node distance=.4] (fpga) [above=of leer] {FPGA};
%		\end{scope}
%		\begin{scope}[on background layer]
%			\node[rectangle,rounded corners,dashed,fit=(master)(leer)(fpga)(fifo)(escribir),draw=black]{};
%		\end{scope}
%	\end{tikzpicture}
\end{frame}

