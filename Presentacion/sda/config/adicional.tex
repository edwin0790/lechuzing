\usepackage{siunitx}
\usepackage{listings}
\usepackage{tikz}
\usetikzlibrary{arrows,babel,backgrounds,fit,patterns,petri,positioning,shapes}

\tikzstyle{core}=[
	rectangle,
	rounded corners,
	draw=black,
	minimum size=40]

\tikzstyle{perif}=[
	core,
	minimum height=20]

\tikzstyle{contenedor}=[
	rectangle,
	draw=black]

\tikzstyle{exterior}=[
	rectangle,
	draw=black,
	minimum size=40]

\tikzstyle{bloque}=[
	exterior,
	align=center,
	minimum size=60,
	text width=60]

\tikzstyle{pid}=[
	draw=black,
	align=center,
	rectangle,
	rounded corners,
	minimum width=20,
	minimum height=110,
	pattern=north east lines,
	pattern color=black!35]

\tikzstyle{dir}=[
	draw,
	rectangle,
	rounded corners,
	minimum width=20,
	minimum height=110,
	align=center]

\tikzstyle{data}=[
	draw=black,
	align=center,
	rectangle,
	rounded corners,
	minimum width=120,
	minimum height=110,
	fill=black!05]

\tikzstyle{crc}=[
	draw=black,
	align=center,
	rectangle,
	rounded corners,
	minimum width=20,
	minimum height=110,
	pattern=dots,
	pattern color=black!25]

\tikzstyle{buf}=[
	core,
	text width=55,
	align=center,
	fill=white]
	
\tikzstyle{obuf}=[
	buf,
	node distance=.7,
	fill=white]
	
\tikzstyle{env}=[
	fill=black!20]

\tikzstyle{moore}=[
	rectangle,
	draw=black,
	minimum height=30,
	text width=80,
	align=left]
	
\tikzstyle{mealy}=[
	rectangle,
	rounded corners=6,
	draw=black,
	text width=80,
	align=left,
	minimum height=40]

\tikzstyle{ask}=[
	diamond,
	text width=50,
	draw=black,
	align=center]
	
\tikzstyle{simple}=[
	rectangle,
	draw=black,
	minimum height=220,
	text width=65,
	align=center]

\newcommand{\epg}[3]{
	Buffer {#1}\\
	[44pt]EP{#2}\\
	[44pt]{#3} Bytes}

\newcommand{\ep}[3]{
	Buffer {#1}\\
	[8pt]EP{#2}\\
	[8pt]{#3} Bytes}

\newcommand{\newwave}[1]{
	\path (0,\value{wavecount}) node[text width=45,anchor=east,align=right]{#1} node[coordinate](t_cur){};
	\draw (0,\value{wavecount}+.3) --++(.2,0);
	\draw (0,\value{wavecount}-.3) --++(.2,0);
	\path (t_cur) --++(.3,0)node[coordinate](t_cur){};
	\addtocounter{wavecount}{-1}}

\newcommand*{\bit}[2]{
	\draw (t_cur) -- ++(0.1,.6*#1-.3) -- ++(#2-.2,0) -- ++(+.1,.3-.6*#1)
	node[coordinate] (t_cur) {};}
%diagramas temporales

